\documentclass{article}
%build with recipe latexmk
\usepackage[utf8]{inputenc}
\usepackage[T1]{fontenc}
\usepackage{textcomp}
\usepackage{fancyhdr}
\pagestyle{fancy}
\usepackage{tcolorbox}
\tcbuselibrary{theorems}
\usepackage{babel}
\usepackage{enumerate}
\usepackage{amsmath, amssymb, amsthm}
%\usepackage{a4wide}
\usepackage{float}
\usepackage{bbm}
\usepackage{tikz-cd}
\usepackage{tikz}
\usepackage{graphicx}
\usepackage{wrapfig}
\graphicspath{ {./images/} }
\usepackage{setspace}
\setstretch{1.1}
\usepackage{color}
\usepackage{hyperref}
\hypersetup{
    colorlinks=true, %set true if you want colored links
    linktoc=all,     %set to all if you want both sections and subsections linked
    linkcolor=black,  %choose some color if you want links to stand out
}

\theoremstyle{definition}
\newtheorem{theorem}{Theorem}[section]
\newtheorem{lemma}[theorem]{Lemma}
\newtheorem{cor}[theorem]{Corollary}
\newtheorem{prop}[theorem]{Proposition}
\newtheorem{example}{Example}[section]
\newtheorem{defn}{Definition}[section]

\title{Part II - Number Theory
    \\ \large
    Lectured by Prof. T. A. Fisher
}
\author{Artur Avameri}
\date{Michaelmas 2022}

% figure support
\usepackage{import}
\usepackage{xifthen}
\pdfminorversion=7
\usepackage{pdfpages}
\usepackage{transparent}
\newcommand{\incfig}[1]{%
    \def\svgwidth{\columnwidth}
    \import{./figures/}{#1.pdf_tex}
}

\pdfsuppresswarningpagegroup=1

\setcounter{section}{-1}
\begin{document}
\maketitle
\tableofcontents
\newpage

\section{Introduction}

\marginpar{06 Oct 2022, Lecture 1}

Books: 
\begin{itemize}
    \item A. Baker, \textit{A concise introduction to the theory of numbers}, CUP 1984
    \item N. Koblitz, \textit{A course in number theory \& cryptography}, Springer 1994
    \item H. Davenport, \textit{The higher arithmetic}, CUP 2008
\end{itemize}

Number theory studies the hidden and mysterious properties of the integers and the rational numbers.

It has always been an experimental science. Examining numerical data leads to \textbf{conjectures}, many of which are very old and still unproven today.

\begin{example}
    \begin{enumerate}[(i)]
        \item Let $N\ge 1$ be an integer of the form $8n+5, 8n+6$ or $8n+7$. Does there exist a right-angled triangle of area $N$, all of whose sides have rational length? We don't know.
        \item Let $\pi(x)$ be the number of primes less than or equal to $x$ and define $\text{li}(x) = \int_{2}^{x} \frac{dt}{\log t}$. Then for all $x \ge 3$, $|\pi(x) - \text{li}(x)| \le \sqrt{x}\log x$. This is in fact equivalent to the Riemann hypothesis.
        \item There are infinitely many twin primes. We now know there is an integer $N\le 246$ such that there are infinitely many pairs of primes the form $p, p+N$.
    \end{enumerate}
\end{example}

\newpage

\section{Euclid's algorithm and factoring}

\begin{defn}[Division algorithm]
    Given $a,b \in \mathbb{Z}$, with $b>0$, there exist $q,r \in \mathbb{Z}$ such that $a = qb + r$, and $0 \le r < b$.
\end{defn}

\textbf{Notation.} If $r=0$, then we write $b | a$, else $b\nmid a$. 

\begin{proof}
    Let $S = \{a - nb ~|~ n \in \mathbb{Z}\}$. This certainly contains integers $\ge 0$, so take the smallest one $r$. We claim $r < b$. Indeed, if not, then $r - b \ge 0$, contradicting minimality.
\end{proof}

Given $a_1, \ldots, a_n \in \mathbb{Z}$ not all zero, let $I = \{\lambda_1 a_1 + \ldots + \lambda_n a_n ~|~ \lambda_i \in \mathbb{Z}\}$.

\begin{lemma}
    $I = d \mathbb{Z}$ for some $d > 0$.
\end{lemma}
\begin{proof}
    $I$ certainly contains integers $\ge 0$. Let $d$ be the least positive element of $I$. We claim it works. Take $a \in I$, then $a = qd + r$ with $0 \le r < d$. But $r = a - qd \in I \implies r = 0$.
\end{proof}

\textbf{Remark.} We get from this that $d$ divides each $a_i$, and any common divisor of the $a_i$ must divide $d$. Why?

We write $d = \text{gcd}(a_1, \ldots, a_n)$ for the \textbf{greatest common divisor} (or \textbf{highest common factor}), or just use the shorthand $d = (a_1, \ldots, a_n)$.

\begin{cor}
    Let $a,b,c \in \mathbb{Z}$. Then there exist $x,y \in \mathbb{Z}$ such that $ax+by=c$ if and only if $(a,b) | c$.
\end{cor}

The division algorithm gives a very efficient way to compute $(a,b)$. Assume $a>b>0$. Apply the division algorithm recursively to get
\begin{align*}
    a &= q_1 b + r_1 & 0 \leq r_1 < b\\
    b &= q_2 r_1 + r_2 & 0 \leq r_2 < r_1\\
    r_1 &= q_3 r_2 + r_3 & 0 \leq r_3 < r_2\\
    &~ \vdots \\
    r_{k-2} &= q_k r_{k-1} + r_k & 0 \leq r_k < r_{k-1}, r_k \neq0\\
    r_{k-1} &= q_{k+1} r_k + 0
\end{align*}

\textbf{Claim.} 
    $r_k = (a,b)$.
Indeed, $(a,b) = (b, r_1) = (r_1, r_2) = \ldots = (r_{k-1}, r_k) = r_k$. This is called \textbf{Euclid's algorithm}.

\textbf{Remark.} If $d = (a,b)$, then by Lemma 1.2, there exist $r,s \in Z$ such that $ra + st = d$. Euclid's algorithm gives us a way to find $r$ and $s$.

\vspace{3mm}


In the following table, $x$ and $y$ stand for 34 and 25, and we then compute remainders as linear combinations of them.

We can use a trick here to speed this up: find each row as $q \cdot$ the row before it + the second row before it, then figure out signs at the end. (In fact, the minus signs zigzag down).

\begin{equation*}
    \begin{array}{r|cc}
        & x & y \\
        a=34 & 1 & 0 \\
        b=25 & 0 & 1 \\
        34 = 1 \cdot 25 + \color{red}{9}& 1 & -1 \\
        25 = 2 \cdot 9 + \color{red}{7}& -2 & 3 \\
        9 = 1 \cdot 7 + \color{red}{2}& 3 & -4 \\
        7 = 3 \cdot 2 + \color{red}{1}& -11 & 15 \\
    \end{array}
\end{equation*}

We hence get $-11 \cdot 34 + 15 \cdot 25 = 1$.

\begin{defn}
    An integer $n>1$ is \textbf{prime} if its only positive divisors are $1$ and $n$. Otherwise $n$ is \textbf{composite}.
\end{defn}
\begin{lemma} %1.2 in lectures
    Let $p$ be a prime, and $a,b \in \mathbb{Z}$. If $p | ab$, then $p \mid a$ or $p \mid b$.
\end{lemma}
\begin{proof}
    Assume $p \nmid a$. Then $(a,p) = 1$. By Lemma 1.2, $\exists r,s \in \mathbb{Z}$ such that $ra + sp = 1 \implies rab + spb = b$. Since $p \mid ab$, $p \mid b$ follows.
\end{proof}

\begin{theorem}[\textbf{Fundamental Theorem of Arithmetic}] 
    Every integer ${n>1}$ can be written as a product of primes. This representation is unique up to reordering.
\end{theorem}
\begin{proof}
    Existence is obvious. For uniqueness, suppose $n = p_1p_2 \ldots p_r = q_1 q_2 \ldots q_s$ for $p_i, q_i$ primes. We have $p_1 \mid q_1 q_2 \ldots q_r$, so by Lemma 1.5, $p_1 \mid q_j$ for some $j$, so $p_1 = q_j$. Now cancel these out and induct.
\end{proof}

\textbf{Remark.} If $m = \prod_{i=1}^{k} p_i^{\alpha_i}$ and $n = \prod_{i=1}^{k} p_i^{\beta_i}$ for $p_i$ distinct primes and $\alpha_i, \beta_i \ge 0$, then \[
(m,n) = \prod_{i=1}^{k} p_i^{\text{min}(\alpha_i,\beta_i)}.
\]

However, if $m$ and $n$ are large, it is more efficient to compute $(m,n)$ using Euclid's algorithm.

\marginpar{08 Oct 2022, Lecture 2}

\vspace{1mm}

Suppose we have some large positive integer $N$. An obvious algorithm for factoring $N$ is to trial divide by 2 and the odd integers up to $\sqrt{N}$.

\begin{defn}
    An algorithm with input a positive integer $N$ is \textbf{polynomial} or a \textbf{polynomial time} algorithm if it takes $\le c (\log N)^b$ \textbf{elementary operations}  for some constants $b$ and $c$. 
\end{defn}

\textbf{Remark.} An elementary operation is just adding/multiplying two numbers in $\{0,1,\ldots,9\}$.

\textbf{Remark.} "Polynomial" makes sense here as it takes $\log N$ digits to write $N$.

\vspace{1mm}

Polynomial algorithms are known for: 
\begin{itemize}
    \item Adding and multiplying integers (the usual way);
    \item Computing gcd's (via Euclid's algorithm);
    \item Detecting $n^{\text{th}}$ powers (compute $\sqrt[n]{}$ numberically and round)
    \item More remarkably, primality testing (Agrawal, Kayal, Saxena in 2002)
\end{itemize}

But trial division up to $\sqrt{N}$ is not polynomial.

\vspace{1mm}

\textbf{Fundamental question:} Is there a polynomial time algorithm for factoring? This is unknown.

Later in this course we study the distribution of the prime numbers, in particular the function $\pi(x)$, the number of primes $\le x$.

\begin{theorem}
    There are infinitely many prime numbers, i.e. ${\lim_{x \to \infty} \pi(x) \to \infty}$.
\end{theorem}
\begin{proof}
    Suppose there are only finitely many, say $p_1,\ldots, p_k$. Consider $N = \prod_{i=1}^{k} p_k + 1$. Then $N$ must be divisible by some prime other than the $p_i$, so we're done.
\end{proof}

All the largest known primes are of the form $2^n-1$ for $n$ a prime. These are called \textbf{Mersenne primes}. 51 of them are known, the largest being $2^{82589933}-1$.


\section{Congruences}

Fix a positive integer $n$ (the modulus).

\begin{defn}
    We say $a \equiv b \pmod{n}$, or that $a$ is congruent to $b \pmod{n}$ if $n$ divides $a-b$. 
\end{defn}

This defines an equivalence relation on $\mathbb{Z}$, and we write $\mathbb{Z}/n\mathbb{Z}$ for the set of equivalence classes. We can denote these by $a + n\mathbb{Z}$, or (more commonly) by ${a \pmod{n}}$. We can check that addition and multiplication are well-defined.

\vspace{1mm}

\textbf{Remark.} $n\mathbb{Z}$ is a subgroup/ideal of $\mathbb{Z}$ and $\mathbb{Z}/n\mathbb{Z}$ is the quotient group/ring.

\begin{lemma}
    Let $a \in \mathbb{Z}/n\mathbb{Z}$. Then the following are equivalent:
    \begin{enumerate}[(i)]
        \item $(a,n) = 1$
        \item $\exists b \in \mathbb{Z}$ such that $ab \equiv 1 \pmod{n}$
        \item $a$ is a generator for $\mathbb{Z}/n\mathbb{Z}$.
    \end{enumerate}
\end{lemma}
\begin{proof}
    (i)$\implies $ (ii): $(a,n)=1 \implies \exists r,s \in \mathbb{Z}$ such that $ra + sn = 1$, so $ra \equiv 1 \pmod{n}$.

    (ii) $\implies $ (i): $ab \equiv 1 \pmod{n} \implies ab + kn = 1$ for some $k \in \mathbb{Z} \implies {(a,b)=1}$.

    (ii) $\iff $(iii): $\exists b \in \mathbb{Z}$ s.t. $ab \equiv 1\pmod{n} \iff 1$ belongs to the subgroup of $\mathbb{Z}/n\mathbb{Z}$ generated by $a$.
\end{proof}

\textbf{Notation.} $(\mathbb{Z}/n\mathbb{Z})^{\times}$ is the group of \textbf{units} in $\mathbb{Z}/n\mathbb{Z}$, i.e. the elements with an inverse under multiplication.

\begin{defn}
    $\phi(n) = |(\mathbb{Z}/n\mathbb{Z})^{\times}|$ is called the \textbf{Euler totient function}. We also have $\phi(n) =|\{1 \le a \le n ~|~ (a,n) = 1\}|$. 
\end{defn}

\textbf{Remark.} $\mathbb{Z}/n\mathbb{Z}$ is a field $\iff \phi(n) = n -1 \iff n$ is prime. 

\begin{theorem}[Euler-Fermat theorem]
    If $(a,n)=1$, then $a^{\phi(n)} \equiv 1\pmod{n}$.
\end{theorem}
\begin{proof}
    Apply Lagrange's theorem to the group $G = (\mathbb{Z}/n\mathbb{Z})^{\times}$. Then for $a \in G$, its order divides $|G|= \phi(n)$.
\end{proof}
As a corollary:
\begin{theorem}[Fermat's little theorem]
    If $p \nmid a$, then $a^{p-1} \equiv 1\pmod{p}$.
\end{theorem}

\begin{lemma}%lemma 2.2 in lectures
    Let $G$ be a cyclic group of order $n$. We have \[
    |\{g \in G ~|~ \text{order}(g) = d\}| = \begin{cases}
        \phi(d) &\text{if }d \mid n \\
        0 &\text{otherwise}
    \end{cases}
    \]
    In particular, $\sum_{d \mid n}^{} \phi(d) = n$.
\end{lemma}
\begin{proof}
    WLOG let $G = (\mathbb{Z}/n\mathbb{Z}, +)$. We have $|\{g \in G ~|~ \text{order}(g) = n\}| \stackrel{(*)}{=} \phi(n)$ by Lemma 2.2. If $d \mid n$, say $n = dk$, then the elements of order dividing $d$ are the classes $0, k, 2k, \ldots, (d-1)k \pmod{n}$. These form a cyclic subgroup of order $d$. Applying $(*)$ to this cyclic subgroup shows that there are $\phi(d)$ elements of order $d$.
\end{proof}

\begin{example}
    Consider the simultaneous linear congruences $x \equiv 7 \pmod{10}$ and $x \equiv 3 \pmod{13}$. Suppose we can find $u,v \in \mathbb{Z}$ such that \[
    \begin{cases}
        u \equiv 1 \pmod{10}\\
        u \equiv 0 \pmod{13}
    \end{cases},
    \begin{cases}
        v \equiv 0 \pmod{10}\\
        v \equiv 1 \pmod{13}
    \end{cases}.
    \]
    Then $x = 7u + 3v$ is a solution. But $(10,13) = 1 \implies \exists r,s \in \mathbb{Z}$ such that $10r+13s = 1$, and we can just take $u = 13s, v = 10r$. To find $r,s$, we can use Euclid's algorithm to get $r=4, s= -3$, so $u = -39, v = 40$, and so $x \equiv 7 \cdot (-39) + 3 \cdot 40 \equiv 107 \pmod{130}$.
\end{example}

\marginpar{11 Oct 2022, Lecture 3}

\begin{theorem}[Chinese Remainder Theorem]
    Let $m_1,\ldots, m_k$ be pairwise coprime integers greater than 1. Let $a_1,\ldots, a_k \in \mathbb{Z}$. Let $M = m_1 m_2 \ldots m_k$. Then $\exists  x \in \mathbb{Z}$ satisfying 
    \[
    \begin{cases}
        x \equiv a_1 \pmod{m_1} \\
        \vdots \\
        x \equiv a_k \pmod{m_k}
        \end{cases}.
    \]
        Moreover, the solution is unique mod $M$.
\end{theorem}
\begin{proof}
    Uniqueness: Suppose $x \equiv x' \pmod{m_i} ~\forall i$. Then by considering the prime factorization of $x-x'$ and using the fact that the $m_i$ are pairwise coprime, we get $x \equiv x' \pmod{M}$.
    \vspace{1mm}
    
    Existence: Put $M_i = \frac{M}{m_i}$, so $(M_i, m_i) = 1 ~\forall i$. Hence we can find $u_i \in \mathbb{Z}$ such that $u_i M_i \equiv 1 \pmod{m_i} ~\forall i$. Let $x = \sum_{j=1}^{k} a_j u_j M_j$. Then $x \equiv a_iu_iM_i \equiv a_i \pmod{m_i}$.
\end{proof}
We can write this theorem in one ling using ring theory.
\begin{defn}
    Let $R_i = \mathbb{Z}/m_i\mathbb{Z}$, and define $R_1 \times \ldots \times R_k = \{(r_1,\ldots,r_k) ~|~ r_i \in R_i\}$ with addition and multiplication defined componentwise. This is a ring.
\end{defn}
\begin{theorem}[CRT, ring-theoretic version] %Theorem 2.3 in lectures
    Let $m_1,\ldots, m_k$ be pairwise coprime integers greater than 1 and put $M = m_1\ldots m_k$. Then the map 
    \begin{align*}
        \theta: \mathbb{Z}/M\mathbb{Z} \to \mathbb{Z}/m_1\mathbb{Z} \times \ldots \times \mathbb{Z}/m_k\mathbb{Z} \\
        a + M\mathbb{Z} \mapsto (a + m_1\mathbb{Z}, \ldots, a + m_k\mathbb{Z})
    \end{align*}
    is an isomorphism of rings.
\end{theorem}
\begin{proof}
    $\theta$ is a well defined ring homomorphism since $m_i | M ~\forall i$. Injectivity of $\theta$ follows from uniqueness in CRT, and surjectivity of $\theta$ follows from existence in CRT.
\end{proof}
\begin{cor}
    $\theta$ induces an isomorphism of groups under multiplication 
    \begin{align*}
        (\mathbb{Z}/m\mathbb{Z})^\times \cong (\mathbb{Z}/m_1\mathbb{Z})^\times \times \ldots \times (\mathbb{Z}/m_k\mathbb{Z})^\times \\
        a + M\mathbb{Z} \mapsto (a + m_1\mathbb{Z}, \ldots, a+m_k\mathbb{Z}).
    \end{align*}
\end{cor}
\textbf{Remark.} If $a \in \mathbb{Z}$, then $(a,M) = 1 \iff (a,m_i) = 1 ~\forall i$.

In particular, by looking at orders of the LHS and the RHS above, we get $\phi(M) = \phi(m_1)\ldots \phi(m_k)$, i.e. the Euler phi function is multiplicative.

\begin{defn}
    A function $f : \mathbb{Z}^+ \to \mathbb{C}$ is \textbf{multiplicative} if $f(m) = f(m)f(n)$ whenever $(m,n)=1$.
\end{defn}
\textbf{Examples}: 
\begin{itemize}
    \item $\phi(n) = |(\mathbb{Z}/n\mathbb{Z})^{\times}|$;
    \item $\tau(n) = \sum_{d \mid n}^{} 1$, the number of divisors of $n$;
    \item $\sigma(n) = \sum_{d \mid n}^{} d$, the sum of divisors of $n$;
    \item more generally, $\sigma_k(n) = \sum_{d \mid n}^{} d^k$, so $\sigma_0 = \tau$ and $\sigma_1 = \sigma$.
\end{itemize} 
To prove this:
\begin{lemma}\label{2.8} % Lemma 2.4 in lectures
    If $f : \mathbb{Z}^+ \to \mathbb{C}$ is multiplicative, then so is $g : \mathbb{Z}^+ \to \mathbb{C}$, defined by $g(n) = \sum_{d \mid n}^{} f(d)$.
\end{lemma}
\begin{proof}
    Let $m,n$ be coprime. Note that every divisor $d$ of $mn$ can be written as $d = d_1d_2$, where $d_1 \mid m$, $d_2 \mid  n$ and $(d_1,d_2)=1$. Thus \[
    g(mn) = \sum_{d \mid mn}^{} f(d) = \sum_{d_1 \mid m}^{} \sum_{d_2 \mid n}^{} f(d_1d_2) = \sum_{d_1 \mid m}^{} \sum_{d_2 \mid n}^{} f(d_1)f(d_2) = g(m)g(n).  
    \]
\end{proof}
\begin{lemma} % Lemma 2.5 in lectures
    \begin{enumerate}[(i)]
        \item For $p$ a prime, $\phi(p^k) = p^{k-1}(p-1) = p^k(1-\frac{1}{p})$.
        \item $\phi(n) = n \prod_{p \mid n}^{} (1-\frac{1}{p})$.
    \end{enumerate}
\end{lemma}
\begin{proof}
    (i): $\phi(p^k)$ counts the number of integers $a$ between $1$ and $p^k$ such that $(p^k,a) = (p,a) = 1$. So we have $p^a$ numbers, and we don't count the multiples of $p$, so $\phi(p^k) = p^k - p^{k-1}$.
    \vspace{1mm}
    
    (ii): Follows from the fact that $\phi$ is multiplicative. 
\end{proof}

\textbf{Alternative proof} that $\sum_{d \mid n}^{} \phi(d) = n$ (cf Lemma 2.6).
\begin{proof}
    Obviously the RHS is multiplicative. Since $\phi(n)$ is multiplicative, the LHS is multiplicative by Lemma 2.13, so it suffices to check for $n$ a prime power, say $n=p^k$. To this end, compute \[
    \sum_{d \mid p^k}^{} \phi(d) = \phi(1) + \phi(p) + \ldots + \phi(p^k) = 1 + (p-1) + (p^2-p) + \ldots + (p^k-p^{k-1}) = p^k.
    \]
\end{proof}

\subsection{Polynomial congruences}
Let $R = \mathbb{Z},\mathbb{Q}, \mathbb{Z}/n\mathbb{Z}$ (or more generally any commutative ring). Set $R[X] = \{\text{\textbf{polynomials} with coefficients in }R\}$, i.e. $a_n X^n + a_{n-1}X^{n-1} + \ldots + a_1 X + a_0$ for $a_i \in R$.

By definition, two polynomials are equal if and only if they have the same coefficients. We can check that $R[X]$ is a ring (with usual $+$ and $\times$).
\vspace{1mm}

\textbf{Warning}. The map $R[X] \to \{\text{functions }R \to R\}$ by $f \mapsto (\alpha \mapsto f(\alpha))$ is not always injective. For example, if $R=\mathbb{Z}/p\mathbb{Z}$ for $p$ a prime, and $f(X)=X^p - X$, then $f(\alpha)=0 ~\forall \alpha \in R$, but $f$ is not the zero function.
\vspace{1mm}

\textbf{Question.} Can we show thatif $f \in R[X]$ has degree $n$, then $f$ has at most $n$ roots in $R$?
\vspace{1mm}

\textbf{Answer.} No. For example, take $R = \mathbb{Z}/8\mathbb{Z}$, then $f(X)=X^2-1$ has 4 solutions in $\mathbb{Z}/8\mathbb{Z}$.

\marginpar{13 Oct 2022, Lecture 4}
\vspace{1mm}

Let $R=\mathbb{Z},\mathbb{Q}, \mathbb{Z}/n\mathbb{Z}$ (or any commutative ring).

We have a \textbf{division algorithm} on $R[X]$: 

Let $f,g \in R[X]$ and suppose the leading coefficient of $g$ is a unit. Then $\exists q, r \in R[X]$ such that $f(X)=Q(X)g(X)+r(X)$ and $\text{deg}(r)<\text{deg}(g)$.

\begin{proof}
    By induction on $\text{deg}(f)$. If $\text{deg}(f) < \text{deg}(g)$, take $q=0, r=f$. Otherwise, let $f(X) = a X^m + \ldots$ and $g(X) = b X^n + \ldots$ with $m \ge n$ and $b$ a unit.
    \vspace{1mm}
    
    Let $f_1(X) = f(X)- ab^{-1}X^{m-n}g(X)$. Then $\text{deg}(f_1) < \text{deg}(f)$, so by the induction hypothesis, $f_1(x)=q_1(x)g(x) + r_1(x)$ for some $q_1,r_1 \in R[X]$ and $\text{deg}(r_1) < \text{deg}(g)$. Now take $q(X) = ab^{-1}X^{m-n} + q_1(X)$ and $r=r_1$, so we're done.
\end{proof}

\begin{cor} % Corollary 2.6 in lectures
    If $f \in R[X]$ and $\alpha \in R$ is such that $f(\alpha)=0$, then $f(X) = (X-\alpha)f_1(X)$ for some $f_1 \in R[X]$. 
\end{cor}
\begin{proof}
    By the division algorithm, $f(X) = (X-\alpha) f_1(X) + r$ for some $r \in R$ (as $\text{deg}(r) < \text{deg}(X-\alpha)$). Plug in $X=\alpha$ to get $r=0$.
\end{proof}
\begin{defn}
    $R$ is an \textbf{integral domain} if $R$ has no zero divisors, i.e. $\alpha,\beta \in R$, $\alpha \beta=0 \implies \alpha=0$ or $\beta=0$.
\end{defn}
\textbf{Note.} Let $n>1$. Then $\mathbb{Z}/n\mathbb{Z}$ is an integral domain $\iff$ $n$ is prime.

\begin{theorem} % Theorem 2.7 in lectures
    If $R$ is an integral domain, then any polynomial $f \in R[X]$ of degree $n$ has at most $n$ roots.
\end{theorem}
\begin{proof}
    By induction on $n$, the degree of $f$. If $n=0$, then our polynomial is a nonzero constant and we're done. Now suppose $\exists \alpha \in R$ such that $f(\alpha)=0$ (otherwise we're done). By Corollary 2.15, $f(X) = (X-\alpha)f_1(X)$. Since $R$ is an integral domain, every root of $f$, except possibly $\alpha$ is a root of $f_1$. By induction, $f_1$ has at most $n-1$ roots, hence $f$ has at most $n$ roots and we're done.
\end{proof}
\begin{cor}[Lagrange's Theorem]
    Let $p$ be a prime and $a_0,\ldots,a_n \in \mathbb{Z}$ with $p \nmid a_n$. Then the congruence 
    \begin{align*}
        a_nx^n + a_{n-1}x^{n-1} + \ldots + a_1x + a_0 \equiv 0 \pmod{p}
    \end{align*}
    has at most $n$ solutions mod $p$.
\end{cor}
\begin{proof}
    Take $R=\mathbb{Z}/p\mathbb{Z}$ in Theorem 2.17.
\end{proof}
\textbf{Remark.} In this course, we will refer to the above theorem as Lagrange's Theorem.
\begin{example}
    Let $p$ be a prime. We will factor $X^{p-1}- 1 \pmod{p}$. Let $f(X) = X^{p-1}-1 - \prod_{a=1}^{p-1}(X-\alpha)$ in $\mathbb{Z}/p\mathbb{Z}[X]$. By Fermat's Little Theorem, $f$ has at least $p-1$ roots mod $p$. But $\text{deg}(f) < p-1$, since the $X^{p-1}$ terms cancel out, so by Lagrange's Theorem, $f = 0$, i.e. $X^{p-1} -1 = \prod_{a=1}^{p-1} (X-a)$ in $\mathbb{Z}/p\mathbb{Z}[X]$. Plugging in $X=0$ gives $(p-1)! \equiv -1 \pmod{p}$, i.e. Wilson's Theorem.
\end{example}
\begin{example}
    Working mod $7$, the powers of $3$ (starting from 0) are $1,3,2,6,4,5$. So $(\mathbb{Z}/7\mathbb{Z})^{\times}$ is cyclic, generated by $3$.
\end{example}
\begin{theorem}%Theorem 2.8 in lectures
    Let $p$ be a prime. Then $(\mathbb{Z}/p\mathbb{Z})^{\times}$ is cyclic.
\end{theorem}
\begin{proof}
    Let $S_d = \{a \in (\mathbb{Z}/p\mathbb{Z})^{\times} ~|~ \text{ord}(a) = d\}$. Suppose $S_d \neq \emptyset$, say $a \in S_d$. Then $1,a,a^2,\ldots,a^{d-1}$ are distinct elements in $\mathbb{Z}/p\mathbb{Z}$ and they are solutions of $x^d \equiv 1 \pmod{p}$. By Lagrange's theorem, this has at most $d$ solutions, and we found $d$ solutions, so those are all of them, i.e. $S_d \subseteq \{1,a,a^2,\ldots,a^{d-1}\}$. Note that the LHS is a cyclic group of order $d$, so this has $\phi(d)$ elements of order $d$.
    \vspace{1mm}
    
    We conclude that for every $d$, $|S_d| = 0$ or $|S_d| = \phi(d)$. In particular, ${|S_d| \le \phi(d)}$. Hence
    \begin{align*}
       p-1 \stackrel{(\star)}{=} \sum_{d \mid (p-1)}^{} |S_d| \le  \sum_{d \mid (p-1)}^{} \phi(d) = p-1,
    \end{align*}
    where $(\star)$ follows since we just count all the elements in $(\mathbb{Z}/p\mathbb{Z})^{\times}$. Hence ${|S_d| = \phi(d) ~\forall  d \mid (p-1)}$. In particular, $S_{p-1} \neq \emptyset$, i.e. $(\mathbb{Z}/p\mathbb{Z})^{\times}$ contains elements of order $p-1$, i.e. $(\mathbb{Z}/p\mathbb{Z})^{\times}$ is cyclic. 
\end{proof}
\textbf{Remark.} The same argument shows that any finite subgroup of the multiplicative group of a field is cyclic.
\begin{defn}
    An integer $a$ such that $a \pmod{n}$ generates $(\mathbb{Z}/n\mathbb{Z})^{\times}$ is called a \textbf{primitive root} mod $n$. 
\end{defn}
Theorem 2.21 showed that primitive roots exist mod $p$.
\begin{example}
    Let $p = 19$. Let $d$ be the order of $2$ in $(\mathbb{Z}/19\mathbb{Z})^{\times}$. We know $d \mid 18$, so we work out 
    \begin{align*}
        2^3 &\equiv 8 \pmod{19}\\
        2^6 &\equiv 7 \not\equiv 1 \pmod{19} \implies d \nmid 6\\
        2^9 &\equiv -1 \not\equiv 1 \pmod{19} \implies d \nmid 9,
    \end{align*}
    so $d=18$ and hence 2 is a primitive root mod $19$.
\end{example}
In general, $g \in \mathbb{Z}$ (coprime to $p$) is a primitive root mod $p$ if and only if $g^{\frac{p-1}{q}} \not\equiv 1 \pmod{p}$ $~\forall \text{primes }q \mid (p-1)$.

\marginpar{15 Oct 2022, Lecture 5}

\textbf{Remark.} The number of primitive roots mod $p$ is $\phi(p-1)=\phi(\phi(p))$.
\vspace{1mm}

Here are some (open) problems concerning primitive roots:
\begin{enumerate}[(i)]
    \item Artin's conjecture (1927) -- Let $a>1$ be an integer which is not a square. Then $a$ is a primitive root mod $p$ for infinitely many primes $p$. This is unknown for $a=2$. Hooley (1967) proved this assuming GRH. Heath-Brown (1986) proved that Artin's conjecture holds for at least one of $2,3$ or $5$. In fact, he proved something stronger: he proved the conjecture fails for at most 2 prime values of $a$.   
    \item How large is the smallest primitive root mod $p$? Burgess (1962) showed it is $\le c p^{1/4 + \epsilon} ~\forall  \epsilon>0$ and some constant $c=c(\epsilon)$. Shoup (1992) showed it is $\le c(\log p)^6$ assuming GRH.
\end{enumerate}

We now consider $\mathbb{Z}/p^n\mathbb{Z}$ for $n>1$. For $n \ge 3$, there is a surjective group homomorphism from $(\mathbb{Z}/2^n\mathbb{Z})^{\times} \to (\mathbb{Z}/8\mathbb{Z})^{\times} = \{\pm 1, \pm3\} \cong C_2 \times C_2$, so $(\mathbb{Z}/2^{n}\mathbb{Z})^{\times}$ is not cyclic (since generators map to generators).

\begin{theorem}% Theorem 2.9 in lectures
    Let $p$ be an odd prime. Then $(\mathbb{Z}/p^n\mathbb{Z})^{\times}$ is cyclic $~\forall n\ge 1$.
\end{theorem}
We divide the proof into 3 lemmas.
\begin{lemma}%Lemma 2.10 in lectures
    Let $n \ge 2$. Then $g$ is a primitive root mod $p^n$ if and only if the following two conditions hold: 
    \[
    \begin{cases}
        g \text{ is a primitive root mod }p \\
        g^{p^{n-2}(p-1)} \not\equiv 1 \pmod{p^n}
    \end{cases}.
    \]
\end{lemma}
\begin{proof}
    ($\implies $) is clear, as $\phi(p^n)=p^{n-1}(p-1)$.

    ($\impliedby$): Let $d$ be the order of $g$ in $(\mathbb{Z}/p^n\mathbb{Z})^{\times}$. Then $d \mid \phi(p^n) = p^{n-1}(p-1)$. Since $g^d \equiv 1 \pmod{p^n}$, we have $g^d \equiv 1 \pmod{p}$. Hence by assumption 1, we have $(p-1)\mid d$. Say $d = p^j (p-1)$ for some $0\le j \le n-1$. If $j\le n-2$, then this contradicts assumption 2. Hence $j=n-1$, so $d=\phi(p^n)$ is a primtive root mod $p^n$.
\end{proof}
Next we show $\exists g \in \mathbb{Z}$ satisfying conditions 1 and 2 in the case $n=2$.
\begin{lemma}
    $\exists g \in \mathbb{Z}$ a primitive root mod $p$ such that $g^{p-1} \not\equiv 1 \pmod{p^2}$.
\end{lemma}
\begin{proof}
    Let $g$ be a primtive root mod $p$. If $g^{p-1} \equiv 1\pmod{p^2}$, then consider $g+p$, which is still a primtive root mod $p$, but
    \[
    (g+p)^{p-1}=g^{p-1} + (p-1)g^{p-2}p + \ldots \equiv 1 + (p-1)g^{p-2}p \pmod{p^2},
    \]
    where the second term is not divisible by $p^2$, so $(g+p)^{p-1} \not\equiv 1 \pmod{p^2}$.
\end{proof}
Next we show that if $g$ is a primitive root mod $p^2$, then it is a primitive root mod $p^n ~\forall n\ge 2$.
\begin{lemma}
    If $g^{p-1} \not\equiv 1 \pmod{p^2}$, then $g^{p^{n-2}(p-1)} \not\equiv 1\pmod{p^n} ~\forall n\ge 2$. 
\end{lemma}
\begin{proof}
    By induction on $n$, the case $n=2$ being given. Suppose the result is true for $n$. By Euler-Fermat, $g^{p^{n-2}(p-1)} \equiv 1 \pmod{p^{n-1}}$, so $g^{p^{n-2}(p-1)} = 1 +b p^{n-1}$ for some $b \in \mathbb{Z}$, where $p \nmid b$ by the induction hypothesis. Taking $p^{\text{th}}$ powers gives 
    \begin{align*}
        g^{p^{n-1}(p-1)} = (1+bp^{n-1})^p = 1 + bp^n + {{p} \choose {2}}b^2p^{2(n-1)}+\ldots \equiv \\ 1 + bp^n + {{p} \choose {2}}b^2 p^{2(n-1)} \stackrel{\star}{\equiv } 1 + bp^n \pmod{p^{n+1}},
    \end{align*}
    where $\star$ follows since $p$ is odd, so $p \mid {{p}\choose{2}}$ (and also we use $3(n-1)\ge n+1$ and $2(n-1)+1\ge n+1$). Thus $g^{p^{n-1}(n-1)} \equiv 1 +bp^n \not\equiv 1 \pmod{p^{n+1}}$, so the result follows for $n+1$.
\end{proof}
This completes the proof of Theorem 2.24.

\begin{example}
    We saw $3$ is a primitive root mod $7$. We calculate $3^3 = -1 + 4\cdot 7$, so $3^6 \equiv 1 - 8\cdot 7 \not\equiv 1 \pmod{7^2}$. Hence $3$ is a primitive root mod $7^n ~\forall n$.
\end{example}

For the case $p=2$, let $G = \{a \in (\mathbb{Z}/2^n\mathbb{Z})^{\times} ~|~ a \equiv 1 \pmod{4}\}$. Then $(\mathbb{Z}/2^n\mathbb{Z})^{\times} \cong \{\pm 1\} \times G$ by $a+2^n\mathbb{Z} \mapsto \begin{cases}
    (1, a+2^n\mathbb{Z}) &\text{ if }a \equiv 1\pmod{4}\\
    (-1, -a+2^n\mathbb{Z}) &\text{ if }a \equiv 3\pmod{4}.
\end{cases}$

\textbf{Exercise.} Show that $G$ is cyclic (and generated by 5).

\textbf{Exercise.} For which $n$ is $(\mathbb{Z}/n\mathbb{Z})^{\times}$ cyclic? 

\marginpar{18 Oct 2022, Lecture 6}

\section{Quadratic residues}

Let $p$ be an odd prime and $a \in \mathbb{Z}$. By Lagrange's theorem, the congruence $x^2 \equiv a \pmod{p}$ has at most 2 solutions. If $a \not\equiv 0\pmod{p}$, then there are either 0 or 2 solutions. Indeed, if $x$ is a solution, then so is $-x \not\equiv x \pmod{p}$.

\begin{defn}
    Suppose $a \not\equiv 0\pmod{p}$. We say $a$ is a \textbf{quadratic residue} (QR) if $x^2 \equiv a\pmod{p}$ is soluble. We say $a$ is a \textbf{quadratic nonresidue} (NQR) if $x^2\equiv a \pmod{p}$ is unsoluble.
\end{defn}
\begin{example}
    $p=7$. $1,2,4$ are QRs and $3,5,6$ are QNRs.
\end{example}
\begin{lemma}
    Let $p$ be an odd prime. Then there are $\frac{p-1}{2}$ quadratic residues mod $p$ (and hence also $\frac{p-1}{2}$ quadratic nonresidues).
\end{lemma}
\begin{proof}[Proof 1.]
    Let $\mathbb{F}_p = \mathbb{Z}/p\mathbb{Z}$ (a field with $p$ elements). We show that the map $\mathbb{F}_p^\times \to \mathbb{F}_p^\times$ by $x \mapsto x^2$ is exactly 2--to--1. 
    
    Indeed, if $x^2 \equiv y^2 \pmod{p}$, then $p \mid x^2-y^2$, so $p \mid (x-y)$ or $p \mid (x+y)$, so $x \equiv \pm y \pmod{p}$.
\end{proof}
\begin{proof}[Proof 2.]
    Let $g$ be a primitive root mod $p$. Then $\mathbb{F}_p^\times = \{1,g,g^2,\ldots,g^{p-2}\}$.

    We claim that $g^i$ is a QR $\iff$ $i$ is even.
    \vspace{1mm}
    
    $\impliedby$ is clear. For $\implies$, suppose $g^i \equiv x^2\pmod{p}$. Then we can write $x = g^j \pmod{p}$, so $g^i \equiv g^{2j} \pmod{p} \implies i \equiv 2j \pmod{p-1}$. But $p-1$ is even, so $i = 2j + k(p-1)$ is even.
\end{proof}

\begin{defn}[Legendre symbol]
    Let $p$ be an odd prime, $a \in \mathbb{Z}$. Then
    \[
    \left(\frac{a}{p}\right) = \begin{cases}
        0 &\text{ if } p \mid a\\
        1 &\text{ if } a \text{ is a QR mod } p\\
        -1&\text{ if } a \text{ is a QNR mod } p
    \end{cases}
    \]
\end{defn}
\begin{theorem}[Euler's Criterion]
    Let $p$ be an odd prime and $a \in \mathbb{Z}$. Then 
    \[
        \left(\frac{a}{p}\right) \equiv a^{\frac{p-1}{2}} \pmod{p}.
    \]
\end{theorem}
\begin{proof}
    This is obvious if $p \mid a$, so suppose $(a,p)=1$. By Fermat's little theorem, $a^{p-1}\equiv 1\pmod{p} \implies a^{\frac{p-1}{2}} \equiv \pm 1 \pmod{p}$.

    If $\left(\frac{a}{p}\right) =1$, then $a \equiv b^2 \pmod{p}$ for some $b \in \mathbb{Z}$, but then $a^{\frac{p-1}{2}} \equiv b^{p-1} \equiv 1 \pmod{p}$. This gives $\frac{p-1}{2}$ solutions to the congruence $x^{\frac{p-1}{2}} \equiv 1 \pmod{p}$. By Lagrange's theorem, these are all the solutions. Hence if $\left(\frac{a}{p}\right) = -1$, then $a^{\frac{p-1}{2}}\not\equiv 1\pmod{p}$, so $a^{\frac{p-1}{2}} \equiv -1 \pmod{p}$ and we're done.
\end{proof}
\begin{cor} % Corolalry 3.2 in lectures
    $\left(\frac{ab}{p}\right) = \left(\frac{a}{p}\right)\left(\frac{b}{p}\right)$.
\end{cor}
\begin{proof}
    \[
    \left(\frac{ab}{p}\right) \equiv (ab)^{\frac{p-1}{2}} \equiv a^{\frac{p-1}{2}}b^{\frac{p-1}{2}} \equiv \left(\frac{a}{p}\right)\left(\frac{b}{p}\right) \pmod{p}.
    \]
    Since $0, \pm 1$ are distinct mod $p$, we have equality in the above.
\end{proof}
The corollary is equivalent to the statements:
\begin{itemize}
    \item $\mathcal{X}: \mathbb{F}_p^{\times} \to \{\pm 1\}$ by $a \mapsto \left(\frac{a}{p}\right)$ is a group homomorphism.
    \item \begin{enumerate}[(i)]
        \item QR $\cdot$ QR $=$ QR
        \item QR $\cdot$ QNR $=$ QNR
        \item QNR $\cdot$ QNR $=$ QR
    \end{enumerate}    
\end{itemize}
We can give an alternative proof for this:
\begin{enumerate}[(i)]
    \item $a \equiv x^2\pmod{p}, b \equiv y^2 \pmod{p} \implies ab \equiv (xy)^2 \pmod{p}$.
    \item If $a \equiv x^2$ and $ab \equiv z^2 \pmod{p}$, then $b \equiv (x^{-1}z)^2 \pmod{p}$, a contradiction.
    \item Suppose $a$ is a QNR. The map $\mathbb{F}_p^\times \to \mathbb{F}_p^\times$ by $x \mapsto ax$ is a bijection sending QRs to NQRs by (ii). By Lemma 3.1, it sends QNRs to QRs, done.
\end{enumerate}
\textbf{Remark.} We can also prove Euler's criterion using primitive roots.

\begin{cor}
    Let $p$ be a odd prime. Then 
    \[
        \left(\frac{-1}{p}\right) = (-1)^{\frac{p-1}{2}} = \begin{cases}
            1 &\text{ if } p \equiv 1\pmod{4}.\\
            -1 &\text{ if } p \equiv -1 \pmod{4}.
        \end{cases}
    \]
\end{cor}
In the next lecture, we show 
\[
\left(\frac{2}{p}\right) = (-1)^{\frac{p^2-1}{8}} = \begin{cases}
    1 &\text{ if } p \equiv \pm 1\pmod{8}.\\
    -1 &\text{ if } p \equiv \pm 3 \pmod{8}.
\end{cases}
\]
Let $p,q$ be distinct odd primes. The law of quadratic reciprocity gives a relation between $\left(\frac{p}{q}\right)$ and $\left(\frac{q}{p}\right)$. Generalizing this result (in many different ways) has been one of the main goals of number theory ever since.

\begin{theorem}[Law of quadratic reciprocity]
    Let $p,q$ be distinct odd primes. Then 
    \[
    \left(\frac{q}{p}\right) = \begin{cases}
        \left(\frac{p}{q}\right) &\text{ if } p\equiv 1\pmod{4} \text{ or }q \equiv 1 \pmod{4}. \\
        -\left(\frac{p}{q}\right) &\text{ if } p \equiv q \equiv 3 \pmod{4}.
    \end{cases}
    \]
\end{theorem}
\begin{example}
    $$\left(\frac{19}{73}\right) = \left(\frac{73}{19}\right) = \left(\frac{16}{19}\right) = 1.$$
\end{example}

\marginpar{20 Oct 2022, Lecture 7}

\textbf{Another proof of Fermat's little theorem:}

If $(a,p)=1$, then working mod $p$, the set $\{a,2a,3a,\ldots,(p-1)a\}$ is the same as $\{1,2,\ldots,(p-1)\}$. Taking the product gives $a^{p-1}(p-1)! \equiv (p-1)! \pmod{p} \implies a^{p-1}\equiv 1\pmod{p}$ as desired.
\vspace{1mm}

We can use the same idea to compute $a^{\frac{p-1}{2}}$ mod $p$:

\begin{lemma}[Gauss' Lemma]
    Let $p$ be an odd prime, let $a \in\mathbb{Z}$ be coprime to $p$, and put $m=\frac{p-1}{2}$. For $j=1,2,\ldots,m$ let $a_j$ be the unique integer such that
    \begin{enumerate}[(i)]
        \item $a_j \equiv ja \pmod{p}$
        \item $-m\le a_j \le m$.
    \end{enumerate}
    Then $\left(\frac{a}{p}\right) = (-1)^\nu$, where $\nu = \{\# 1\le j \le m ~|~ a_j < 0\}$.
\end{lemma}
\begin{proof}
    Consider $a_1,\ldots,a_m \in \{\pm1,\pm2,\ldots,\pm m\}$. Can any two of these be the same? No, since $a_i \equiv a_j \implies ai\equiv aj \implies i \equiv j \pmod{p}$. 

    Can any two differ by a sign? No, since $a_i \equiv -a_j \implies ia \equiv -ja \implies i \equiv -j \pmod{p}$.

    Hence $a_1,\ldots,a_m$ are $\pm1,\pm2, \ldots, \pm m$ in some order with some choice of signs. Taking the product gives $$a_1\ldots a_m \equiv (-1)^\nu 1\cdot \ldots \cdot m \pmod{p} \implies a^m m! \equiv (-1)^\nu m! \pmod{p}.$$
    So by Euler's criterion, $\left(\frac{a}{p}\right) \equiv  a^m \equiv (-1)^\nu \pmod{p}$.
\end{proof}

\begin{cor}
    Let $p$ be an odd prime. Then $$\left(\frac{2}{p}\right) = (-1)^{\frac{p^2-1}{8}}=  \begin{cases}
        1 &\text{ if } p \equiv  \pm 1 \pmod{8}.\\
        -1 &\text{ if } p \equiv  \pm 3 \pmod{8}.
    \end{cases}$$
\end{cor}
\begin{proof}
    Let $m =\frac{p-1}{2}$. Then $a_j = \begin{cases}
        2j &\text{ for } 1\le j\le \frac{m}{2}. \\
        2j-p &\text{ for } \frac{m}{2}< j \le m.
    \end{cases}$
    Hence $$\nu = m - \left\lfloor \frac{m}{2} \right\rfloor = \begin{cases}
        \frac{m}{2} \text{ if } m \text{ is even.}\\
        \frac{m+1}{2}\text{ if } m \text{ is odd.}
    \end{cases}$$
    It follows that $\left(\frac{2}{p}\right) = 1 \iff \nu$ is even $\iff m \equiv 0,3$ mod $4 \iff p \equiv \pm 1 \pmod{8}$.
\end{proof}
\vspace{1mm}

\begin{theorem}[Law of quadratic reciprocity]
    Let $p,q$ be distinct odd primes. Then \[
    \left(\frac{p}{q}\right)\left(\frac{q}{p}\right) \equiv (-1)^{\frac{p-1}{2}\frac{q-1}{2}}.
    \]
\end{theorem} 
\begin{proof}
    Step 1: Let $a,p,\nu$ be as in Gauss' Lemma (with $a\ge 1$).
    \vspace{1mm}
    
    Claim: $$\nu = \sum_{i=1}^{2n} (-1)^{i} \left\lfloor \frac{ip}{2a} \right\rfloor$$ where $n = \left\lfloor \frac{a}{2} \right\rfloor$. Moreover, $\frac{ip}{2a} \not\in \mathbb{Z} ~\forall~1\le i\le 2n$.
    \vspace{1mm}
    
    Proof: Consider all multiples of $a$ less than $\frac{ap}{2}$ ($=np$ or $(n+\frac{1}{2})p)$. Hence $\nu$ is the number of multiples of $a$ in the intervals $$\left[\frac{1}{2}p,p\right], \left[\frac{3}{2}p, 2p\right],\ldots, \left[(n-\frac{1}{2})p,np\right].$$ On dividing through by $a$, we see that $\nu$ is the number of integers in $$\left[\frac{p}{2a},\frac{2p}{2a}\right],\left[\frac{3p}{2a},\frac{4p}{2a}\right],\ldots,\left[\frac{(2n-1)p}{2a},\frac{2np}{2a}\right].$$ The end points are not in $\mathbb{Z}$, since the end points of the original intervals are not multiples of $a$. Hence $\# ([\alpha,\beta] \cap \mathbb{Z}) = \left\lfloor \beta \right\rfloor - \left\lfloor \alpha \right\rfloor$. This proves the claim.
    \vspace{1mm}
    
    Step 2: Let $p_1,p_2$ be primes and $a \in \mathbb{Z}$ coprime to $p_1p_2$. By Gauss' lemma, $\left(\frac{a}{p_i}\right) = (-1)^{\nu_i}$.
    \begin{enumerate}[(i)]
        \item Suppose $p_1 \equiv p_2 \pmod{4a}$. Then $\left\lfloor \frac{ip_1}{2a} \right\rfloor \equiv \left\lfloor \frac{ip_2}{2a} \right\rfloor \pmod{2}$. By Step 1, we have $\nu_1 \equiv \nu_2 \pmod{2}$. Hence $\left(\frac{a}{p_1}\right) = \left(\frac{a}{p_2}\right)$.
        \item Suppose $p_1 \equiv -p_2 \pmod{4a}$. Then $\left\lfloor \frac{ip_1}{2a} \right\rfloor \equiv \left\lfloor \frac{ip_2}{2a} \right\rfloor +1 \pmod{2}$. (We use the fact that if $\alpha \in \mathbb{R}/\mathbb{Z}$, then $\left\lfloor -\alpha \right\rfloor = -\left\lfloor \alpha \right\rfloor-1$). By Step 1, we again deduce $\left(\frac{a}{p_1}\right) = \left(\frac{a}{p_2}\right)$.
    \end{enumerate}
    \vspace{1mm}
    
    Step 3: Conclusion of the proof. 
    \begin{enumerate}[(i)]
        \item Suppose $p \equiv q \pmod{4}$, say $p = 4a +q$. Then $\left(\frac{p}{q} \right) = \left(\frac{4a+q}{q} \right) = \left(\frac{a}{q} \right)$, and $\left(\frac{q}{p} \right) = \left(\frac{p-4a}{p} \right) = \left(\frac{-1}{p} \right) \left(\frac{a}{p} \right) $. But $p \equiv q \pmod{4a}\stackrel{\text{Step 2(i)}}{\implies} \left(\frac{a}{p} \right) =\left(\frac{a}{q} \right)$, hence we conclude 
        \[
            \left(\frac{p}{q}\right)\left(\frac{q}{p}\right) \equiv (-1)^{\frac{p-1}{2}\frac{q-1}{2}}.
        \]
        \item Suppose $p \neq q \pmod{4}$, say $p+q = 4a$. Then $\left(\frac{p}{q} \right) = \left(\frac{4a-q}{q} \right) = \left(\frac{a}{q} \right)$ and $\left(\frac{q}{p}  \right) = \left(\frac{4a-p}{p} \right) = \left(\frac{a}{p} \right)$. But $p \equiv -q \pmod{4a} \stackrel{\text{Step 2(ii)}}{\implies} \left(\frac{a}{p} \right) = \left(\frac{a}{q} \right)$, so $\left(\frac{p}{q} \right) =\left(\frac{q}{p} \right)$, done.
    \end{enumerate}
\end{proof}

\marginpar{22 Oct 2022, Lecture 8}
\begin{example}
    Compute the Legendre symbol $\left(\frac{7411}{9283} \right)$. In fact, 7411 and 9283 are both prime. Hence
    \begin{align*}
        \left(\frac{7411}{9283}\right) = -\left(\frac{9283}{7411} \right) = -\left(\frac{1872}{7411} \right).  
    \end{align*}
    As $1872 = 2^4 \cdot 3^2 \cdot 13$, we get
    \begin{align*}
        -\left(\frac{1872}{8411} \right) = -\left(\frac{13}{7411} \right) = - \left(\frac{7411}{13} \right) = -\left(\frac{1}{13} \right) = -1.   
    \end{align*}
    Hence $7411$ is not a QR mod $9283$.
\end{example}
Recall that the Legendre symbol $\left(\frac{a}{p} \right) $ is only defined for $p$ an odd prime.
\begin{defn}
    Let $n$ be an odd positive integer, say $n=p_1\ldots p_k$ for $p_i$ (not necessarily distinct) odd primes. Let $a \in \mathbb{Z}$. We define the \textbf{Jacobi symbol} as \[
    \left(\frac{a}{n} \right) = \prod_{i=1}^{k} \left(\frac{a}{p_i} \right).
    \]
\end{defn}
\textbf{Remark.} If $(a,n)\neq 1$, then $\left(\frac{a}{n} \right)=0$.
\begin{prop}
    \begin{enumerate}[(i)]
        \item $\left(\frac{a}{n} \right) $ depends only on $a$ mod $n$.
        \item $\left(\frac{ab}{n} \right) =\left(\frac{a}{n} \right)\left(\frac{b}{n} \right)$ and $\left(\frac{a}{mn} \right) = \left(\frac{a}{m} \right) \left(\frac{a}{n} \right)$.
        \item $\left(\frac{-1}{n} \right) = (-1)^{\frac{n-1}{2}}$.
        \item $\left(\frac{2}{n} \right) =(-1)^{\frac{n^2-1}{8}}$.
    \end{enumerate}
\end{prop}
\begin{proof}
    \begin{enumerate}[(i)]
        \item Clear, since the Legendre symbol only depends on $a$ mod $p$.
        \item The first part follows since the Legendre symbol is totally multiplicative, and the second follows from the definition of the Jacobi symbol.
        \item This holds for $n=p$ a prime by previous results. We will now show that if they hold for odd integers $m,n$, then they hold for $mn$. But \[
        \left(\frac{-1}{mn} \right) = \left(\frac{-1}{m} \right) \left(\frac{-1}{n} \right) = (-1)^{\frac{m-1}{2}}(-1)^{\frac{n-1}{2}} \stackrel{\star}{=} (-1)^{\frac{mn-1}{2}},
        \]
        where we can check that $\star$ holds, since $(m-1)(n-1) \equiv  0 \pmod{4}$, which gives $mn-1 \equiv (m-1) + (n-1) \pmod{4}$.
        \item This is analogous to above, except we get \[
        (-1)^{\frac{m^2-1}{8}} (-1)^{\frac{n^2-1}{8}} = (-1)^{\frac{(mn)^2-1}{8}},
        \]
        since $(m^2-1)(n^2-1)\equiv 0 \pmod{16}$, so $(mn)^2-1 \equiv (m^2-1)+(n^2-1) \pmod{16}$.
    \end{enumerate}
\end{proof}
\begin{theorem}[Law of Quadratic Reciprocity for Jacobi Symbols]
    If $m,n$ are odd positive integers, then \[
    \left( \frac{m}{n} \right) = (-1)^{\frac{m-1}{2}\frac{n-1}{2}} \left(\frac{n}{m} \right).
    \]
\end{theorem}
\textbf{Remark.} If $(m,n) \neq 1$, this says $0=0$.
\begin{proof}
    Again, we deduce this from the corresponding result for the Legendre symbol. Assume $(m,n)=1$. Write $m = \prod_{i=1}^{k} p_i$ and $n=\prod_{j=1}^{l} q_j$ for $p_i,q_j$ (not necessarily distinct) primes.
    \vspace{1mm}
    
    Let $r$ count the number of $p_i$ with $p_i \equiv 3 \pmod{4}$ and $s$ count the number of $q_j$ with $q_j \equiv 3 \pmod{4}$. Then
    \begin{align*}
        \left(\frac{m}{n} \right) = \prod_{i=1}^{k} \prod_{j=1}^{l} \left(\frac{p_i}{q_j} \right) = \prod_{i=1}^{k} \prod_{j=1}^{l} (-1)^{\frac{p_i-1}{2}\frac{q_j-1}{2}} \left(\frac{q_j}{p_i} \right) = \\
        (-1)^{rs} \prod_{i=1}^{k} \prod_{j=1}^{l} \left(\frac{q_j}{p_i} \right) = (-1)^{rs} \left(\frac{n}{m}\right).  
    \end{align*}
    But $m \equiv 1 \pmod{4} \iff r$ is even, and $n \equiv 1 \pmod{4} \iff s$ is even, hence $(-1)^{rs} = (-1)^{\frac{m-1}{2}\frac{n-1}{2}}$. 
\end{proof}
\textbf{Remark.} The Jacobi symbol $\left(\frac{a}{n} \right) $ tells us surprisingly little about whether the congruence $x^2 \equiv a \pmod{n}$ is soluble. 

If $x^2 \equiv a\pmod{n}$ is soluble, then so is $x^2\equiv  a \pmod{p}$ for all primes $p \mid n$. So $\left(\frac{a}{p} \right) = 1 ~\forall p \mid n$, hence $\left(\frac{a}{n} \right) = 1$.

But the converse is false. For example, $\left(\frac{2}{15} \right) = \left(\frac{2}{3} \right) \left(\frac{2}{5} \right) = (-1)\cdot (-1) = 1$, yet $x^2 \equiv 2 \pmod{15}$ is not soluble.

The point of the Jacobi symbol is rather that it allows us to compute Legendre symbols without having to factor (except for removing powers of 2).
\begin{example}
    \[
    \left(\frac{33}{73} \right) = \left(\frac{73}{33} \right) = \left(\frac{7}{33} \right) = \left(\frac{33}{7} \right) = \left(\frac{5}{7} \right) = -1, 
    \]    
    so $33$ is not a QR mod $73$.
\end{example}

Three tricks to evaluate Legendre symbols:
\begin{example}
    \begin{enumerate}[(i)]
        \item $\sum_{a=1}^{p-1} \left(\frac{a}{p} \right) = 0$
        \item $\sum_{a=1}^{p-1} a \left(\frac{a}{p} \right) \equiv 0\pmod{p}$ if $p > 3$.
        \item $\sum_{a=1}^{p-1} \left(\frac{a(a+1)}{p} \right) = -1$.
    \end{enumerate}
\end{example}
\begin{proof}
    \begin{enumerate}[(i)]
        \item We have already done this since we have an equal number of QRs and QNRs. However, alternate proof:
        \vspace{1mm}
        
        Let $b$ be a QNR $\pmod{p}$. Then 
        \[
        \sum_{a=1}^{p-1} \left(\frac{a}{p} \right) = \sum_{a=1}^{p-1} \left(\frac{ab}{p} \right) = \left(\frac{b}{p} \right) \sum_{a=1}^{p-1} \left(\frac{a}{p} \right) = - \sum_{a=1}^{p-1} \left(\frac{a}{p} \right), 
        \]
        so $\sum_{a=1}^{p-1} \left(\frac{a}{p} \right) = 0$.
        \item Since $p>3$, we can choose $b \not\equiv 0, \pm 1 \pmod{p}$, whence \[
        \sum_{a=1}^{p-1} a \left(\frac{a}{p} \right) \equiv  \sum_{a=1}^{p-1} ab \left(\frac{ab}{p} \right) \equiv  \pm b \sum_{a=1}^{p-1} a \left(\frac{a}{p} \right) \pmod{p}.
        \]
        Since $b \not\equiv \pm 1 \pmod{p}$, we deduce $\sum_{a=1}^{p-1} a \left(\frac{a}{p} \right) \equiv 0 \pmod{p}$.
        \item  If $ab \equiv 1\pmod{p}$, then \[
        \left(\frac{a(a+1)}{p} \right) \equiv \left(\frac{a^2(1+b)}{p} \right) = \left(\frac{b+1}{p} \right).  
        \]
        Then \[
        \sum_{a=1}^{p-1} \left(\frac{a(a+1)}{p} \right) = \sum_{b=1}^{p-1} \left(\frac{b+1}{p} \right) = -1.
        \]
    \end{enumerate}
\end{proof}

\section{Binary quadratic forms}

\marginpar{25 Oct 2022, Lecture 9}

\textbf{Question.} Which numbers can be written as the sum of two squares?

Fermat gave an answer around 1630, and Euler published the first proof in 1749.

\begin{theorem}
    Let $N$ be a positive integer. Then $N$ is the sum of two squares if and only if every prime $p \equiv 3 \pmod{4}$ that divides $N$ divides it to an even power.
\end{theorem}
\begin{proof}[Proof of the easy direction.]
    $\implies$: Suppose $N=x^2+y^2$ and $p \mid N$, then $x^2 + y^2 \equiv 0 \pmod{p}$. If $p \equiv 3 \pmod{4}$, then $\left(\frac{-1}{p} \right) = -1$, so we must have $x\equiv y \equiv 0 \pmod{p}$. Then divide $N$ by $p^2$ and repeat until $p \nmid N$.
    \vspace{1mm}
    
    $\impliedby$: Since $(x^2+y^2)(z^2+t^2)=(xz-yt)^2+(xt+yz)^2$, it suffices to prove the result the case $N=p$ with $p=2$ or $p \equiv 1\pmod{4}$. $p=2$ is easy, but $p \equiv 1 \pmod{4}$ is a little more involved, and we will prove it a later lecture.
\end{proof}

Euler also studied $x^2+2y^2, x^2+3y^2$, etc. In this section we study \textbf{binary quadratic forms} with integer coefficients, i.e. $f(x,y)=ax^2+bxy+cy^2$ for $a,b,c \in \mathbb{Z}$.
\begin{defn}
    We say $f$ \textbf{represents} n if $f(x,y)=n$ for some $x,y \in \mathbb{Z}$.
\end{defn}
We may write $f$ as $(a,b,c)$ or in matrix notation as $$f(x,y) = \begin{pmatrix} x & y \end{pmatrix} \begin{pmatrix} a & \frac{b}{2} \\ \frac{b}{2} & c \end{pmatrix} \begin{pmatrix} x \\ y \end{pmatrix}.$$
\begin{example}
    $f(x,y)=x^2+y^2$ may be written as $(1,0,1)$ or $\begin{pmatrix} 1 & 0\\0&1 \end{pmatrix}$.

    $g(x,y) = 4x^2 + 12xy + 10y^2$ may be written as $(4,12,10)$ or $\begin{pmatrix} 4 & 6 \\6 & 10 \end{pmatrix}$.
\end{example}
Note that $g(x,y) = (2x+3y)^2+y^2 = f(2x+3y,y)$. Do $f$ and $g$ represent the same numbers? No, as $g$ only represents even numbers.
\vspace{1mm}

Let $X=2x+3y, Y=y$, then $$\begin{pmatrix} X\\Y \end{pmatrix} = \begin{pmatrix} 2 & 3\\0 & 1 \end{pmatrix}\begin{pmatrix} x \\y \end{pmatrix} \implies \begin{pmatrix} x\\y \end{pmatrix} = \frac{1}{2}\begin{pmatrix} 1 & -3 \\0 & 2 \end{pmatrix}\begin{pmatrix} X \\ Y \end{pmatrix}.$$
Note that we can have $X,Y \in \mathbb{Z}$, yet $x,y \not\in \mathbb{Z}$. 

\begin{defn}
    A \textbf{unimodular substitution} is one of the form $X = \alpha x + \gamma y, Y = \beta X + \delta Y$ where $\alpha,\beta,\gamma,\delta \in \mathbb{Z}$ and $\alpha \delta - \beta \gamma = 1$. 
\end{defn}
\begin{defn}
    Two BQFs $f$ and $g$ are \textbf{equivalent}, written $f\sim g$, if they are related by a unimodular substitution.
\end{defn}
Exercise: Check $\sim$ is an equivalence relation (this is on the example sheet).

\textbf{Note.} Equivalent forms represent the same integers.
\vspace{1mm}

The group $SL_2(\mathbb{Z}) = \left\{\begin{pmatrix} \alpha &\beta \\ \gamma& \delta \end{pmatrix} ~|~ \alpha,\beta,\gamma,\delta \in \mathbb{Z}, \alpha \delta - \beta \gamma = 1\right\}$ acts on the set of BQFs via $\begin{pmatrix} \alpha &\beta \\ \gamma&\delta \end{pmatrix} : f(x,y) \mapsto f(\alpha x + \gamma y, \beta x + \delta y)$. The equivalence classes are the orbits of this action.

To check a group action, we need to check 
\begin{enumerate}[(i)]
    \item $\begin{pmatrix} 1 & 0\\0&1 \end{pmatrix} f = f$, which is true.
    \item $\sigma(\tau f) = (\sigma \tau)f ~\forall \sigma, \tau \in SL_2(\mathbb{Z})$.
\end{enumerate}
Suppose $f=(a,b,c)$ and $g=(a',b',c')$ are equivalent, say $g = \sigma f$ for $\sigma = \begin{pmatrix}  \alpha & \beta\\ \gamma &\delta \end{pmatrix}$. Then 
\begin{align*}
    g(x,y)=f(\alpha x + \gamma y, \beta x + \delta y) = \begin{pmatrix} \alpha x + \gamma y & \beta x + \delta y \end{pmatrix} \begin{pmatrix} a & \frac{b}{2} \\ \frac{b}{2} & c \end{pmatrix} \begin{pmatrix} \alpha x  + \gamma y \\ \beta x + \delta y \end{pmatrix} = \\
    \begin{pmatrix} x & y \end{pmatrix}\begin{pmatrix} \alpha & \beta \\ \gamma & \delta \end{pmatrix}\begin{pmatrix}  a & \frac{b}{2} \\ \frac{b}{2} & c \end{pmatrix}\begin{pmatrix} \alpha & \gamma \\ \beta & \delta \end{pmatrix}\begin{pmatrix} x \\ y \end{pmatrix}.
\end{align*}
Hence $\begin{pmatrix} a' & \frac{b'}{2} \\ \frac{b'}{2} & c' \end{pmatrix} = \sigma \begin{pmatrix} a & \frac{b}{2} \\ \frac{b}{2} & c \end{pmatrix} \sigma^\top$. Call this $(\star)$.

To check (ii), we note that \[
\sigma\left(\tau \begin{pmatrix} a & \frac{b}{2} \\ \frac{b}{2} & c \end{pmatrix} \tau^\top\right)\sigma^\top = (\sigma \tau)\begin{pmatrix} a & \frac{b}{2} \\ \frac{b}{2} & c \end{pmatrix} (\sigma \tau)^\top.
\]
\begin{defn}
    The \textbf{discriminant} of $f(x,y) = ax^2+bxy+cy^2$ is $$\text{disc}(f) = b^2-4ac.$$
\end{defn}
\begin{example}
    $\text{disc}(1,0,1) = -4, \text{disc}(4,12,10) = -16$.
\end{example}
\begin{lemma}
    Equivalent BQFs have the same discriminant.
\end{lemma}
\begin{proof}
    Taking determinants in $(\star)$ gives
    \[
        a'c' - \left(\frac{b'}{2}\right)^2 = \left(\text{det }\sigma\right)^2\left(ac - \left(\frac{b}{2}\right)^2\right).
    \]
    But $\det \sigma = 1$, so multiplying both sides by $-4$ gives $(b')^2-4a'c' = b^2-4ac$ as desired.
\end{proof}
\textbf{Remark.} The converse is not true, i.e. there exist BQFs with the same discriminant which are not equivalent.

For example, $(1,0,6)$ and $(2,0,3)$ both have discriminant $-24$, but $(1,0,6)$ represents 1 (with $x=1,y=0$), but $(2,0,3)$ does not.
\begin{lemma}
    There exists a BQF $f$ with $\text{disc}(f)=d \iff d \equiv 0,1 \pmod{4}$. 
\end{lemma}
\begin{proof}
    $\implies$: $d = b^2-4ac \equiv b^2 \equiv 0,1 \pmod{4}$.

    $\impliedby$: If $d \equiv 0 \pmod{4}$, let $f = (1,0,-\frac{d}{4})$. If $d \equiv 1 \pmod{4}$, take $f = (1,1,\frac{1-d}{4})$.
\end{proof}

\marginpar{27 Oct 2022, Lecture 10}

\begin{defn}
    A quadratic form $f(x_1,\ldots,x_n)=\sum_{i\le j}^{} a_{ij}x_ix_j$ with $a_{ij} \in \mathbb{R}$ is:
    \begin{itemize}
        \item \textbf{positive definite} if $f(x)>0 ~\forall 0 \neq x \in \mathbb{R}^n$.
        \item \textbf{negative definite} if $f(x)<0 ~\forall 0 \neq x \in \mathbb{R}^n$.
        \item \textbf{indefinite} if $f(x)>0$ and $f(x')<0$ for some $x,x' \in \mathbb{R}^n$. 
    \end{itemize}
\end{defn}
We are interested in the case $n=2$ and $a_{ij} \in \mathbb{Z}$.

\begin{lemma}
    Let $f(x,y)=ax^2+bxy+cy^2$ be a BQF which has discriminant $d=b^2-4ac$. 
    \begin{enumerate}[(i)]
        \item If $d<0$ and $a>0$, then $f$ is positive definite.
        \item If $d<0$ and $a<0$, then $f$ is negative definite.
        \item If $d>0$, then $f$ is indefinite.
        \item If $d=0$, then $f= \lambda(mx+ny)^2$ for $\lambda,m,n \in \mathbb{Z}$.
    \end{enumerate}
\end{lemma}
\begin{proof}
    \begin{align*}
        4af(x,y)=4a^2x^2+4abxy+&4acy^2 = \\ 
        (2ax+by)^2 + &(4ac-b^2)y^2 = (2ax+by)^2 - dy^2.
    \end{align*}

    (i) and (ii): If $d<0$ and $a\neq 0$, then it follows that $4af(x,y)\ge 0$ with equality if and only if $x=y=0$. The cases $a>0$ and $a<0$ now show $f$ is either positive or negative definite as desired.
    \vspace{1mm}
    
    (iii): Suppose $d>0$. If $a\neq 0$, then the above equation shows us that $4af(1,0)>0$ and $4af(-b,2a)<0$, so $f$ is indefinite. 
    
    If $a=0$, then replace $f(x,y) \mapsto f(y,x)$. This works unless $a=c=0$, but then $b\neq 0$, so $f(x,y)=bxy$, which is obviously indefinite.
    \vspace{1mm}
    
    (iv): Omitted (not interesting nor difficult).
\end{proof}

\textbf{Remark.} It is possible for a BQF $(a,b,c)$ with $a,b,c>0$ to be indefinite, e.g. $(1,3,1)$.

It is also possible for $(a,b,c)$ with $b<0$ to be positive definite, e.g. $(1,-1,2)$.
\vspace{1mm}

From now on, we will concentrate on positive definite BQFs, i.e. forms $(a,b,c)$ with $d=b^2-4ac<0$ and $a>0$ (and hence $c>0$).

We have an equivalence relation $\sim$ on positive definite BQFs, and we want to study the equivalence classes. It will help if we can specify a "simplest" form for each equivalence class.

\begin{example}
    Consider $(10,34,29)$. The middle coefficient is large -- can we decrease it? If $f(x) = ax^2+bxy+cy^2$, then one substitution we may try is 
    \begin{align*}
        f(x+\lambda y,y)=a(x+\lambda y)^2+ b(x + \lambda y)y + &cy^2 = \\ 
        ax^2 &+ (b+2\lambda a)xy + (\lambda^2 a + \lambda b + c)y^2.
    \end{align*}
    Taking $\lambda = \pm 1$ shows $$(a,b,c) \sim (a,b \pm 2a, a \pm b + c). ~~~~(\dagger)$$
    In our example, we get $(10,34,29) \sim (10,14,5) \sim (10, -6, 1)$.

    Making the substitution $X=y, Y=-x$ gives $$(a,b,c) \sim (c,-b, a). ~~~~ (\ddagger)$$ In our example we now get $$(10,-6,1) \sim (1, 6, 10) \sim (1,4, 5) \sim (1,2,2) \sim (1,0,1).$$
\end{example}
\textbf{Remark.} It is a good idea to check that the discriminant doesn't change (to catch mistakes).

\textbf{Remark.} We can ensure $|b|\le a$ via $(\dagger)$, and $a\le c$ via $(\ddagger)$.
\begin{defn}
    A positive definite BQF is \textbf{reduced} if either \[
    -a < b \le a < c ~\text{, or } ~ 0\le b \le a = c.
    \]
\end{defn}
(Think of this as $|b|\le a \le c$ with some extra conditions).
\begin{lemma}\label{4.5}
    Every positive definite BQF is equivalent to a reduced form.
\end{lemma}
\begin{proof}
    We have operations $$S : (a,b,c) \mapsto (c, -b, a), ~T_{\pm}:(a,b,c) \mapsto (a, b \pm 2a, a \pm b +c).$$
    If $a>c$, then use $S$ to decrease $a$ while leaving $|b|$ unchanged. If $a\le c$ and $|b|>a$, then use $T_{\pm}$ to decrease $|b|$ while leaving $a$ unchanged. 
    
    Repeat these steps. Each step decreases $a + |b|$, so this procedure must eventually reach a form with $|b|\le a\le c$. Finally, to get the form we want in the lemma:

    \begin{itemize}
        \item If $b=-a$, then apply $T_+$ to replace $(a,-a,c) \mapsto (a, a, c)$.
        \item If $a=c$ and $b<0$, then apply $S$ to get $b>0$.
    \end{itemize}
\end{proof}

\marginpar{29 Oct 2022, Lecture 11}

\begin{lemma}\label{4.6}
    Let $f=(a,b,c)$ be a reduced positive definite BQF with discriminant $d$. Then $|b|\le a \le \sqrt{\frac{|d|}{3}}$ and $b \equiv d \pmod{2}$.
\end{lemma}
\begin{proof}
    Being reduced implies $|b|\le a\le c$, and $d = b^2-4ac \le ac -4ac = -3ac \le -3a^2 \implies a^2 \le \frac{|d|}{3}$. Also $d = b^2-4ac \implies b \equiv d \pmod{2}$.
\end{proof}
\begin{example}
    Consider $d = -4$. We must have $a=1$ by the lemma above (as $a>0$), and $b=0$ (by parity), so solve for $c$ to get $c=1$, i.e. $x^2+y^2$ is the only positive definite reduced BQF with discriminant $-4$. 
\end{example}

We can now return to the beginning of this section and answer our original question: which numbers can be written as the sum of two squares?
\begin{proof}[Proof of Theorem 4.1 (continued)]
    Let $p$ be a prime, $p \equiv 1\pmod{4}$. We have $\left(\frac{-1}{p} \right) = 1$, so $\exists u \in \mathbb{Z}$ such that $u^2 \equiv -1 \pmod{p} \implies u^2 = -1 + kp$ for some $k \in \mathbb{Z}$. Let $f = (p,2u,k)$, so $\text{disc}(f) = 4u^2 - 4pk = -4$.
    \vspace{1mm}
    
    By Lemma \ref{4.5}, $f \sim g$ for some reduced form $g$, but by our above example, $g(x,y)= x^2+y^2$. Now $f$ represents $p$ (take $x=1,y=0$), so $g$ also represents $p$, i.e. $p$ is the sum of two squares as required.
\end{proof}
\textbf{Question.} Can reduced forms be equivalent?

\begin{defn}
    Let $f$ be a BQF and $n \in \mathbb{Z}$. We say $f$ \textbf{represents} $n$ if $n=f(x,y)$ for some $x,y \in \mathbb{Z}$. We say $f$ \textbf{properly represents} $n$ if $n=f(x,y)$ for some coprime $x,y \in \mathbb{Z}$.
\end{defn}
\textbf{Remark.} Equivalent forms properly represent the same integers, since if $X=\alpha x  + \gamma y, Y = \beta x + \delta y$ with $\alpha,\beta,\gamma,\delta \in \mathbb{Z}$, then $\alpha \delta - \beta \gamma = 1$ implies $\gcd(X,Y)=1 \iff \gcd(x,y)=1.$
\begin{lemma}\label{4.7}
    The smallest integers properly represented by a reduced positive definite BQF $f=(a,b,c)$ are $a,c, a - |b| + c$ in that order.\footnote{Values on this list are repeated if they are represented in more than one way, not counting repeats of the form $f(x,y)=f(-x,-y)$.}
\end{lemma}
\begin{proof}
    $f$ reduced $\implies |b| \le a \le c \implies a \le c \le a - |b| + c$. We have $f(1,0)=a$, $f(0,1)=c$. If $x=0$, then $\gcd(x,y)=1 \implies y=\pm 1$. Likewise, if $y=0$, then $x = \pm 1$.

    So it remains to show that the smallest number represented by $f$ using nonzero $x,y$ is $a-|b|+c$. But if $|x|\ge |y|\ge 1$, then 
    \begin{align*}
        f(x,y)=ax^2+bxy+cy^2\ge ax^2 - |b||x||y| + cy^2 \ge (a-|b|)x^2 + cy^2 \ge a - |b| + c.
    \end{align*}
    We can achieve equality with $f(1,\pm 1)$. We proceed similarly if $|y|\ge |x|\ge 1$.
\end{proof}
\begin{theorem}\label{4.8}
    Every positive definite BQF is equivalent to a unique reduced form.
\end{theorem}
\begin{proof}
    Existence follows from Lemma \ref{4.5}.

    Uniqueness: Suppose $f=(a,b,c)$ and $g=(a',b',c')$ are equivalent reduced BQFs. We want to show $a=a',b=b',c=c'$. By Lemma \ref{4.7}, $a=a',c=c'$ and $a-|b|+c = a' -|b'|+ c'$, so $(a,b,c)=(a',\pm b',c')$.
    \vspace{1mm}
    
    If $b=0$, we're done. If $b\neq 0$, can $(a,b,c)$ and $(a,-b,c)$ both be reduced? If yes, then $a<c$ (since $a=c$ requires $b\ge 0$ by definition) and $|b| <a$ (since we can't have $b=-a$). Hence $a < c < a - |b| + c$. By Lemma \ref{4.7} again, $f(x,y)=a \iff (x,y) = (\pm 1,0)$ and $f(x,y) = c \iff (x,y) = (0,\pm 1)$, and likewise for $g$.
    \vspace{1mm}
    
    Suppose $g(x,y) = f(\alpha x + \gamma y, \beta x + \delta y) = f(X,Y)$. Then 
    \begin{align*}
        (X,Y) = (\pm 1, 0) \iff (x,y) = (\pm 1, 0) \\
        (X,Y) = (0, \pm 1) \iff (x,y) = (0, \pm 1),
    \end{align*}
    i.e. $\begin{pmatrix} \alpha & \beta \\ \gamma & \delta \end{pmatrix} = \begin{pmatrix} \pm 1 & 0 \\ 0 & \pm 1 \end{pmatrix}$. But $\alpha \delta - \beta \gamma = 1$, so $\begin{pmatrix} \alpha & \beta \\ \gamma & \delta \end{pmatrix} = \pm \begin{pmatrix} 1 & 0 \\ 0 & 1 \end{pmatrix}$, so $f=g$ as required.
\end{proof}

\marginpar{01 Nov 2022, Lecture 12}

\textbf{Question.} How many reduced forms are there with a given discriminant?

\begin{example}
    Consider $d = -24$. We want to find $f = (a,b,c)$ reduced with $b^2-4ac=-24$. By Lemma \ref{4.6}, $|b|\le a \le \sqrt{8}$ and $b$ is even.
    \begin{itemize}
        \item If $a=1$, then $b=0$ and hence $c=6 \implies (1,0,6)$. We can check that this is reduced.
        \item If $a=2$, then $c=\frac{b^2+24}{8}$.
        \begin{itemize}
            \item If $b=0$, then $c=3$. This is reduced.
            \item If $b=\pm 2$, then $c \not\in \mathbb{Z}$.
        \end{itemize}
    \end{itemize}
    So the only reduced forms with discriminant $-24$ are $(1,0,6)$ and $(2,0,3)$.
\end{example}
More generally, Lemma \ref{4.6} shows that for every $d$, there are only finitely many reduced forms with discriminant $d$.
\begin{defn}
    The \textbf{class number} of $d$, denoted $h(d)$ is the number of equivalence classes of positive definite BQFs with discriminant $d$.
\end{defn}
By Theorem \ref{4.8}, this is the number of reduced forms with discriminant $d$, hence finite by the last remark.
\begin{example}
    As we have already seen, $h(-4)=1, h(-24)=2$.
\end{example}
\begin{defn}
    $d \equiv 0,1 \pmod{4}$ is a \textbf{fundamental discriminant} if it is not of the form $d=k^2d_1$ for some integer $k\ge 1$ and $d_1 \equiv 0,1 \pmod{4}$.
\end{defn}

Aside:

\textbf{Remark.} Let $d<0$ be a fundamental discriminant. Gauss defined a group law on the set of equivalence classes of positive definite BQFs with discriminant $d$. The abelian group obtained in this way is the same as the class group of the field $\mathbb{Q}(\sqrt{d})$ (see Part II Number Fields). We insisted that $\alpha \delta - \beta \gamma = 1$ in the definition of equivalence (not just $=\pm 1$), since otherwise inverse elements in the class group would be the same element, hence it is no longer a group. End of aside.

\textbf{Some theorems about class numbers.}

\begin{enumerate}[(i)]
    \item (Mertens 1874). \[
    \sum_{-X<d<0}^{} h(d) \sim \frac{\pi}{18} X^{\frac{3}{2}} \text{ as } X \to \infty.
    \]
    \item (Heilbronn 1934) $h(d) \to \infty$ as $|d| \to \infty$.
    \item (Siegel 1935) For every $\epsilon>0, \exists c>0$ such that $h(d)>c |d|^{\frac{1}{2}-\epsilon}$.
    \item (Baker-Stark 1967) $h(d)=1 \iff d \in \{-3,-4,-7,-8,-11,-19,-43,-67,-163\}$.
\end{enumerate}
End of aside.

\begin{lemma}\label{4.9}
    Let $f$ be a BQF and $n \in \mathbb{Z}$. Then $f$ properly represents $n$ if and only if $f$ is equivalent to a form with first coefficient $n$.
\end{lemma}
\begin{proof}
    $\impliedby$: Suppose $f \sim g(n,b,c)$. Then $g(1,0)=n \implies g$ properly represents $n$, so $f$ properly represents $n$.
    \vspace{1mm}
    
    $\implies$: $f(\alpha,\beta)=n$ for some $\alpha,\beta \in \mathbb{Z}$ coprime. By Euclid's algorithm, $\exists \gamma,\delta \in \mathbb{Z}$ such that $\alpha \delta - \beta \gamma = 1$. Then $f$ is equivalent to $g(x,y) = f(\alpha x + \gamma y, \beta x + \delta y)$ with first coefficient $g(1,0) = f(\alpha,\beta)=n$.
\end{proof}
\begin{theorem}
    Let $n$ be a positive integer and $d<0$ a discriminant. Then $n$ is properly represented by some positive definite BQF with discriminant $d$ if and only if the congurence $$x^2 \equiv d \pmod{4n}$$ is soluble.
\end{theorem}
\begin{proof}
    $\implies$: Lemma \ref{4.9} shows $f \sim g$ with $g = (n,b,c)$. Then $$d = \text{disc}(f) = \text{disc}(g) = b^2 - 4nc \equiv b^2 \pmod{4n}.$$
    \vspace{1mm}
    
    $\impliedby$: We are given $b,c \in \mathbb{Z}$ such that $b^2 = d + 4nc$. Then $f = (n,b,c)$ is a form of discriminant $d$ and it properly represents $n$ (with $x=1,y=0$).
\end{proof}
\begin{example}
    Which integers are properly represented by $f(x,y)={x^2+xy+2y^2}$?
    
    We have $\text{disc}(f)=-7$, so $f$ is positive definite. By Lemma \ref{4.6}, any reduced form with discriminant $-7$ satisfies $|b|\le a \le 1$ and $b$ is odd. Hence $(a,b,c)=(1,1,2)$ or $(a,b,c) = (-1,-1,2)$. But the second one is not reduced, hence $h(-7)=1$ and all positive definite BQFs with discriminant $-7$ are equivalent.
    \vspace{1mm}
    
    Hence $n$ is properly represented by $x^2+xy+2y^2$ if and only if $x^2 \equiv -7 \pmod{4n}$ is soluble.

    \marginpar{03 Nov 2022, Lecture 13}

    Assume $n=p$ is prime and $p \neq 2,7$. By CRT, the above is equal to 
    \[
    \begin{cases}
        x^2 \equiv -7 \pmod{4}. \text{ This is soluble.} \\
        x^2 \equiv  -7 \pmod{p}. \text{This is soluble }\iff \left(\frac{-7}{p} \right) = 1. 
    \end{cases}
    \]
    But $\left(\frac{-7}{p} \right)  = \left(\frac{-1}{p}\right)\left(\frac{7}{p}\right) = (-1)^{\frac{p-1}{2}}(-1)^{\frac{p-1}{2}}\left(\frac{p}{7}\right)=\left(\frac{p}{7}\right)$.

    We conclude that $p = x^2+xy+2y^2$ for some $x,y \in \mathbb{Z}$ means that $p \equiv 1,2,4 \pmod{7}$ or $p=2,7$ (we check $p=2,7$ separately).
\end{example}
\begin{lemma}
    Let $p$ be an odd prime and $a \in \mathbb{Z}$. If $\left(\frac{a}{p} \right) =1$, then the congruence $x^2 \equiv a \pmod{p^n}$ is soluble $~\forall n\ge 1$.
\end{lemma}
\begin{proof}
    Induction on $n$. The case $n=1$ is clear.
    \vspace{1mm}
    
    Now let $n\ge 1$ and suppose $x^2 \equiv a \pmod{p^n}$, i.e. $x^2 = a +k p^n, k \in \mathbb{Z}$. For $t \in \mathbb{Z}$, we have $(x + tp^n)^2 \equiv x^2 + 2xt p^n \equiv a + (2xt+k)p^n \pmod{p^{n+1}}$. Now we have $(2x,p)=1$, so we can solve $2xt+k \equiv 0\pmod{p}$, so we're done.
\end{proof}
\textbf{Remark.} A similar argument shows that $a \in \mathbb{Z}$ with $a \equiv 1\pmod{8}$, then $x^2 \equiv a \pmod{2^n}$ is soluble $~\forall n\ge 1$.

\textbf{Above example continued:} Write $n = 2^{\alpha}7^{\beta}p_1^{\gamma_1}\ldots p_{r}^{\gamma_r}$ for $p_i$ distinct powers. Then
\[
x^2 \equiv -7 \pmod{4n} \text{ is soluble} \iff \begin{cases}
    &x^2 \equiv -7 \pmod{2^{\alpha+2}} \text{ is soluble.} \\
    &x^2 \equiv -7 \pmod{7^{\beta}} \text{ is soluble.} \\
    &x^2 \equiv -7 \pmod{p_i^{\gamma_i}} \text{ is soluble}~\forall 1\le i\le r. \\
\end{cases}
\]
The first condition is always true by the remark above. The second one has no solutions mod $49$, so hence $\beta\le 1$. For the last condition, use the above lemma to get that we need $\left(\frac{-7}{p_i} \right) = 1 ~\forall 1 \le i \le r$.
\vspace{1mm}

Hence we want $7^2 \nmid n$ and all primes $p \mid n$ with $p \neq 7$ satisfy $p \equiv 1,2,4 \pmod{7}$.
\vspace{1mm}

The integers represented by $x^2+xy+y^2$ (not necessarily properly) are then of the form $k^2n$ for $k \in \mathbb{Z}$ and $n$ as described above.
\vspace{1mm}

\textbf{Conclusion.} $n = x^2+xy+2y^2$ for some $x,y \in \mathbb{Z} \iff$ every prime $p \equiv 3,5,6$ which divides $n$ divides it to an even power.
\vspace{1mm}

\textbf{Remarks.}
\begin{enumerate}[(i)]
    \item If $h(d)=1$, we have shown how to solve the problem of which integers are represented by a given form of discriminant $d<0$.
    
    If $h(d)>1$, we can determine which integers are represented by \textit{some} form of discriminant $d$. For some values of $d$ we can still distinguish which forms represent which numbers using congruence conditions.
    \item What about quadratic forms in more variables?
    \begin{theorem}[Lagrange 1770]
        Every positive integer is a sum of four squares.
    \end{theorem}
    \begin{theorem}[Legendre 1797]
        A positive integer $n$ is a sum of 3 squares if and only if $n \neq 4^a(8b+7)$ for some integers $a,b \ge 0$.
    \end{theorem}
    \item A geometric way to think about reduction: Let $f(x,y) = ax^2+bxy+cy^2$ be a positive definite BQF, so $d=b^2-4ac<0$. Let $\tau \in \mathbb{C}$ with $f(\tau,1)=0$ and $\text{Im}(\tau)>0$, so $\tau = \frac{-b \pm \sqrt{|d|} i}{2a}$, and $|\tau|^2 = \frac{b^2-d}{4a^2} = \frac{c}{a}$.
    
    So $|b|\le a\le c \iff |\text{Re}(\tau)|\le \frac{1}{2}$ and $|\tau| \ge 1$. Let $\mathcal{F}$ be this subregion of $\mathbb{C}$. Then $SL_2(\mathbb{Z}) \subset  SL_2(\mathbb{R})$ acts on $\mathcal{H} = \{\tau \in \mathbb{C} \mid \text{Im}(\tau)>0\}$ via $\begin{pmatrix} a & b\\c &d \end{pmatrix} : \tau \to \frac{a\tau + b}{c\tau + d}$, and the operations $S$ and $T_{\pm}$ in the proof of Lemma \ref{4.5} correspond to the Möbius maps $S: \tau \mapsto \frac{-1}{\tau}$ and $T_{\pm } : \tau \mapsto \tau \pm 1$. So we just start somewhere in the complex plane and apply these transformations until we end up in $\mathcal{F}$.

    \item Extra conditions in the definition of a reduced form correspond to conditions concerning the boundary of $\mathcal{F}$.
\end{enumerate}

\section{The distribution of primes}

\marginpar{05 Nov 2022, Lecture 14}


Define $\pi(x)$ to the number of primes $\le x$. In lecture 2, we saw that $\pi(x) \to \infty$ as $x \to \infty$ (by Euclid). On Example Sheet 1, we saw $\pi(x) \ge \frac{\log x}{\log \log x}$ if $x\ge 8$.
\begin{lemma}
    $\exists c>0$ such that $\pi(x)> c \log x$.
\end{lemma}
\begin{proof}
    For $n \le x$ we can write $n = k^2 p_1^{\alpha_1} p_r^{\alpha_r}$ with $k\le \sqrt{x}$, $p_i$ all the primes $\le x$ and $\alpha_i \in \{0,1\}$ (so $p_1^{\alpha_1} p_r^{\alpha_r}$ is squarefree).
    \vspace{1mm}
    
    There are $\le \sqrt{x}$ choices for $k$ and $\le 2^r$ choices for $\alpha_1,\ldots,\alpha_r$, so $$x \le \sqrt{x}2^{\pi(x)} \implies \pi(x) \ge \frac{\log x}{2\log 2}.$$
\end{proof}
The following result gives another proof of the infinitude of primes.
\begin{theorem}\label{5.2}
    $\sum_{p}^{} \frac{1}{p}$ diverges and $\prod_{p}^{} (1-\frac{1}{p})^{-1}$ diverges.
\end{theorem}
\begin{proof}
    For $x\ge 2$, we define $P(x)=\prod_{p\le x}^{} (1-\frac{1}{p})^{-1}$ and $S(x)=\sum_{p\le x}^{} \frac{1}{p}$. We show that $P(x) \to \infty$ and $S(x) \to \infty$ as $n \to \infty$.
    \begin{enumerate}[(i)]
        \item Let $p_1,\ldots,p_r$ be the primes $\le x$. Then 
        \begin{align*}
            P(x) = \prod_{i=1}^{r} (1+\frac{1}{p_i}+\frac{1}{p_i^2}+ \ldots) = \sum_{\alpha_1=0}^{\infty} \ldots \sum_{\alpha_r=0}^{\infty} \frac{1}{p_1^{\alpha_1} \ldots p_r^{\alpha_r}} \ge \sum_{n=1}^{\left\lfloor x \right\rfloor} \frac{1}{n} \stackrel{n \to \infty}{\to} \infty.
        \end{align*}
        \item 
        \begin{align*}
            \log P(x) = -\sum_{i=1}^{r} \log\left(1-\frac{1}{p_i}\right) \stackrel{(\star)}{=}  \sum_{i=1}^{r} \sum_{m=1}^{\infty} \frac{1}{mp_i^m} = S(x) + \sum_{i=1}^{r} \sum_{m=2}^{\infty} \frac{1}{m p_i^m}
        \end{align*}
        where $(\star)$ follows from the Taylor series expansion of $\log(1+x)$. But $\sum_{m=2}^{\infty} \frac{1}{p^m} = \frac{p^{-2}}{1-p^{-1}} = \frac{1}{p(p-1)}$, so \[
        0 < \log P(x) - S(x) < \frac{1}{2}\sum_{i=1}^{r} \frac{1}{p_i(p_i-1)} \le \frac{1}{2}\sum_{n=2}^{\infty} \frac{1}{n(n-1)} = \frac{1}{2}.
        \]
        Thus $S(x) \to \infty$ as $n \to \infty$.
    \end{enumerate}
\end{proof}
\textbf{Remark.} $\sum_{n=1}^{\left\lfloor x \right\rfloor} \frac{1}{n} > \int_{1}^{\left\lfloor x \right\rfloor+1} \frac{\mathrm{d}u}{u} = \log(\left\lfloor x \right\rfloor+1) \ge \log x$. So the proof of (i) shows $P(x) > \log(x)$ and the proof of (ii) shows $S(x) > \log \log x - \frac{1}{2}$. This is a rather good approximation:
\begin{theorem}[Mertens 1874]\label{5.3}
    There exists a constant $B$ such that $S(x) = \log \log x + B + O(\frac{1}{\log x})$.
\end{theorem}
\begin{proof}
    Omitted, but a key ingredient is the following theorem which we will later prove.
\end{proof}
\begin{theorem}[Tchebychev 1852]
    There exist constants $a,b >0$ such that $\frac{ax}{\log x} < \pi(x) < \frac{bx}{\log x}$.
\end{theorem}
\begin{lemma}
    If $\frac{\pi(x) \log x}{x}$ tends to a limit as $x \to \infty$, then that limit must be 1.
\end{lemma}
\begin{proof}
    \begin{align*}
        S(x) = \sum_{p\le x}^{} \frac{1}{p} = \sum_{n\le x}^{} \frac{\pi(n)-\pi(n-1)}{n} = \sum_{n=2}^{\left\lfloor x \right\rfloor-1} \pi(x)\left(\frac{1}{n}-\frac{1}{n-1}\right) + \frac{\pi(x)}{\left\lfloor x \right\rfloor} = \\
        \sum_{n=2}^{\left\lfloor x \right\rfloor-1} \int_{n}^{n+1} \frac{\pi(u)}{u^2}\mathrm{d}u + \int_{\left\lfloor x \right\rfloor}^{x} \frac{\pi(u)}{u^2}\mathrm{d}u + \frac{\pi(x)}{x} = \frac{\pi(x)}{x} + \sum_{2}^{x} \frac{\pi(u)}{u^2}\mathrm{d}u.
    \end{align*}
    If $\frac{\pi(x)\log x}{x} \to \alpha$ as $x \to \infty$, then we get 
    \begin{align*}
        S(x) \sim \alpha \int_{2}^{x} \frac{\mathrm{d}u}{u \log u} = \alpha \left[\log \log u \right]_2^x \implies S(x) \sim \alpha \log \log x. 
    \end{align*}
    By Theorem \ref{5.2}, $\alpha\ge 1$, but by Mertens (Theorem \ref{5.3}), $\alpha=1$.
\end{proof}
\begin{theorem}[Prime Number Theorem]
    \[
    \pi(x) \sim \frac{x}{\log x}.
    \]
\end{theorem}
\textbf{Remarks.}
\begin{itemize}
    \item Equivalently, this says $\frac{\pi(x) \log x}{x} \to 1$ as $x \to \infty$.
    \item This was proved independently by Hadamard and de la Vallee Poussin.
    \item The proof uses the Riemann zeta function and complex analysis.
\end{itemize}
\begin{defn}[Riemann zeta function]
    For $s \in \mathbb{C}$ with $\text{Re}(s)>1$, we say $$\zeta(s) = \sum_{n=1}^{\infty} \frac{1}{n^s}.$$
\end{defn}
\textbf{Remark.} In this context, the convention is to write $s = \sigma + it$.
\begin{lemma}
    For $\text{Re}(s)>1$, the series $\sum_{n=1}^{\infty} \frac{1}{n^s}$ converges absolutely. Moreover, for any $\delta>0$, it converges uniformly on $\text{Re}(s)\ge 1+\delta$ (and hence is analytic on $\text{Re}(s)>1$).    
\end{lemma}
\begin{proof}
    For $s = \sigma + it$, we have 
    \begin{align*}
        |n^s| = |n^{\sigma + it}| = |e^{(\sigma+it)\log n}| = e^{\sigma \log n} = n^\sigma.
    \end{align*}
    But $\sum_{n=1}^{\infty} \frac{1}{n^\sigma}$ converges for $\sigma>1$, and it converges uniformly for $\sigma \ge 1 + \delta$ (by IA Analysis).
\end{proof}
The following result links $\zeta$ to the primes.
\begin{prop}[Euler product for $\zeta$]
    For $\text{Re}(s)>1$, we have \[
    \zeta(s) = \prod_{p}^{} \left(1-\frac{1}{p^s}\right)^{-1}. 
    \]
\end{prop}
\begin{proof}
    The rough idea:
    \[
    \prod_{p}^{} \left(1-\frac{1}{p^s}\right)^{-1} = \prod_{p}^{} \left(1+\frac{1}{p^s}+\frac{1}{p^{2s}}+\ldots \right) \stackrel{(\star)}{=} \sum_{n=1}^{\infty} \frac{1}{n^s}
    \] where $(\star)$ follows from the Fundamental Theorem of Arithmetic.
    \vspace{1mm}
    
    In detail: Fix $s$ with $\text{Re}(s)>1$. If $M>\frac{\log N}{\log 2}$, then $p^M > N ~\forall $primes $p$. Now:
    \begin{align*}
        \prod_{p\le N}^{} \sum_{j=0}^{M} \frac{1}{p^{js}} = 1+ \frac{1}{2^s} + \frac{1}{3^s} + \ldots + \frac{1}{N^s} + \left(\text{extra terms }\frac{1}{n^s} \text{ for }n>N \right).
    \end{align*}
    Hence
    \begin{align*}
        \left|\sum_{n=1}^{\infty} \frac{1}{n^s} - \prod_{p\le N}^{} \sum_{j=0}^{m} \frac{1}{p^{js}} \right| \le \sum_{n=N+1}^{\infty} \frac{1}{n^\sigma}.
    \end{align*}
    Take the limit as $M \to \infty$ to get 
    \begin{align*}
        \left|\zeta(s)- \prod_{p\le N}^{} \left(1-\frac{1}{p^s}\right)^{-1}\right| = \sum_{n=N+1}^{\infty} \frac{1}{n^{\sigma}} \stackrel{N \to \infty}{\to} 0.
    \end{align*}
\end{proof}

\marginpar{08 Nov 2022, Lecture 15}

\begin{cor}\label{5.9}
    If $\text{Re}(s)>1$, then $\zeta(s)\neq 0$.
\end{cor}
\begin{proof}
    If $\text{Re}(s) >1$, then 
    \begin{align*}
        \left[\prod_{p\le N}^{} \left(1-\frac{1}{p^s}\right)\right] \zeta(s) = \prod_{p>N}^{} \left(1-\frac{1}{p^s}\right)^{-1} = \prod_{p>N}^{} \left(1+\frac{1}{p^s}+\frac{1}{p^{2s}}+\ldots \right) \\
        \implies \left|\prod_{p\le N}^{} \left(1-\frac{1}{p^s}\right)\zeta(s)\right| \ge 1-\sum_{n=N+1}^{\infty} \frac{1}{n^\sigma} \stackrel{N \to \infty}{\to} 1. 
    \end{align*}
    Hence $\zeta(s) \neq 0$.
\end{proof}
\begin{theorem}\label{5.10}
    $\zeta(s)-\frac{1}{s-1}$ has an analytic continuation to $\text{Re}(s) > 0$.
\end{theorem}
\begin{proof}
    If $\text{Re}(s)>2$ we have 
    \begin{align*}
        &\zeta(s)=\sum_{n=1}^{\infty} \frac{n-(n-1)}{n^s} =\sum_{n=1}^{\infty} n\left(\frac{1}{n^s}-\frac{1}{(n+1)^s}\right) = s \sum_{n=1}^{\infty} n \int_{n}^{n+1} \frac{\mathrm{d}x}{x^{s+1}} = \\
        &=s \int_{1}^{\infty} \frac{\left\lfloor x \right\rfloor}{x^{s+1}} \mathrm{d}x = s \int_{1}^{\infty} \frac{\mathrm{d}x}{x^s} - s \int_{1}^{\infty} \frac{\{x\}}{x^{s+1}} \mathrm{d}x = \frac{s}{s-1} - s \int_{1}^{\infty} \frac{\{x\}}{x^{s+1}}\mathrm{d}x.
    \end{align*}
    Since $\{x\}$ is bounded, the second integral converges to an analytic function for $\text{Re}(s+1)>1$, i.e. $\text{Re}(s)>0$.
\end{proof}
For $\text{Re}(s)>0$, the \textbf{Gamma function} is defined as
\[
\Gamma(s)=\int_{0}^{\infty} e^{-x} x^{s-1} \mathrm{d}x.
\] 
This can be extended to a meromorphic\footnote{Analytic except on a set of isolated points.} function on $\mathbb{C}$ with simple poles at $s=0,-1,-2,\ldots$ using the rule $s \Gamma(s) = \Gamma(s+1)$. For an integer $n\ge 1$, $\Gamma(n)=(n-1)!$.
\vspace{1mm}

Theorem \ref{5.10} tells us that $\zeta$ extends to a meromorphic function on the set $\{s \in \mathbb{C} \mid \text{Re}(s)>0\}$ with just one pole at $s=1$ with residue 1. In fact, $\zeta$ extends to a meromorphic function on $\mathbb{C}$ and there are no further poles.

Moreover, the completed zeta function \[
\Xi(s)=\pi^{-\frac{s}{2}} \Gamma \left(\frac{s}{2}\right)\zeta(s)
\] satisfies the functional equation $\Xi(1-s)=\Xi(s)$.
\vspace{1mm}

$\zeta$ has trivial zeroes at $s=-2,-4,-6,\ldots$. By Corollary \ref{5.9} and the functional equation, any further zeroes lie in the critical strip $0\le \text{Re}(s)\le 1$.

The key step in the proof of the Prime Number Theorem is showing that $\zeta(s) \neq 0$ for $\text{Re}(s)=1$.

\begin{theorem}[The Riemann Hypothesis]
    All zeroes of $\zeta$ in the critical strip lie on the line $\text{Re}(s)=\frac{1}{2}$.
\end{theorem}
\begin{proof}
    lol
\end{proof}
RH is equivalent to $$|\pi(x)-\text{li}(x)|\le \sqrt{x}\log x ~\forall x\ge 3,$$ where $\text{li}(t)=\int_{2}^{x} \frac{\mathrm{d}t}{\log t}$. Integrating by parts shows $\text{li}(x) \sim\frac{x}{\log x}$. Numerical evidence suggested to Gauss that $\text{li}(x)$ is a better approximation to $\pi(x)$ than $\frac{x}{\log x}$. We have $\pi(x)<\text{li}(x) ~\forall x \le 10^{21}$, but Littlewood showed that $\pi(x)-\text{li}(x)$ changes sign infinitely often.
\vspace{1mm}

A \textbf{Dirichlet series} is a series of the form $$\sum_{n=1}^{\infty} \frac{a_n}{n^s}$$ where $(a_i) \in \mathbb{C}$. 

A useful tool for manipulating the all the aforementioned series is the Möbius function. Let $f: \mathbb{N} \to \mathbb{C}$ be any function. Define $g : \mathbb{N} \to \mathbb{C}$ by $$g(n)=\sum_{d \mid n}^{} f(d).$$

\textbf{Question.} How do we compute $f$ from $g$?

Let's compute $f(6)$. We have 
\begin{align*}
    &g(1)=f(1)\\
    &g(2)=f(1)+f(2) \hspace{25mm} \implies f(6)=g(6)-g(3)-g(2)+g(1).\\
    &g(3)=f(1)+f(3)\\
    &g(6)=f(1)+f(2)+f(3)+f(6)
\end{align*}
\begin{defn}
    The \textbf{Möbius function} $\mu : \mathbb{N} \to \{0,\pm 1\}$ is defined by 
    \[
    \mu(n) = \begin{cases}
        (-1)^k &\text{ if } n=p_1p_2\ldots p_k \text{ is a product of distinct primes.}\\
        0 &\text{ if }n \text{ is not square-free.}
    \end{cases}
    \]
\end{defn}
\textbf{Remark.} We have $\mu(1)=1$.
\vspace{1mm}

\textbf{Exercise.} $\mu$ is a multiplicative function. (This is on ES3).
\vspace{1mm}

Let $\nu(n) = \sum_{d \mid n}^{} \mu(d)$. By Lemma \ref{2.8}, $\nu$ is multiplicative. But $\nu(p^r) = \mu(1)+\mu(p)= 1 - 1 = 0$, so $\sum_{d \mid n}^{} \mu(d) = \begin{cases}
    1 &\text{ if } n=1. \\
    0 &\text{otherwise.}
\end{cases}$
\begin{prop}
    If $g(n)=\sum_{d \mid n}^{} f(d)$, then $f(n)=\sum_{m \mid n}^{} \mu(m)g \left(\frac{n}{m}\right)$.
\end{prop}
\begin{proof}
    \begin{align*}
        \sum_{m \mid n}^{} \mu(m)g\left(\frac{n}{m}\right) &= \sum_{m \mid n}^{} \mu(m) \sum_{d \mid \frac{n}{m}}^{} f(d) = \\ 
        =\sum_{d \mid n}^{} \left( \sum_{m \mid \frac{n}{d}}^{} \mu(m)\right)f(d) &= \sum_{d \mid n}^{} \nu\left(\frac{n}{d} \right)f(d) = f(n).
    \end{align*}
\end{proof}

\marginpar{10 Nov 2022, Lecture 16}

\textbf{Notation.} Let $n \in \mathbb{N}$ and $p$ a prime. Then $\nu_p(n)$ denotes the exponent of $p$ in the prime factorization of $n$.

\textbf{Remarks.}
\begin{itemize}
    \item We can write $n=p^{\nu_p(n)}b$ for $p \nmid b$.
    \item $\nu_p(mn)=\nu_p(m)+\nu_p(n) ~\forall m,n \in \mathbb{N}$.
    \item $\nu_p(n!)=\sum_{j=1}^{\infty} \left\lfloor \frac{n}{p^j} \right\rfloor$ (this is also on ES3).
\end{itemize} 

\begin{prop}\label{5.13}
    Let $n \in \mathbb{N}$. Let $N={{2n}\choose{n}}$.
    \begin{enumerate}[(i)]
        \item We have $$\frac{2^{2n}}{2n}\le N\le 2^{2n}.$$
        \item If $p^k \mid N$, then $p^k\le 2n$.
        \item We have $$n^{\pi(2n)-\pi(n)}\le N \le (2n)^{\pi(2n)}.$$
    \end{enumerate}
\end{prop}
\begin{proof}
    \begin{enumerate}[(i)]
        \item $$(1+1)^{2n}=\sum_{j=0}^{2n} {{2n}\choose{j}} = 2 + \sum_{j=1}^{2n-1} {{2n}\choose{j}}.$$
        Hence $N\le 2^{2n} \le 2 +(2n-1)N \le 2nN$.
        \item We have $N=\frac{(2n)!}{(n!)^2}$, so $$\nu_p(N) = \nu_p((2n)!) - 2\nu_p(n!) = \sum_{j=1}^{\infty} \left(\left\lfloor \frac{2n}{p^j} \right\rfloor- 2\left\lfloor \frac{n}{p^j} \right\rfloor\right).$$
        But for $x \in \mathbb{R}$ we have $\left\lfloor 2x \right\rfloor- 2 \left\lfloor x \right\rfloor = \begin{cases}
            0 &\text{ if } \{x\}<\frac{1}{2}. \\
            1 &\text{ if } \{x\}\ge \frac{1}{2}. \\
        \end{cases}$
        If $p^k>2n$, then $\left\lfloor \frac{2n}{p^k} \right\rfloor=0$, so \[
        \nu_p(N) = \sum_{j=1}^{k-1} \left(\left\lfloor \frac{2n}{p^j} \right\rfloor- 2\left\lfloor \frac{n}{p^j} \right\rfloor\right) \le k-1.
        \]
        Thus if $\nu_p(N)\ge k$, then $p^k \le 2n$.
        \item \[
        N = \frac{(2n)(2n-1)\ldots(n+1)}{n(n-1)\ldots 1} \ge \prod_{n<p<2n}^{} p \ge n^{\pi(2n)-\pi(n)}. 
        \]
        But also \[
        N=\prod_{p\le 2n}^{} p^{\nu_p(n)} \le (2n)^{\pi(2n)}
        \] by part (ii).
    \end{enumerate}
\end{proof}
\begin{theorem}[Tchebychev]
    $\exists c_2 > c_1 > 0$ such that $~\forall x\ge 4$,
    \[
    c_1 \frac{x}{\log x} \le \pi(x) \le c_2 \frac{x}{\log x}.
    \]
    Our proof will give $c_1 = \frac{\log 2}{2} \approx 0.346$ and $c_2 = 6\log 2 \approx 4.158$.
\end{theorem}
\begin{proof}[Proof of the upper bound.]
    By Proposition \ref{5.13}, we have 
    \begin{align*}
        &n^{\pi(2n)-\pi(n)} \le N \le 2^{2n} \\
        \implies &\pi(2n)-\pi(n) \le 2 \log 2 \frac{n}{\log n} ~(\star).
    \end{align*}
    We prove by induction on $k$ that $\pi(2^k) \le 3 \frac{2^k}{k} ~\forall k\ge 1 ~(\dagger)$. This is obvious for $k\le 6$ as $\pi(x)\le \frac{x}{2} ~\forall x\ge 2$ even. Induction step: $$\pi(2^{k+1}) \stackrel{(\star)}{\le} \pi(2^k) + 2\log 2 \frac{2^k}{\log(2^k)} \stackrel{(\dagger)}{\le} 3\frac{2^k}{k} + 2\frac{2^k}{k} \le 6 \frac{2^k}{k+1} = 3\frac{2^{k+1}}{k+1}$$
    as $\frac{5}{k}\le \frac{6}{k+1}$ for $k\ge 5$.
    \vspace{1mm}
    
    $\frac{x}{\log x}$ is increasing for $~\forall x \ge e$ (as its derivative is $\frac{\log x-1}{(\log x)^2}$), so if $2^k\le x \le 2^{k+1}$, then 
    \[
    \pi(x) \le \pi(2^{k+1}) \le 3\frac{2^{k+1}}{k+1} < 6\frac{2^k}{k} = 6\log 2 \frac{2^k}{\log(2^k)}\le 6 \log 2 \frac{x}{\log x}.
    \]
\end{proof}
\begin{proof}[Proof of the lower bound.]
    By Proposition \ref{5.13}, we have 
    \begin{align*}
    &\frac{2^{2n}}{2n} \le N \le  (2n)^{\pi(2n)} \\
    \implies &2n \log 2 - \log(2n) \le \pi(2n) \log(2n) \\
    \implies &\pi(2n) \ge \log 2 \frac{2n}{\log(2n)} -1.
    \end{align*}
    So if $2n\le x\le 2n+2$, then $$\pi(x) \ge \pi(2n) \ge \log 2 \frac{(x-2)}{\log x}-1.$$
    To complete the proof, it is enough to show that \[
    \log 2 \frac{(x-2)}{\log x} - 1 \ge \frac{\log 2}{2} \frac{x}{\log x}.
    \]
    This is equivalent to $\frac{\log 2}{2}\frac{x}{\log x}\ge 1 + \frac{2\log2}{\log x}$, which is true for $x=16$ and hence for all $x\ge 16$ since the LHS is increasing and the LHS is decreasing.
    \vspace{1mm}
    
    Finally, if $4\le x\le 16$, then $\frac{\log 2}{2}\frac{x}{\log x} \le 2 \le \pi(x)$.
\end{proof}
\begin{theorem}[Bertrand's postulate]
    If $n>1$ is an integer, then there exists a prime with $n<p<2n$.
\end{theorem}
\begin{proof}
    Let $N={{2n}\choose{n}}$. If $\frac{2n}{3}<p\le n$, then 
    \begin{align*}
        &\nu_p((2n)!) = 2 \text{ as } 2p\le 2n<3p. \\
        &\nu_p((n)!) = 1 \text{ as } p\le n<2p.
    \end{align*}
    Hence $\nu_p(N)=0$. Suppose Bertrand's postulate is false. Then, using Proposition \ref{5.13} (ii),
    \begin{align*}
        N = \prod_{p\le \frac{2n}{3}}^{} p^{\nu_p(n)} \le \prod_{p \le \sqrt{2n}}^{} p^{\nu_p(n)}\prod_{p\le \frac{2n}{3}}^{} p \le (2n)^{\sqrt{2n}} \prod_{p\le \frac{2n}{3}}^{} p.
    \end{align*}
    On Example Sheet 3 we show that $\prod_{p\le m}^{} p = 4^m$, hence (again by Proposition \ref{5.13})
    \begin{align*}
        &\frac{2^{2n}}{2n} \le N \le (2n)^{\sqrt{2n}} 2^{\frac{4n}{3}} \\
        \implies& 2^{\frac{2n}{3}} \le (2n)^{1+\sqrt{2n}} \\
        \implies& 2n \log 2 \le 3(1+\sqrt{2n})\log2n.
    \end{align*}

    \marginpar{12 Nov 2022, Lecture 17}

    Choose $2n=2^{2x}$ (so $x = \frac{\log(2n)}{2\log 2}$), so
    \begin{align*}
        &\implies 2^{2x}\log 2 \le 3(1+2^x)2x \log 2 \\
        &\implies 2^x \le 6x(1+2^{-x})
    \end{align*}
    If $x>5$, say $x=5(y+1)$ for some $y>0$, we get
    \begin{align*}
        &2^{5y} \le \frac{6}{32}5(y+1)(1+\frac{1}{32})\le y+1< e^y \\
        &\implies 5y \log 2 < y,
    \end{align*}
    contradiction, so $x\le 5$ and so $n\le 2^9 = 512$. For $n<512$ it suffices to take $2,3,5,7,13,23,43,83,163,317,631$.
\end{proof}
\vspace{1mm}

\textbf{Legendre's formula.} Let $p_n$ be the $n^{\text{th}}$ prime.

\begin{defn}
   Let $N_r(x) = |\{1\le n\le x \mid n \text{ is coprime to }p_1,p_2,\ldots,p_r\}|$ and let $A_i = \{1\le n\le x \mid p_i \mid n\}$, $A_i^c = \{1\le n\le x \mid p_i \nmid n \}$.
\end{defn}
By the inclusion--exclusion principle,
\begin{align*}
    &N_r(x)=|\bigcap_{i=1}^r A_i^c| = \left\lfloor x \right\rfloor - \sum_{i}^{} |A_i| + \sum_{i<j}^{} |A_i \cap A_j| - \ldots +(-1)^r |A_1 \cap \ldots \cap A_r| = \\
    &=\left\lfloor x \right\rfloor - \sum_{i}^{} \left\lfloor \frac{x}{p_i} \right\rfloor + \sum_{i<j}^{} \left\lfloor \frac{x}{p_ip_j}  \right\rfloor-\ldots + (-1)^r \left\lfloor \frac{x}{p_1\ldots p_r} \right\rfloor.
\end{align*}
For ease of calculation, remember that $\left\lfloor \frac{x}{p_1p_2} \right\rfloor=\left\lfloor \frac{\left\lfloor \frac{x}{p_1} \right\rfloor}{p_2} \right\rfloor.$   

\begin{theorem}[Legendre's formula]
    Let $r=\pi(\sqrt{x})$. Then 
    \begin{align*}
        \pi(x) - \pi(\sqrt{x}) + 1 = N_r(x).
    \end{align*}
\end{theorem}
\begin{proof}
    Every composite integer $n\le x$ is divisible by some prime $\le \sqrt{x}$. So if $1\le n\le x$, then
    \begin{align*}
        n \text{ coprime to } p_1,\ldots,p_r \iff n=1 \text{ or }n \text{ is a prime with } \sqrt{x}<n\le x.
    \end{align*}
\end{proof}
\textbf{Remark.} If we set $P=p_1\ldots p_r$, then 
\begin{align*}
    &N_r(x) = |\{1\le n\le x \mid (n,P)=1\}| =\\
    =\sum_{n=1}^{\left\lfloor x \right\rfloor} \sum_{d \mid (n,P)}^{} &\mu(d) = \sum_{d \mid P}^{} \mu(d) \sum_{n=1}^{\left\lfloor x \right\rfloor} \mathbbm{1}_{\{d \mid n\}} = \sum_{d \mid P}^{} \mu(d) \left\lfloor \frac{x}{d} \right\rfloor, 
\end{align*}
which is the same formula as above.

\begin{defn}
    A \textbf{Dirichlet series} is a series of the form $\sum_{n=1}^{\infty} \frac{a^n}{n^s}$ for some sequence $a_1,a_2,\ldots \in \mathbb{C}$. 
\end{defn}
\textbf{Remark.} If $|a_n|\le \text{const}\cdot n^k$ for all $n$ large enough for some $k$, then the series converges for $\text{Re}(s)>k+1$.

Assuming absolute convergence, we can multiply two Dirichlet series:
\begin{align*}
    \sum_{m=1}^{\infty} \frac{a_m}{m^s} \sum_{n=1}^{\infty} \frac{b_n}{n^s} = \sum_{N=1}^{\infty} \frac{c_N}{N^s}
\end{align*}
where $N=mn$ and $c_N = \sum_{d \mid N}^{} a_db_{N/d}$.

For example, for $\text{Re}(s)>2$, we get 
\begin{align*}
    \zeta(s)\zeta(s-1) = \sum_{m=1}^{\infty} \frac{1}{m^s}\sum_{n=1}^{\infty} \frac{n}{n^s} = \sum_{N=1}^{\infty} \frac{\sigma(N)}{N^s}
\end{align*}
where $\sigma(N)=\sum_{d \mid N}^{} d$.
\vspace{1mm}

The following until the end of the section is now non-examinable.

\begin{defn}
    Define the \textbf{von Mangoldt function} as 
    \begin{align*}
        \Lambda(n) = \begin{cases}
            \log p &\text{ if } n=p^r \text{ is a prime power.}\\
            0 &\text{ otherwise.}
        \end{cases}
    \end{align*} 
\end{defn}
\begin{lemma}
    For $\text{Re}(s)>1$, we have \[
    \frac{\zeta'(s)}{\zeta(s)} = -\sum_{n=1}^{\infty} \frac{\Lambda(n)}{n^s}.
    \]
\end{lemma}
\begin{proof}
    \begin{align*}
        &\zeta(s) = \prod_{p}^{} \left(1-\frac{1}{p^s}\right)^{-1} \\
        &\implies \log \zeta(s) = -\sum_{p}^{} \log (1-p^{-s}) \\
        &\stackrel{\text{differentiate}}{\implies}  \frac{\zeta'(s)}{\zeta(s)} =- \sum_{p}^{} \frac{(\log p)p^{-s}}{1-p^{-s}} \\
        &\implies \frac{\zeta'(s)}{\zeta(s)} = -\sum_{p}^{} \log p \sum_{j=1}^{\infty} p^{-js} = - \sum_{n=1}^{\infty} \frac{\Lambda(n)}{n^s}.
    \end{align*}
\end{proof}
Let $\mathbbm{1}_{\text{prime}}(n) = \begin{cases}
    1 &\text{ if }p \text{ is prime.}\\
    0 &\text{ otherwise}.
\end{cases}$ Then $\pi(x) = \sum_{n\le x}^{} \mathbbm{1}_{\text{prime}}(x)$. We should think about the von Mangoldt function as a modified version of this indicator function. Indeed, let 
\[
    \psi(x) = \sum_{n\le x}^{} \Lambda(n).  
\]
On Example Sheet 3, we will show that $\psi(x) \sim \pi(x)\log x$ as $x \to \infty$. The Prime Number Theorem is then equivalent to $\psi(x) \sim x$ as $x \to \infty$. This is proved by integrating 
\begin{align*}
    \frac{\zeta'(s)}{\zeta(s)} \frac{x^{s+1}}{s(s+1)}
\end{align*} 
around a suitable contour.
\begin{theorem}[Dirichlet's theorem on primes in arithmatic progressions, 1839]
    Let $N>1$ be an integer and $a \in \mathbb{Z}$ with $(a,N)=1$. Then there are infinitely many primes $p$ with $p \equiv a \pmod{N}$.

    In other words, the arithmetic progression $a, a+N, a+2N,\ldots$ contains infinitely many primes.
\end{theorem}

Let $\chi : (\mathbb{Z}/n\mathbb{Z})^\times \to \mathbb{C}^{*}$ be a group homomorphism. Define $\overline{\chi} : \mathbb{Z} \to \mathbb{C}$ by $$a \mapsto \begin{cases}
    \chi(a) &\text{ if } (a,N)=1.\\
    0 &\text{ otherwise.}
\end{cases}$$
Let 
\begin{align*}
    L(s,\chi) = \sum_{n=1}^{\infty} \frac{\overline{\chi}(n)}{n^s},
\end{align*}
called the Dirichlet L--function.
\begin{itemize}
    \item  It can be shown that if $\chi \neq1$, then this converges for $\text{Re}(s)>0$.
    \item Like $\zeta$, this has an Euler product 
    \[
    L(s,\chi) =\prod_{p \nmid n }^{} \left(1-\frac{\chi(p)}{p^s}\right)^{-1}.
    \]
    In a neighborhood of $s=1$, we have $$\log L(s,\chi) = \sum_{p \nmid N}^{} \frac{\chi(p)}{p^s} + (\text{a function bounded near }s=1).$$
    Taking linear combinations of this formula (fixing $N$ and varying $\chi$), Dirichlet was able to show that 
    \[
    \sum_{p\equiv a\pmod{N}}^{} \frac{1}{p^s} \to \infty \text{ as } s \to 1,
    \]
    which then implies the theorem.
\end{itemize} 
The key step in the proof (that we completely glossed over) is to show that $L(1,\chi) \neq 0$ for $\chi \neq 1$.

\section{Continued fractions}

\marginpar{15 Nov 2022, Lecture 18}

The continued fraction algorithm systematically produces the best (for a given size of denominator) rational approximations to a given real number.

\subsubsection*{Description of the algorithm.}

We take $\theta \in \mathbb{R}$ (usually $\theta>0$), and define integers $a_0,a_1,a_2,\ldots$ as follows:
\vspace{1mm}

Let $a_0 = \left\lfloor \theta \right\rfloor$. Stop if $\theta = a_0$, otherwise write $\theta_0 = a_0 + \frac{1}{\theta_1}$.
\vspace{1mm}

Now let $a_1 = \left\lfloor \theta_1 \right\rfloor$. Stop if $\theta_1=a_1$, otherwise write $\theta_1 = a_1 + \frac{1}{\theta_2}$. Continue analogously.
\vspace{1mm}

If the algorithm stops, we get a \textbf{finite} continued fraction:
\[
\theta = a_0 + \frac{1}{a_1+\frac{1}{a_2+\frac{1}{\ldots+\frac{1}{a_n}}}} \stackrel{\text{def}}{=} [a_0,a_1,a_2,\ldots,a_n]. 
\]
Otherwise, the continued fraction is \textbf{infinite} and we write $\theta = [a_0,a_1,a_2,\ldots]$.
\begin{defn}
    $a_0,a_1,a_2,\ldots$ are called \textbf{partial quotients}.     
\end{defn}
\begin{lemma}
    The continued fraction of $\theta$ is finite $\iff$ $\theta \in \mathbb{Q}$.
\end{lemma}
\begin{proof}
    $(\implies)$ is clear, multiply out and we get a rational number.
    \vspace{1mm}
    
    $(\impliedby)$: Suppose $\theta \in \mathbb{Q}$, say $\theta=\frac{a}{b}$ for $a,b \in \mathbb{Z}, b>0$. By Euclid's algorithm, write 
    \begin{align*}
        &a=a_0b+r_1,~ 0\le r_1<b  &\theta=\frac{a}{b}=a_0+\frac{r_1}{b}\\
        &b=a_1r_1+r_2,~ 0\le r_2<r_1 &\theta_1 = \frac{b}{r_1}=a_1+\frac{r_2}{r_1}\\
        &r_1=a_2r_2+r_3, ~0\le r_3<r_2 &\theta_2 = \frac{r_1}{r_2}=a_2+\frac{r_3}{r_2}.
    \end{align*}
    We eventually get $r_n=0$, so the algorithm stops.
\end{proof}
Let $[a_0,a_1,a_2,\ldots]$ be an infinite continued fraction. This may be approximated by the finite continued fraction $[a_0,a_1,\ldots,a_n]$.
\vspace{1mm}

\textbf{Motivation}:
\begin{align*}
    &[a_0]=a_0 \\
    &[a_0,a_1]=a_0+\frac{1}{a_1}=\frac{a_0a_1+1}{a_1}\\
    &[a_0,a_1,a_2]=a_0 + \frac{a_2}{a_1a_2+1} = \frac{a_0a_1a_2+a_0+a_2}{a_1a_2+1}
\end{align*}
\begin{defn}
    Given $a_0,a_1,a_2,\ldots$, we define sequences $(p_n)$ and $(q_n)$ as follows:
    \begin{align*}
        &p_0=a_0 &q_0=1 \\
        &p_1=a_0a_1+1 &q_1=a_1 \\
        &p_n=a_np_{n-1}+p_{n-2} &q_n=a_nq_{n-1}+q_{n-2} ~\forall n\ge 2. 
    \end{align*}
\end{defn}
\textbf{Remark.} The $(q_n)$ are an increasing sequence of positive integers.
\begin{lemma}\label{6.2}
    \begin{enumerate}[(i)]
        \item For $n\ge 0$, we have $\frac{p_n}{q_n}=[a_0,a_1,\ldots,a_n]$.
        \item Let $\beta>0$ be a real number. For $n\ge 2$, we have $$\frac{\beta p_{n-1}+p_{n-2}}{\beta q_{n-1}+q_{n-2}}=[a_0,a_1,\ldots,a_{n-1},\beta].$$
    \end{enumerate}
\end{lemma}
\begin{proof}
    \begin{enumerate}[(i)]
        \item Check above for $n=0,1$. The general case follows by part (ii) with $\beta=a_n$.
        \item Induction on $n$. If $n=2$, $[a_0,a_1,\beta]=\frac{\beta(a_0a_1+1)+a_0}{\beta a_1+1} = \frac{\beta p_1 + p_0}{\beta q_1 + q_0}$.
        \vspace{1mm}
        
        Suppose the claim is now true for $n$. Then
        \begin{align*}
            [a_0,\ldots,a_n,\beta] = [a_0,a_1,\ldots,a_{n-1},a_n+\frac{1}{\beta}]=\\
            =\frac{(a_n+\frac{1}{\beta})p_{n-1}+p_{n-2}}{(a_n+\frac{1}{\beta})q_{n-1}+q_{n-2}} = \frac{\beta p_n + p_{n-1}}{\beta q_n + q_{n-1}},
        \end{align*}
        so the result is true for $n+1$ by induction.
    \end{enumerate}
\end{proof}
\begin{defn}
    The fraction $\frac{p_n}{q_n}$ is called a \textbf{convergent} to $\theta$.
\end{defn}
\begin{lemma}\label{6.3}
    \begin{enumerate}[(i)]
        \item For $n\ge 1$, we have $p_nq_{n-1}-p_{n-1}q_n = (-1)^{n-1}$.
        \item For $n\ge 2$, we have $p_nq_{n-2}-p_{n-2}q_n = (-1)^na_n$.
    \end{enumerate}
\end{lemma}
\begin{proof}
    \begin{enumerate}[(i)]
        \item By induction on $n$. For $n=1, p_1q_0-p_0q_1=(a_0a_1+1)-a_0a_1=1$.
        Assuming the claim is now true for $n-1$, we get 
        \begin{align*}
            &p_nq_{n-1}-p_{n-1}q_n =\\& (a_np_{n-1}+p_{n-2})q_{n-1}+p_{n-1}(a_nq_{n-1}+q_{n-2}) = \\
            &-(p_{n-1}q_{n-2}-p_{n-2}q_{n-1})=(-1)^{n-1}.
        \end{align*}
        \item
        \begin{align*}
            &p_nq_{n-2}-p_{n-2}q_n =\\& (a_np_{n-1}+p_{n-2})q_{n-2}-p_{n-2}(a_nq_{n-1}+q_{n-2}) =\\
            & a_n(p_{n-1}q_{n-2}-p_{n-2}q_{n-1}) =(-1)^na_n
        \end{align*}
        by part (i).
    \end{enumerate}
\end{proof}
\textbf{Remarks.}
\begin{itemize}
    \item Lemma \ref{6.3} (i) shows that $p_n$ and $q_n$ are coprime and that $$\frac{p_n}{q_n}-\frac{p_{n-1}}{q_{n-1}} = \frac{(-1)^{n-1}}{q_nq_{n-1}} \stackrel{n \to \infty}{\to} 0.$$
    \item Lemma \ref{6.3} (ii) shows that \[
    \frac{p_n}{q_n} - \frac{p_{n-2}}{q_{n-2}} = \frac{(-1)^na_n}{q_nq_{n-2}}.
    \]
    \item Therefore \[
    \frac{p_0}{q_0}<\frac{p_2}{q_2}<\frac{p_4}{q_4}<\ldots < \frac{p_5}{q_5}<\frac{p_3}{q_3}<\frac{p_1}{q_1}.
    \]
    So the limit as $\lim_{n \to \infty}\frac{p_n}{q_n}$ exists. We will now show that this limit is exactly the number $\theta$, which justifies the notation $\theta = [a_0,a_1,\ldots]$ and calling $\frac{p_n}{q_n}$ the convergent fractions.
\end{itemize} 
\marginpar{17 Nov 2022, Lecture 19}
\begin{theorem}\label{6.4}
    Let $\theta$ be an irrational number. Define $a_n,p_n,q_n$ as above. Then for all $n\ge 0$, we have
    \[
    \left|\theta - \frac{p_n}{q_n}\right| < \frac{1}{q_nq_{n-1}} < \frac{1}{q_n^2}.
    \]
\end{theorem}
\begin{proof}
    By the continued fraction algorithm, $\theta = [a_0,a_1,a_2,\ldots,a_n,\theta_{n+1}]$, where $\left\lfloor \theta_{n+1} \right\rfloor=a_{n+1}$. By Lemma \ref{6.2},
    \[
    \theta - \frac{p_n}{q_n}= \frac{\theta_{n+1} p_n + p_{n-1}}{\theta_{n+1} q_n + q_{n-1}} - \frac{p_n}{q_n} = \frac{p_{n-1}q_n-p_nq_{n-1}}{q_n(\theta_{n+1}q_n+q_{n-1})} = \frac{(-1)^n}{q_n(\theta_{n+1}q_n+q_{n-1})}.
    \]
    But $\theta_{n+1}q_n+q_{n-1} > a_{n+1}q_nq_{n-1} = q_{n+1}$, hence
    \[
        \left|\theta - \frac{p_n}{q_n}\right| < \frac{1}{q_nq_{n-1}}
    \]
    as desired.
\end{proof}
\begin{cor}\label{6.5}
    $\frac{p_n}{q_n}\to \theta$ as $n \to \infty$. In particular,
    \[
        \frac{p_0}{q_0}<\frac{p_2}{q_2}<\frac{p_4}{q_4}<\ldots < \theta < \ldots< \frac{p_5}{q_5}<\frac{p_3}{q_3}<\frac{p_1}{q_1}.
    \]
\end{cor}
\begin{theorem}\label{6.6}
    Let $\theta$ be an irrational number. Let $p,q \in \mathbb{Z}$ with $0<q<q_{n+1}$. Then \[
    |q \theta - p| \ge |q_n \theta - p_n|.
    \]
\end{theorem}
\begin{cor}\label{6.7}
    If $p,q \in\mathbb{Z}, q>0$ with $|\theta-\frac{p}{q}|<|\theta-\frac{p_n}{q_n}|$, then $q>q_n$.
\end{cor}
\begin{proof}[Proof of Corollary \ref{6.7}]
    Suppose $q\le q_n$. Then by Theorem \ref{6.6}, \[
    \left|\theta-\frac{p}{q}\right| \ge \frac{q_n}{q} \left|\theta-\frac{p_n}{q_n}\right| \ge \left|\theta-\frac{p_n}{q_n}\right|,
    \]
    contradiction, so $q>q_n$.
\end{proof}
\begin{proof}[Proof of Theorem \ref{6.6}]
    Let us write $p = u p_n + v p_{n+1}, q=uq_n+vq_{n-1}$. Since $p_nq_{n+1}-p_{n+1}q_n = \pm 1$, we can find $u,v \in \mathbb{Z}$ satisfying the equations. If $v=0$, then the result is clear, so suppose $v \neq 0$. The hypothesis $0<q<q_{n+1}$ now implies that $u\neq 0$ and that $u,v$ have opposite signs. Now $$q \theta - p  = u(q_n \theta - p_n) + v (q_{n+1} \theta - p_{n+1}).$$
    By Corollary \ref{6.5}, $q_{n} \theta - p_n$ and $q_{n+1} \theta - p_{n+1}$ have opposite signs, which implies that $u (q_n \theta - p_n)$ and $v(q_{n+1}\theta - p_{n+1})$ have the same sign. Hence (as $u\neq 0$) \[
    |q \theta - p| = |u(q_n \theta - p_n)| + |v(q_{n+1}\theta - p_{n+1})|\ge |q_n \theta - p_n|.
    \]
\end{proof}
\begin{theorem}\label{6.8}
    Let $\theta$ be an irrational number. 
    \begin{enumerate}[(i)]
        \item At least one of any two successive convergents satisfies $$\left|\theta-\frac{p}{q}\right|<\frac{1}{2q^2}.$$ 
        \item Conversely, if $p,q \in \mathbb{Z}, q>0$ satisfy $\left|\theta-\frac{p}{q}\right|<\frac{1}{2q^2}$, then $\frac{p}{q}$ is a convergent, i.e. $\frac{p}{q}=\frac{p_n}{q_n}$ for some $n$.
    \end{enumerate}
\end{theorem}
\begin{proof}
    \begin{enumerate}[(i)]
        \item Since $\theta-\frac{p_n}{q_n}$ and $\theta-\frac{p_{n+1}}{q_{n+1}}$ have opposite signs, Lemma \ref{6.3} gives \[
        \left|\theta-\frac{p_n}{q_n}\right| + \left|\theta - \frac{p_{n+1}}{q_{n+1}}\right| = \left|\frac{p_n}{q_n}-\frac{p_{n+1}}{q_{n+1}} \right| = \frac{1}{q_nq_{n+1}}.
        \]
        If $\alpha \neq \beta$ are real numbers, then $(\alpha-\beta)^2>0 \implies \alpha \beta < \frac{1}{2}(\alpha^2+\beta^2)$, so \[
        \left|\theta - \frac{p_n}{q_n} \right| + \left|\theta-\frac{p_{n+1}}{q_{n+1}}   \right| < \frac{1}{2} \left(\frac{1}{q_n^2}+\frac{1}{q_{n+1}^2} \right).
        \] This proves (i).
        \item Assume $\left|\theta-\frac{p}{q}\right|<\frac{1}{2q^2} ~(\dagger)$. Choose $n$ such that $q_n\le q <q_{n+1}$. Then \[
        \left|\frac{p}{q}-\frac{p_n}{q_n}\right| \le \left|\theta-\frac{p}{q}\right| + \left|\theta-\frac{p_n}{q_n}\right| = \frac{1}{q} |q \theta - p| + \frac{1}{q_n} |q_n \theta - p_n|.
        \]
        As $q<q_{n+1}$, Theorem \ref{6.6} implies $|q \theta - p| \ge |q_n \theta - p_n|.$ Hence \[
            \left|\frac{p}{q}-\frac{p_n}{q_n}\right| \le \left(\frac{1}{q}+\frac{1}{q_n}\right) \underbrace{|q \theta - p|}_{\le \frac{1}{2q} \text{ by }(\dagger)}
        \]
        Since $q_n\le q$, we get \[
            \left|\frac{p}{q}-\frac{p_n}{q_n}\right| < \frac{1}{q_n q}.
        \]
        But $\left|\frac{p}{q}-\frac{p_n}{q_n}\right| = \left|\frac{pq_n-p_nq}{q_nq}\right|$, so if this is nonzero, then it is $\ge \frac{1}{qq_n}$, so hence $\frac{p}{q}=\frac{p_n}{q_n}$ and we're done.
    \end{enumerate}
\end{proof}
\begin{example}
    Compute the continued fraction for $\theta = \sqrt{14}$. We have   
    \begin{align*}
        &\theta = 3 + (\sqrt{14}-3).\\
        &\theta_1 = \frac{1}{\sqrt{14}-3} = \frac{\sqrt{14}+3}{5} = 1 + \frac{\sqrt{14}-2}{5}.\\
        & \theta_2 = \frac{5}{\sqrt{14}-2} = \frac{\sqrt{14}-2}{2} = 2 + \frac{\sqrt{14}-2}{2}.\\
        &\theta_3 = \frac{2}{\sqrt{14}-2} = \frac{\sqrt{14}+2}{5} = 1 + \frac{\sqrt{14}-3}{5}.\\
        &\theta_4 = \frac{5}{\sqrt{14}-3} = \sqrt{14}+3 = 6 + (\sqrt{14}-3).
    \end{align*}
    We now see this repeats, hence $\theta_5 = \theta_1$. We hence write \[
    \sqrt{14} = [3,\overline{1,2,1,6}].
    \]
\end{example}
Let us tabulate our results. We also keep track of $p_n^2-14q_n^2$ in the last column (since we will be looking at Pell's equations soon), from which we conclude that $x^2-14y^2=1$ has solutions, since we found $(x,y)=(15,4)$ and $(x,y)=(449,120)$.
\begin{center}
    \begin{tabular}{l|llll}
    $n$ & $a_n$ & $p_n$ & $q_n$ & $p_n^2-14q_n^2$\\ \hline
    0 & 3 & 3 & 1 & \hspace{6mm}-5\\
    1 & 1 & 4 & 1 & \hspace{7mm}2\\
    2 & 2 & 11 & 3 & \hspace{6mm}-5\\
    3 & 1 & 15 & 4 & \hspace{7mm}1\\
    4 & 6 & 101 & 27 & \hspace{6mm}-5\\
    5 & 1 & 116 & 31 & \hspace{7mm}2\\
    6 & 2 & 333 & 89 & \hspace{6mm}-5\\
    7 & 1 & 449 & 120 & \hspace{7mm}1\\
    \end{tabular}
\end{center}

\marginpar{19 Nov 2022, Lecture 20}

\begin{defn}
    A continued fraction is \textbf{periodic} if it is of the form \[
    [a_0,a_1,\ldots,a_{m-1},\overline{a_m,a_{m+1},\ldots,a_{m+n-1}}]
    \]
    and \textbf{purely periodic} if $m=0$.
\end{defn}
\textbf{Remark.} If $\phi=[\overline{a_0,\ldots,a_{n-1}}]$, then $\phi=[a_0,\ldots,a_{n-1},\phi] = \frac{\phi p_{n-1}-p_{n-2}}{\phi q_{n-1}-q_{n-2}}$, hence $a \phi^2 + b \phi + c = 0$ for some $a,b,c \in \mathbb{Z}, a\neq 0$. Hence $\phi = r + s \sqrt{d}$ for some $r,s \in \mathbb{Q}, s\neq 0, d>1$ squarefree. We say $\phi$ is a \textbf{quadratic irrational}.

\begin{theorem}[Lagrange]
    The continued fraction of $\theta$ is periodic $\iff$ $\theta$ is a quadratic irrational.
\end{theorem}
\begin{proof}
    $(\implies )$: We have $\theta = [a_0,\ldots,a_{n-1},\phi]$, where $\phi$ is purely periodic. By the last remark, $\phi$, and hence $\theta$, is a quadratic irrational.
    \vspace{1mm}
    
    $(\impliedby)$: Suppose $\theta$ is a root of $ax^2+bx+c=0$ for some $a,b,c \in \mathbb{Z}, a\neq0$. Let $f(x,y)=ax^2+bxy+cy^2$, so $f(\theta,1)=0$. If $\theta = [a_0,\ldots,a_n,\theta_{n+1}] = \frac{p_n \theta_{n+1} + p_{n-1}}{q_n \theta_{n+1} + q_{n-1}}$, then $f(p_n \theta_{n+1} + p_{n-1}, q_n \theta_{n+1} + q_{n-1})= 0 \implies \theta_{n+1}$ is a root of some quadratic $A_n x^2 + B_n x + C_n=0$ for some $A_n,B_n,C_n \in \mathbb{Z}$, where $A_n = f(p_n,q_n), C_n = f(p_{n-1}, q_{n-1})$, and $B_n^2-4A_nC_n = b^2-4ac$ (since $p_nq_{n-1}-p_{n-1}q_n=\pm 1$).
    \vspace{1mm}
    
    Now the claim is that $f(p_n,q_n)$ is bounded independently of $n$.
    
    Assuming the claim, there are only finitely many possibilities for $A_n,B_n,C_n$, so there are only finitely many possibilities for $\theta_{n+1}$, so eventually we have $\theta_r=\theta_s$ for some $r \neq s$, so the continued fraction is periodic.
    \begin{proof}[Proof of claim]
        \begin{align*}
            &f\left(\frac{p_n}{q_n},1\right) - f(\theta,1) = a\left(\left(\frac{p_n}{q_n}\right)^2 - \theta^2\right) + b\left(\frac{p_n}{q_n}-\theta\right) = \\
            &\underbrace{\left(a \left(\frac{p_n}{q_n}+\theta\right) + b\right)}_{\to 2 a \theta + b \text{ as }n \to \infty } \cdot  \underbrace{\left(\frac{p_n}{q_n}- \theta\right)}_{|\cdot|<\frac{1}{q_n^2} \text{ by Theorem }\ref{6.4}}.
        \end{align*}
        Hence \[
        \left|f \left(\frac{p_n}{q_n},1\right)\right|\le \text{const} \cdot \frac{1}{q_n^2} \implies |f(p_n,q_n)|\le \text{const}.
        \]
    \end{proof}
\end{proof}

\subsection*{Pell's equation}
Let $d \in \mathbb{N}$ for $d$ not a square. We seek to solve $x^2-dy^2 = 1$ for $x,y \in \mathbb{Z}$. 

We see we have the trivial solutions $(x,y)=(\pm 1,0)$.

\begin{lemma}
    If $(x,y) \in \mathbb{N}$ satisfy $x^2-dy^2=1$, then $\frac{x}{y}$ is a convergent to $\sqrt{d}.$ 
\end{lemma}
\begin{proof}
    $x^2-dy^2=1 \implies (x-\sqrt{d}y)(x+\sqrt{d}y)=1$.Thus 
    \begin{align*}
        &0<x - \sqrt{d}y = \frac{1}{x+ \sqrt{d}y} < \frac{1}{2\sqrt{d}y} < \frac{1}{2y} \\
        &\left|\frac{x}{y}-\sqrt{d}\right|<\frac{1}{2y^2},
    \end{align*}
    so we're done by Theorem \ref{6.8}.
\end{proof}
\begin{lemma}\label{6.11}
    Let $\theta$ be a quadratic irrational with conjugate $\theta'$ (i.e. $\theta,\theta'$ are roots of $ax^2+bx+c=0$ for $a,b,c \in \mathbb{Z}, a \neq 0$). Then the continued fraction of $\theta$ is purely periodic $\iff$ $\theta>1$ and $-1<\theta'<0$.
\end{lemma}
\begin{proof}
    Omitted, but can be found in Baker's \textit{A concise introduction to the theory of numbers}. 
\end{proof}

Let $d \in \mathbb{N}$ not a square. Let $a_0 = \left\lfloor \sqrt{d} \right\rfloor$, so $\sqrt{d}=a_0 + \frac{1}{\theta_1}$. Applying Lemma \ref{6.11} to $\theta_1 = \frac{1}{\sqrt{d}-a_0}$ shows that $\sqrt{d}=[a_0,\overline{a_1,\ldots,a_n}]$ (e.g. $\sqrt{14} = [3,\overline{1,2,1,6}]$ as we saw before).

\begin{theorem}\label{6.12}
    Let $d \in \mathbb{N}$ not a square. Then $x^2-dy^2 =1$ has a nontrivial solution in integers $x$ and $y$.
\end{theorem}
\begin{proof}
    Write $\sqrt{d} = [a_0, \overline{a_1,\ldots,a_n}]$.
    \vspace{1mm}
    
    We then have 
    \begin{align*}
        &\sqrt{d}=[a_0,\theta_1]=a_0+\frac{1}{\theta_1} \implies \theta_1 = \frac{1}{\sqrt{d}-a_0} \\
        &\sqrt{d}=[a_0,a_1,\ldots,a_n,\theta_1]= \frac{\theta_1 p_n + p_{n-1}}{\theta_1 q_n + q_{n-1}} \\
    \end{align*}
    \begin{align*}
        &\implies \sqrt{d} = \frac{p_n + (\sqrt{d}-a_0)p_{n-1}}{q_n + (\sqrt{d}-a_0)q_{n-1}} \\
        &\implies (q_n + (\sqrt{d}-a_0)q_{n-1})\sqrt{d} = p_n + (\sqrt{d}-a_0)p_{n-1} \\
        &\implies dq_{n-1} + (q_n-a_0q_{n-1})\sqrt{d} = p_n - a_0p_{n-1} + p_{n-1}\sqrt{d}.
    \end{align*}
    As $\sqrt{d}$ is irrational, equate coefficients to get 
    \begin{align*}
        \begin{cases}
            p_{n-1} = q_n - a_0q_{n-1}\\
            dq_{n-1} = p_n - a_0p_{n-1}
        \end{cases}
        \implies p_{n-1}^2 - d q_{n-1}^2 = p_{n-1}q_n - p_n q_{n-1} =(-1)^n.
    \end{align*}
    Hence Pell's equation has a solution \[
    (x,y) = \begin{cases}
        (p_{n-1},q_{n-1}) &\text{ if }n \text{ is even.}\\
        (p_{2n-1},q_{2n-1}) &\text{ if }n \text{ is odd.}
    \end{cases}
    \]
\end{proof}
\begin{prop}
    Let $U_d = \{x+y\sqrt{d} \mid x,y \in \mathbb{Z}, x^2-dy^2 = 1\}$. Then 
    \begin{enumerate}[(i)]
        \item $U_d$ is a subgroup of $\mathbb{R}^*$.
        \item If $0<a<b$, then $U_d \cap [a,b]$ is finite.
        \item $U_d \neq \{\pm 1\}$.
    \end{enumerate}
\end{prop}
\begin{proof}
    \begin{enumerate}[(i)]
        \item Left as an exercise, not hard.
        \item $a\le x + y \sqrt{d} \le b $ and $\frac{1}{b} \le x-y\sqrt{d} = \frac{1}{x+y\sqrt{d}} \le \frac{1}{a}$, so $x,y$ are bounded.
        \item See Theorem \ref{6.12}.
    \end{enumerate}
\end{proof}
We conclude that $U_d$ contains a least element $>1$, say $x_0+y_0\sqrt{d}$. Then $U_d = \{\pm (x_0+y_0\sqrt{d})^m \mid m \in \mathbb{Z}\}$.

\section{Primality testing \& factoring}
\marginpar{22 Nov 2022, Lecture 21}


    We have two key questions:
    \begin{enumerate}[(i)]
        \item Given a large integer $N$, can we efficiently determine if $N$ is prime?
        \item Given a large composite integer $N$, can we find a non--trivial factor of $N$?
    \end{enumerate}

Trial division up to $\sqrt{N}$ is not efficient. However, small factors are easily found by trial division. Accordingly we assume throughout this section that $N$ is odd.
\vspace{1mm}

Fermat's little theorem states that if $p$ is prime and $(a,p)=1$, then $a^{p-1} \equiv 1 \pmod{p}$. 
\vspace{1mm}

So for example, to prove $15$ is not prime, we can check that $2^{14}\equiv 4 \not\equiv 1 \pmod{15}$.
\begin{defn}
    $b$ is a \textbf{base} for $N$ if $(b,N)=1$. We usually take ${b \in \{1,2,\ldots,N-1\}}$.
\end{defn}
\begin{defn}
    A composite integer $N$ is a (Fermat) \textbf{pseudoprime}  to the base $b$ if $b^{N-1} \equiv 1\pmod{N}$.
\end{defn}
\begin{example}
    91 is a pseudoprime to the base $3$, as we can check:
    \begin{align*}
        3^6 \equiv 1 \pmod{7},~ 3^6 \equiv 1 \pmod{13} \implies 3^6 \equiv 1 \pmod{91} \implies 3^{90} \equiv 1 \pmod{91}.
    \end{align*}
\end{example}
\begin{lemma}
    For every integer $b>1$, there exist infinitely many pseudoprimes to the base $b$.
\end{lemma}
\begin{proof}
    Let $p$ be any prime not dividing $2b(b^2-1)$. Let $N=\frac{b^{2p}-1}{b^2-1} \in \mathbb{Z}$.
    \begin{enumerate}[(i)]
        \item $N$ is composite: $N = \frac{b^p-1}{b-1}\frac{b^p+1}{b+1}$, where both factors are $\in \mathbb{Z}$ and $>1$.
        \item If we can show that $2p \mid (N-1)$, say $N-1=2pm$ for some $m \in\mathbb{Z}$, then \[
        b^{N-1} \equiv (b^{2p})^m \equiv 1 \pmod{N}.
        \]
        But $N-1 = \frac{b^{2p}-b^2}{b^2-1}=b^{2p-2}+b^{2p-4}+\ldots+b^4+b^2$, where the RHS is a sum of $p-1$ terms, all with the same parity, so as $p-1$ is even, $2 \mid (N-1)$.
        \vspace{1mm}
        
        $(b,p)=1 \implies b^{p-1} \equiv 1\pmod{p} \implies b^{2p} -b^2 \equiv 0 \pmod{p}$, so $p \mid (b^{2p}-b^2)$, but $p \nmid b^2-1$, so $p \mid (N-1)$.
        \vspace{1mm}
        
        Since $p$ is odd, it follows that $2p \mid (N-1)$, so we're done.
    \end{enumerate}
\end{proof}
\textbf{Remark.} If $b_1^{N-1} \equiv 1\pmod{N}$, $b_2^{N-1} \equiv 1 \pmod{N}$, then $(b_1b_2)^{N-1} \equiv 1\pmod{N}$. So for $N$ an odd composite integer, $$\{b \in (\mathbb{Z}/n\mathbb{Z})^\times \mid N \text{ is a pseudoprime to the base }b\}$$ is a subgroup of $(\mathbb{Z}/n\mathbb{Z})^\times$. So if $\exists $ base $b$ such that $N$ is not a pseudoprime to the base $b$, then $N$ is not a pseudoprime for at least half the bases (since a proper subgroup has index $\ge 2$).

\textbf{Problem:} $N$ might be a pseudoprime to all bases $b$. Such $N$ are called \textbf{Carmichael numbers} (see Example Sheet 4 for more on these).

\begin{theorem}[Alford, Pomerance, Granville (1994)]
    There are infinitely many Carmichael numbers.
\end{theorem}
In fact, they showed $|\{\text{Carmichael numbers }\le x\}|>x^{2/7}$ for $x$ sufficiently large.
\vspace{1mm}

Euler's criterion says that if $p$ is an odd prime and $(a,p)=1$, then $a^{\frac{p-1}{2}} \equiv \left( \frac{a}{p} \right) \pmod{p}$.
\begin{defn}
    An odd composite integer $N$ is an \textbf{Euler pseudoprime} to the base $b$ if $b^{\frac{N-1}{2}} \equiv \left( \frac{b}{N} \right) \pmod{N}$ (where the RHS is a Jacobi symbol). 
\end{defn}
\begin{example}
    91 is not an Euler pseudoprime to the base 3. Indeed,
    \begin{align*}
        3^6 \equiv 1 \pmod{91} \implies 3^{45} \equiv 3^{3} \equiv 27 \pmod{91}.
    \end{align*}
\end{example}
\textbf{Remark.} Since $\left( \frac{b_1b_2}{N} \right) =\left( \frac{b_1}{N} \right) \left( \frac{b_2}{N} \right)$ for Jacobi symbols, we again have that for $N$ odd and composite that \[
\{b \in (\mathbb{Z}/n\mathbb{Z})^\times \mid N\text{ is an Euler pseudoprime to the base }b\}
\]
is a subgroup of $(\mathbb{Z}/n\mathbb{Z})^\times$.
\begin{theorem}
    Let $N$ be an odd composite integer. Then $\exists $ a base $b$ such that $N$ is not an Euler pseudoprime to the base $b$ (i.e. there is no analogue of the Carmichael numbers).
\end{theorem}
\begin{proof}
    \begin{itemize}
        \item Case 1: $N$ is squarefree. Write $N=pm$ for $p \nmid m, m\ge 3$. Pick $u \in \mathbb{Z}$ such that $\left( \frac{u}{p} \right) =-1$. By CRT, $\exists b \in \mathbb{Z}$ such that $\begin{cases}
            &b \equiv u \pmod{p}\\
            &b \equiv 1 \pmod{m}
        \end{cases}$ (using here that $p$ and $m$ are coprime).
        \vspace{1mm}
        
        Then $\left( \frac{b}{N} \right) = \left( \frac{b}{p} \right) \left( \frac{b}{m} \right) = -1$. But if $b^{\frac{N-1}{2}} \equiv -1 \pmod{N}$, then $1 \equiv -1 \pmod{m}$, a contradiction.
        \item Case 2: $p^2 \mid N$. Write $N=p^r m$ for $r\ge 2$ and $p \nmid m$. By CRT, $\exists b \in\mathbb{Z}$ such that $\begin{cases}
            &b \equiv p+1 \pmod{p^2}.\\
            &b \equiv 1 \pmod{m}.
        \end{cases}$
        Then $(b,N)=1$ and \[
        b^{N-1} \equiv (1+p)^{N-1} \equiv 1 + (N-1)p \not\equiv  1 \pmod{p^2}.
        \]
        So $b^{N-1} \not\equiv 1 \pmod{N}$. So $N$ is not a Fermat pseudoprime to the base $b$, hence not an Euler pseudoprime to the base $b$.
    \end{itemize}
\end{proof}
\begin{cor}
    \marginpar{24 Nov 2022, Lecture 22}
    We have \[
    \left|\{b \in (\mathbb{Z}/n\mathbb{Z})^\times \mid N \text{ is an Euler pseudoprime to base }b\}\right| \le \frac{1}{2}\phi(N).
    \]
\end{cor}
\textbf{Soloray--Strassen primality test.}
\vspace{1mm}

Given $N$ an odd integer, we:
\begin{itemize}
    \item Test whether $b^{\frac{N-1}{2}} \equiv \left( \frac{b}{N} \right) \pmod{N}$ holds for some randomly chosen $b \in \{1,2,\ldots,N-1\}$.
    \item If any of these fail, then $N$ is composite.
    \item The probability that a composite number passes the test $k$ times for $k$ different bases is $\le \frac{1}{2^k}$.
\end{itemize} 
This is a probabilistic primality test, but it is very useful for practical purposes.
\vspace{1mm}

\textbf{Remark.} If $p$ is a prime, then $x^2 \equiv 1\pmod{p} \implies x \equiv \pm 1 \pmod{p}$.
\vspace{1mm}

Suppose $p$ is an odd prime and $a \in \mathbb{Z}$ such that $(a,p)=1$. Fermat's little theorem tells us that $a^{p-1} \equiv 1\pmod{p} \implies a^{\frac{p-1}{2}} \equiv \pm 1\pmod{p}$.
\vspace{1mm}

If $p\equiv 1\pmod{4}$ and $a^{\frac{p-1}{2}} \equiv 1 \pmod{p}$, then we deduce $a^{\frac{p-1}{4}} \equiv \pm 1\pmod{p}$. If $p\equiv 1\pmod{8}$ and $a^{\frac{p-1}{4}} \equiv 1\pmod{p}$, then $a^{\frac{p-1}{8}}\equiv \pm 1\pmod{p}$ and so on.

\begin{defn}[Miller--Rabin]
    Let $N$ be an odd integer. Write $N-1=2^s t$, where $s\ge 1$ and $t$ is odd. We say $N$ passes the strong (also known as Miller--Rabin) test to the base $b$ if either \[
    b^t \equiv 1\pmod{N} \text{ or } b^{2^rt}\equiv -1\pmod{N}\text{ for some }0\le r<s.
    \] 
    If $N$ fails the strong test, then it is certainly composite. A composite number that passes the test is called a \textbf{strong pseudoprime} to the base $b$.
\end{defn}
\begin{theorem}\label{7.5}
    If $N$ is a strong pseudoprime to the base $b$, then it is an Euler pseudoprime.
\end{theorem}
\begin{proof}
    Omitted, see e.g. Koblitz' \textit{A Course in Number Theory and Cryptography}. 
\end{proof}
\begin{theorem}\label{7.6}
    Let $N>9$ be an odd composite integer. Then \[
    |\{b \in (\mathbb{Z}/n\mathbb{Z})^\times \mid N\text{ is a strong pseudoprime to the base }b\}| \le \frac{1}{4}\phi(N).
    \]
\end{theorem}
We will prove this in the special case where $N$ is squarefree (but the general case is not too much harder).
\begin{lemma}
    Let $p$ be an odd prime and $m \in \mathbb{N}$ with $\nu_2(m)<\nu_2(p-1)$. Then
    \begin{enumerate}[(i)]
        \item $x^m \equiv 1\pmod{p}$ has exactly $(m,p-1)$ solutions.
        \item $x^m \equiv -1\pmod{p}$ also has exactly $(m,p-1)$ solutions.
    \end{enumerate}
\end{lemma}
\begin{proof}[Outline of proof.]
    \begin{enumerate}[(i)]
        \item Use the fact that $(\mathbb{Z}/p\mathbb{Z})^\times$ is cyclic of order $p-1$.
        \item $x^m \equiv -1\pmod{p} \iff \begin{cases}
            &x^{2m} \equiv 1\pmod{p}.\\
            &x^m \not\equiv 1 \pmod{p}.
        \end{cases}$

        But $\nu_2(m)<\nu_2(p-1) \implies (2m,p-1)=2(m,p-1)$, so apply (i) to both $m$ and $2m$ to conclude.
    \end{enumerate}
\end{proof}
\begin{proof}[Proof of Theorem \ref{7.6} for $N$ squarefree]
    Let $N=\prod_{i=1}^{k} p_1\ldots p_k$ for distinct primes, $k\ge 2$. Write $N-1=2^st$ and $p_i-1=2^{s_i}t_i$, where $s,s_i\ge 1$ and $t,t_i$ are odd. Note that $p_i \equiv 1 \pmod{2^{s_i}} \implies N \equiv 1\pmod{2^{\min(s_i)}} \implies {\min(s_i)\le s ~(\dagger)}$.
    \vspace{1mm}
    
    Suppose $N$ is a strong pseudoprime to the base $b$, so either $b^t \equiv 1\pmod{N}$ or $b^{2^rt} \equiv -1\pmod{N}$ for some $0\le r<s$.
    \vspace{1mm}
    
    In the latter case we have $b^{2^rt} \not\equiv 1 \pmod{p_i}, b^{2^{r+1}t} \equiv 1 \pmod{p_i}$. Hence $2^{r+1}$ divides the order of $b$ in $(\mathbb{Z}/p_i\mathbb{Z})^{\times}$ and hence divides $p_i-1=2^{s_i}t_i \implies r+1\le s_i$. Repeating this for all $i$ gives us that $r+1\le \min(s_i)$. 
    \vspace{1mm}
    
    So if $N$ is a strong pseudoprime to the base $b$, then $b^m \equiv \pm 1 \pmod{N}$, where $m=2^{\min(s_i)-1}t ~(\ddagger)$ (so $\nu_2(m)<\nu_2(p_i-1)$).

    So Lemma \ref{7.5} and CRT imply that the number of solutions to $(\ddagger)$ is $$2 \prod_{i=1}^{k} (m,p_i-1).$$

    This number divides $2\prod_{i=1}^{k} \frac{p_i-1}{2} = \frac{\phi(N)}{2^{k-1}}$. If it is a proper factor or $k\ge 3$, then we're done. Otherwise, $k=2$ and $(m,p_i-1)=\frac{p_i-1}{2}$ for $i=1,2$. Then $p_i -1 \mid 2m = 2^{\min(s_i)}t \stackrel{(\dagger)}{\mid } 2^s t = N-1$, so
    \begin{align*}
        p_1-1 \mid N-1 = p_1p_2-1 = p_2(p_1-1) + (p_2-1) \implies p_1-1 \mid p_2-1 \implies p_1\le p_2.
    \end{align*}
    But the same argument gives $p_2\le p_1$, a contradiction and we're done.
\end{proof}

\marginpar{26 Nov 2022, Lecture 23}
\begin{theorem}[Bach, 1990]
    Let $N$ be an odd composite integer. Then, assuming GRH, $N$ fails the strong test to base $b$ for some $b < 2(\log N)^2$.
\end{theorem}
Hence, conditional on GRH, this is a polynomial time primality test.
\vspace{1mm}

The first unconditional polynomial time primality test was found by Agrawal, Kayal and Saxena in 2002.
\vspace{1mm}

\textbf{Fermat factorization.}
\begin{lemma}\label{7.9}
    Let $N$ be an odd positive integer. Then there a bijection between factorizations of $N=ab$ with $a\ge b>0$ and representations of $N$ as a difference of two squares, $r^2-^2$ for $r\ge s>0$, given by $a=r+s,b=r-s$ and $r=\frac{a+b}{2}, s=\frac{a-b}{2}$.
\end{lemma}
\begin{proof}
    $(\implies):$ If $N$ is odd and $N=ab$ for $a>b$, then $a,b$ are both odd, so $r=\frac{a+b}{2}, s=\frac{a-b}{2}$ are both integers and $r^2-s^2=ab=N$.
    \vspace{1mm}
    
    $(\impliedby)$: If $N=r^2-s ^2$, then $N=(r+s)(r-s)$.
\end{proof}

To factor $N$, take $r = \left\lfloor \sqrt{N} \right\rfloor+1, \left\lfloor \sqrt{N} \right\rfloor+2,\ldots$, and test whether $r^2-N$ is a square. If so, call that $\sigma ^2$, and we get our factorization by Lemma \ref{7.9}.
\begin{example}
    Suppose we want to factorize 200819. Then $\left\lfloor \sqrt{200819} \right\rfloor=448$, so try $r=449$. $449^2-N = 782$, which is not a square. Try $r=450$, so $450^2 - N = 1681 = 41^2$. Hence $N=450^2-41^2=409\cdot 491$.
\end{example}
\textbf{Remark.} Looping over $r$ is more efficient than looping over $s$.
\vspace{1mm}

In general, this method will factor any composite $N$, but this is no better than trial division unless $N$ has factors close to $\sqrt{N}$. 
\vspace{1mm}

Some small improvements:
\begin{enumerate}[(i)]
    \item Speed up our process by using congruences, e.g. $200819=r^2-s ^2 \implies r \equiv 0 \pmod{3}, r \equiv \pm 2 \pmod{5} \implies r \equiv 0,\pm 3 \pmod{15}$.
    \item Try $r=\left\lfloor \sqrt{kN} \right\rfloor+1, \left\lfloor \sqrt{kN} \right\rfloor+2,\ldots$ for some small integer $k$. Is $r^2-kN$ is a square, say $kN=r^2-s ^2$, then $\gcd(r-s,N)$ might be a factor of $N$. 
\end{enumerate}
\textbf{Remark.} If $N=ab$ and $\frac{a}{b} \approx \frac{u}{v}$ for $u,v$ small integers with $u \equiv v\pmod{2}$, then putting $k=uv$ in (ii) works well. Indeed, $kN=uvab=(\frac{av+bu}{2})^2+(\frac{av-bu}{2})^2$, and the second term is fairly small, so we should find it quickly.

\subsection{Factor base methods}

\textbf{Idea.} We want to find $r,s$ with $r^2 \equiv s ^2 \pmod{N}$, but with $r \not\equiv \pm s\pmod{N}$. If we do that, then $\gcd(N,r-s)$ computed using Euclid's algorithm is a nontrivial factor of $N$.
\vspace{1mm}

To check this: if $\gcd(N,r-s)=1$, then as $N \mid r^2-s^2$, we get $N \mid r+s$, so $r \equiv -s \pmod{N}$, contradiction. Similarly, if $\gcd(N,r-s)=N$, then $N \mid r-s$, so $r \equiv s \pmod{N}$.
\vspace{1mm}

How do we find $r$ and $s$? Trying values of $r$ and hoping $r^2-N$ (or $r^2-kN$) is a square is just Fermat factorization. 
\vspace{1mm}

Instead, we look at several values of $r$ such that $r^2 \pmod{N}$ is ''small'', meaning we can factor it. Then look for a product of these small numbers that is a square.
\begin{defn}
    The \textbf{least absolute residue} of $b \pmod{N}$ is the unique integer $\langle b \rangle \in (-\frac{n}{2}, \frac{n}{2})$ with $\langle b \rangle \equiv b \pmod{N}$.
\end{defn}
\begin{defn}
    A \textbf{factor base} $B$ is a finite set of primes together with $-1$.
\end{defn}
\begin{defn}
    We say $b$ is a \textbf{$B$--number} if $\langle b^2 \rangle$ is a product of numbers from $B$ (with repetition allowed). 
\end{defn}
We have the \textbf{factor base method}:
\begin{enumerate}[(1)]
    \item Choose a suitable factor base $B$.
    \item Find some $B$--numbers $b_1,\ldots,b_m$.
    \item Find some subset $I \subset \{1,\ldots,m\}$ such that $\prod_{i \in I}^{} \langle b_i^2 \rangle$ is a square, say $c^2$.
    \item Let $b=\prod_{i \in I}^{} b_i$. Then $b^2 \equiv c^2 \pmod{N}$.
    \item Compute $(N,b-c)$ or $(N,b+c)$ and hope it gives a nontrivial factor of $N$. If not, keep going, i.e. try some more $B$--numbers.
\end{enumerate} 
\begin{example}
    $B=\{-1,2,3\}$ and $N=4633$. $67,68,69$ are $B$--numbers mod $N$, as 
    \begin{align*}
        &67^2 \equiv -144 = -1 \cdot 2^{4} \cdot 3^2\pmod{N}\\
        &68^2 \equiv -9 = -1 \cdot 3^2 \pmod{N}\\
        &69^2 \equiv 128  = 2^7 \pmod{N}.
    \end{align*}
    We put $b=67\cdot 68 = 4556 = -77 \pmod{N}$ and $c = 2^2 3^2 \equiv 36 \pmod{N}$. Then $77^2 \equiv 36^2 \pmod{N}$. In fact, $N=(77-36)(77+36)=41\cdot 113$.
\end{example}
\textbf{Remarks.}
\begin{itemize}
    \item We want to work over $\mathbb{F}_2$. Write $B = \{\gamma_1,\ldots,\gamma_m\}$ for $\gamma_i=-1$ or a prime. Let $S(N,B)$ be the set of $B$--numbers mod $N$, which is not closed under multiplication. Then $b \in S(N,B)$ means $\langle b^2 \rangle = \gamma_1^{\alpha_1}\ldots \gamma_k^{\alpha_{k}}$ for some $\alpha_i \in \mathbb{Z}_{\ge 0}$. Define a map 
    \begin{align*}
        &\lambda : S(B) \to \mathbb{F}_2^k \\
        b \mapsto (\alpha_1~(\text{mod }2), &\alpha_2~(\text{mod }2),\ldots,\alpha_k~(\text{mod }2)).
    \end{align*}
    In step 3, we seek $I \subset \{1,\ldots,m\}$ such that $\prod_{i \in I}^{} \langle b_i^2 \rangle$ is a square. In other words, we want $\sum_{i \in I}^{} \lambda(b_i) = 0$, i.e. we seek a linear dependence relation between the $m$ elements $\lambda(b_1),\ldots,\lambda(b_m)$, which live in the $k$--dimensional vector space $\mathbb{F}_2^k$.

    \marginpar{29 Nov 2022, Lecture 24}

    So if $m \ge k+1$, then we are guaranteed to find a dependence relation. This gives $b^2 \equiv c^2 \pmod{N}$, which factors $N$ if $b \not\equiv \pm c \pmod{N}$.

    \item How do we find $b$--numbers? We want $b$ such that $\langle b^2 \rangle$ is a product of small primes, so we look for $b$ such that $\langle b^2 \rangle$ is small. 
    
    One approach would be to choose integers close to $\sqrt{kN}$ (as in the example above). Another, even better approach is to use continued fractions. Assume $N$ is not a square (else we immediately get a factor, so we're done).
\end{itemize}
\begin{lemma}
    Let $\frac{p_n}{q_n}$ be a convergent to $\sqrt{N}$. Then $|p_n^2 - N q_n^2|\le 2\sqrt{N}$.
\end{lemma}
\begin{proof}
    Using Theorem \ref{6.4}, 
    \begin{align*}
        &|p_n^2-Nq_n^2| = q_n^2 \left|\frac{p_n}{q_n}-\sqrt{N}\right| \left|\frac{p_n}{q_n}+\sqrt{N}\right| < q_n^2 \frac{1}{q_n q_{n+1}} \left(2\sqrt{N}+\frac{1}{q_n q_{n+1}}\right) = \\
        & \frac{1}{q_{n+1}} \left(2q_n\sqrt{N} + \frac{1}{q_{n+1}}\right) < \frac{2}{q_{n+1}} (q_{n}+1)\sqrt{N} \le 2\sqrt{N}
    \end{align*}
    since $q_n < q_{n+1}$.
\end{proof}
\textbf{Remarks.}
\begin{enumerate}[(i)]
    \item Since $2\sqrt{N}<\frac{1}{2}N$, it follows that $\langle p_n^2 \rangle = p_n^2 - N q_n^2$.
    \item We only need to know $p_n \pmod{N}$, so in the recurrence $p_0=a_0, p_1 =a_0a_1+1, p_n = a_np_{n-1}+p_{n-2}$ we work mod $N$ throughout.
\end{enumerate} 
\begin{example}
    Take $N=12403$. Choose $B = \{-1,2,3,5,7,11,13\}$.
    \begin{center}
        \begin{tabular}{l|lll}
        $n$ & $a_n$ & $p_n \text{ mod }N$ & $\langle p_n^2 \rangle$ \\ \hline
        0 & 111 & 111 & -82 \\
        1 & 2 & 223 & 117 \\
        2 & 1 & 334 & -71 \\
        3 & 2 & 891 & 89 \\
        4 & 2 & 2116 & -27 \\
        5 & 7 & 3300 & 166 \\
        6 & 1 & 5416 & -39 \\
        \end{tabular}
    \end{center}
    Note we can factor $117 = 3^2 \cdot 13, -27 = -3^3, -39 =3 \cdot 13$. Hence 
    \begin{align*}
        &(223 \cdot 2116 \cdot 5416)^2 \equiv 3^6 \cdot 13^2 \pmod{N} \\
        & \implies 1062^2 \equiv 351^2 \pmod{N}.
    \end{align*}
    We find $\gcd(N,1062-351)= 79$ and so $N = 79 \cdot 157$.
\end{example}
\textbf{Remark.} Other factor base methods include the Quadratic Sieve and the Number Fields sieve.
\vspace{1mm}

Another class of factoring algorithms are based on the idea that given a prime $p$ we can construct some interesting groups, e.g. $(\mathbb{Z}/p\mathbb{Z})^\times$ or $E(\mathbb{Z}/p\mathbb{Z})$ for $E$ an elliptic curve.
\vspace{1mm}

\textbf{Pollard's $p-1$ algorithm.}
\vspace{1mm}

This method works well if $N$ has a prime number $p$ such that $p-1$ is a product of small primes.
\vspace{1mm}

\begin{enumerate}[(1)]
    \item Choose $k$ that is a product of lots of small primes, for example $k=m!$ or $k = \text{lcm}(1,\ldots,m)$.
    \item Choose at random a small integer $a$ coprime to $N$.
    \item Compute $a^k$ mod $N$ via repeated squaring.
    \item Compute $\gcd(a^k-1,N)$ and hope it is a non-trivial factor of $N$. If not, repeat for other choices of $a$ and $k$.
\end{enumerate}
\textbf{Remark.} If $p$ is a prime factor of $N$ and $p-1 \mid k$, then $a^k \equiv 1 \pmod{p}$, so $p \mid \gcd(a^k-1,N)$, in which case we get a nontrivial factor of $N$ unless it so happens that $a^k \equiv 1\pmod{N}$.

\begin{example}
    Take $N=540143$. Try $k= \text{lcm}(1,\ldots,8)=840$ and $a=2$. We have $2^{840} \equiv 53047 \pmod{N}$. We find $\gcd(540143,53047-1)=421$, so we get $N = 421 \cdot 1283$.
\end{example}
\textbf{Remark.} $(\mathbb{Z}/p\mathbb{Z})^\times$ has order $p-1$, but $E(\mathbb{Z}/p\mathbb{Z})^\times$ has order in the range $[p+1-2\sqrt{p},p+1+2\sqrt{p}]$ (depending on the elliptic curve), so we have a better chance of finding a number with lots of small prime factors.


\end{document}

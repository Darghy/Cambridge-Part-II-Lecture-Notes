\documentclass{article}
%build with recipe latexmk
\usepackage[utf8]{inputenc}
\usepackage[T1]{fontenc}
\usepackage{textcomp}
\usepackage{fancyhdr}
\pagestyle{fancy}
\usepackage{tcolorbox}
\tcbuselibrary{theorems}
\usepackage{babel}
\usepackage{enumerate}
\usepackage{amsmath, amssymb, amsthm}
%\usepackage{a4wide}
\usepackage{float}
\usepackage{bbm}
\usepackage{tikz-cd}
\usepackage{tikz}
\usepackage{graphicx}
\usepackage{wrapfig}
\graphicspath{ {./images/} }
\usepackage{setspace}
\setstretch{1.1}
\usepackage{color}
\usepackage{hyperref}
\hypersetup{
    colorlinks=true, %set true if you want colored links
    linktoc=all,     %set to all if you want both sections and subsections linked
    linkcolor=black,  %choose some color if you want links to stand out
}

\theoremstyle{definition}
\newtheorem{theorem}{Theorem}[section]
\newtheorem{lemma}[theorem]{Lemma}
\newtheorem{cor}[theorem]{Corollary}
\newtheorem{prop}[theorem]{Proposition}
\newtheorem{example}{Example}[section]
\newtheorem{defn}{Definition}[section]

\title{Part II - Number Theory
    \\ \large
    Lectured by Prof. T. A. Fisher
}
\author{Artur Avameri}
\date{Lent 2022}

% figure support
\usepackage{import}
\usepackage{xifthen}
\pdfminorversion=7
\usepackage{pdfpages}
\usepackage{transparent}
\newcommand{\incfig}[1]{%
    \def\svgwidth{\columnwidth}
    \import{./figures/}{#1.pdf_tex}
}

\pdfsuppresswarningpagegroup=1

\setcounter{section}{-1}
\begin{document}
\maketitle
\tableofcontents
\newpage

\section{Introduction}

\marginpar{06 Oct 2022, Lecture 1}

Books: 
\begin{itemize}
    \item A. Baker, \textit{A concise introduction to the theory of numbers}, CUP 1984
    \item N. Koblitz, \textit{A course in number theory \& cryptography}, Springer 1994
    \item H. Davenport, \textit{The higher arithmetic}, CUP 2008
\end{itemize}

Number theory studies the hidden and mysterious properties of the integers and the rational numbers.

It has always been an experimental science. Examining numerical data leads to \textbf{conjectures}, many of which are very old and still unproven today.

\begin{example}
    \begin{enumerate}[(i)]
        \item Let $N\ge 1$ be an integer of the form $8n+5, 8n+6$ or $8n+7$. Does there exist a right-angled triangle of area $N$, all of whose sides have rational length? We don't know.
        \item Let $\pi(x)$ be the number of primes less than or equal to $x$ and define $\text{li}(x) = \int_{2}^{x} \frac{dt}{\log t}$. Then for all $x \ge 3$, $|\pi(x) - \text{li}(x)| \le \sqrt{x}\log x$. This is in fact equivalent to the Riemann hypothesis.
        \item There are infinitely many twin primes. We now know there is an integer $N\le 246$ such that there are infinitely many pairs of primes the form $p, p+N$.
    \end{enumerate}
\end{example}

\newpage

\section{Euclid's algorithm and factoring}

\begin{defn}[Division algorithm]
    Given $a,b \in \mathbb{Z}$, with $b>0$, there exist $q,r \in \mathbb{Z}$ such that $a = qb + r$, and $0 \le r < b$.
\end{defn}

\textbf{Notation.} If $r=0$, then we write $b | a$, else $b\nmid a$. 

\begin{proof}
    Let $S = \{a - nb ~|~ n \in \mathbb{Z}\}$. This certainly contains integers $\ge 0$, so take the smallest one $r$. We claim $r < b$. Indeed, if not, then $r - b \ge 0$, contradicting minimality.
\end{proof}

Given $a_1, \ldots, a_n \in \mathbb{Z}$ not all zero, let $I = \{\lambda_1 a_1 + \ldots + \lambda_n a_n ~|~ \lambda_i \in \mathbb{Z}\}$.

\begin{lemma}
    $I = d \mathbb{Z}$ for some $d > 0$.
\end{lemma}
\begin{proof}
    $I$ certainly contains integers $\ge 0$. Let $d$ be the least positive element of $I$. We claim it works. Take $a \in I$, then $a = qd + r$ with $0 \le r < d$. But $r = a - qd \in I \implies r = 0$.
\end{proof}

\textbf{Remark.} We get from this that $d$ divides each $a_i$, and any common divisor of the $a_i$ must divide $d$. Why?

We write $d = \text{gcd}(a_1, \ldots, a_n)$ for the \textbf{greatest common divisor} (or \textbf{highest common factor}), or just use the shorthand $d = (a_1, \ldots, a_n)$.

\begin{cor}
    Let $a,b,c \in \mathbb{Z}$. Then there exist $x,y \in \mathbb{Z}$ such that $ax+by=c$ if and only if $(a,b) | c$.
\end{cor}

The division algorithm gives a very efficient way to compute $(a,b)$. Assume $a>b>0$. Apply the division algorithm recursively to get
\begin{align*}
    a &= q_1 b + r_1 & 0 \leq r_1 < b\\
    b &= q_2 r_1 + r_2 & 0 \leq r_2 < r_1\\
    r_1 &= q_3 r_2 + r_3 & 0 \leq r_3 < r_2\\
    &~ \vdots \\
    r_{k-2} &= q_k r_{k-1} + r_k & 0 \leq r_k < r_{k-1}, r_k \neq0\\
    r_{k-1} &= q_{k+1} r_k + 0
\end{align*}

\textbf{Claim.} 
    $r_k = (a,b)$.
Indeed, $(a,b) = (b, r_1) = (r_1, r_2) = \ldots = (r_{k-1}, r_k) = r_k$. This is called \textbf{Euclid's algorithm}.

\textbf{Remark.} If $d = (a,b)$, then by Lemma 1.2, there exist $r,s \in Z$ such that $ra + st = d$. Euclid's algorithm gives us a way to find $r$ and $s$.

\vspace{3mm}


In the following table, $x$ and $y$ stand for 34 and 25, and we then compute remainders as linear combinations of them.

We can use a trick here to speed this up: find each row as $q \cdot$ the row before it + the second row before it, then figure out signs at the end. (In fact, the minus signs zigzag down).

\begin{equation*}
    \begin{array}{r|cc}
        & x & y \\
        a=34 & 1 & 0 \\
        b=25 & 0 & 1 \\
        34 = 1 \cdot 25 + \color{red}{9}& 1 & -1 \\
        25 = 2 \cdot 9 + \color{red}{7}& -2 & 3 \\
        9 = 1 \cdot 7 + \color{red}{2}& 3 & -4 \\
        7 = 3 \cdot 2 + \color{red}{1}& -11 & 15 \\
    \end{array}
\end{equation*}

We hence get $-11 \cdot 34 + 15 \cdot 25 = 1$.

\begin{defn}
    An integer $n>1$ is \textbf{prime} if its only positive divisors are $1$ and $n$. Otherwise $n$ is \textbf{composite}.
\end{defn}
\begin{lemma} %1.2 in lectures
    Let $p$ be a prime, and $a,b \in \mathbb{Z}$. If $p | ab$, then $p \mid a$ or $p \mid b$.
\end{lemma}
\begin{proof}
    Assume $p \nmid a$. Then $(a,p) = 1$. By Lemma 1.2, $\exists r,s \in \mathbb{Z}$ such that $ra + sp = 1 \implies rab + spb = b$. Since $p \mid ab$, $p \mid b$ follows.
\end{proof}

\begin{theorem}[\textbf{Fundamental Theorem of Arithmetic}] 
    Every integer ${n>1}$ can be written as a product of primes. This representation is unique up to reordering.
\end{theorem}
\begin{proof}
    Existence is obvious. For uniqueness, suppose $n = p_1p_2 \ldots p_r = q_1 q_2 \ldots q_s$ for $p_i, q_i$ primes. We have $p_1 \mid q_1 q_2 \ldots q_r$, so by Lemma 1.5, $p_1 \mid q_j$ for some $j$, so $p_1 = q_j$. Now cancel these out and induct.
\end{proof}

\textbf{Remark.} If $m = \prod_{i=1}^{k} p_i^{\alpha_i}$ and $n = \prod_{i=1}^{k} p_i^{\beta_i}$ for $p_i$ distinct primes and $\alpha_i, \beta_i \ge 0$, then \[
(m,n) = \prod_{i=1}^{k} p_i^{\text{min}(\alpha_i,\beta_i)}.
\]

However, if $m$ and $n$ are large, it is more efficient to compute $(m,n)$ using Euclid's algorithm.

\marginpar{08 Oct 2022, Lecture 2}

\vspace{1mm}

Suppose we have some large positive integer $N$. An obvious algorithm for factoring $N$ is to trial divide by 2 and the odd integers up to $\sqrt{N}$.

\begin{defn}
    An algorithm with input a positive integer $N$ is \textbf{polynomial} or a \textbf{polynomial time} algorithm if it takes $\le c (\log N)^b$ \textbf{elementary operations}  for some constants $b$ and $c$. 
\end{defn}

\textbf{Remark.} An elementary operation is just adding/multiplying two numbers in $\{0,1,\ldots,9\}$.

\textbf{Remark.} "Polynomial" makes sense here as it takes $\log N$ digits to write $N$.

\vspace{1mm}

Polynomial algorithms are known for: 
\begin{itemize}
    \item Adding and multiplying integers (the usual way);
    \item Computing gcd's (via Euclid's algorithm);
    \item Detecting $n^{\text{th}}$ powers (compute $\sqrt[n]{}$ numberically and round)
    \item More remarkably, primality testing (Agrawal, Kayal, Saxena in 2002)
\end{itemize}

But trial division up to $\sqrt{N}$ is not polynomial.

\vspace{1mm}

\textbf{Fundamental question:} Is there a polynomial time algorithm for factoring? This is unknown.

Later in this course we study the distribution of the prime numbers, in particular the function $\pi(x)$, the number of primes $\le x$.

\begin{theorem}
    There are infinitely many prime numbers, i.e. ${\lim_{x \to \infty} \pi(x) \to \infty}$.
\end{theorem}
\begin{proof}
    Suppose there are only finitely many, say $p_1,\ldots, p_k$. Consider $N = \prod_{i=1}^{k} p_k + 1$. Then $N$ must be divisible by some prime other than the $p_i$, so we're done.
\end{proof}

All the largest known primes are of the form $2^n-1$ for $n$ a prime. These are called \textbf{Mersenne primes}. 51 of them are known, the largest being $2^{82589933}-1$.


\section{Congruences}

Fix a positive integer $n$ (the modulus).

\begin{defn}
    We say $a \equiv b \pmod{n}$, or that $a$ is congruent to $b \pmod{n}$ if $n$ divides $a-b$. 
\end{defn}

This defines an equivalence relation on $\mathbb{Z}$, and we write $\mathbb{Z}/n\mathbb{Z}$ for the set of equivalence classes. We can denote these by $a + n\mathbb{Z}$, or (more commonly) by ${a \pmod{n}}$. We can check that addition and multiplication are well-defined.

\vspace{1mm}

\textbf{Remark.} $n\mathbb{Z}$ is a subgroup/ideal of $\mathbb{Z}$ and $\mathbb{Z}/n\mathbb{Z}$ is the quotient group/ring.

\begin{lemma}
    Let $a \in \mathbb{Z}/n\mathbb{Z}$. Then the following are equivalent:
    \begin{enumerate}[(i)]
        \item $(a,n) = 1$
        \item $\exists b \in \mathbb{Z}$ such that $ab \equiv 1 \pmod{n}$
        \item $a$ is a generator for $\mathbb{Z}/n\mathbb{Z}$.
    \end{enumerate}
\end{lemma}
\begin{proof}
    (i)$\implies $ (ii): $(a,n)=1 \implies \exists r,s \in \mathbb{Z}$ such that $ra + sn = 1$, so $ra \equiv 1 \pmod{n}$.

    (ii) $\implies $ (i): $ab \equiv 1 \pmod{n} \implies ab + kn = 1$ for some $k \in \mathbb{Z} \implies {(a,b)=1}$.

    (ii) $\iff $(iii): $\exists b \in \mathbb{Z}$ s.t. $ab \equiv 1\pmod{n} \iff 1$ belongs to the subgroup of $\mathbb{Z}/n\mathbb{Z}$ generated by $a$.
\end{proof}

\textbf{Notation.} $(\mathbb{Z}/n\mathbb{Z})^{\times}$ is the group of \textbf{units} in $\mathbb{Z}/n\mathbb{Z}$, i.e. the elements with an inverse under multiplication.

\begin{defn}
    $\phi(n) = |(\mathbb{Z}/n\mathbb{Z})^{\times}|$ is called the \textbf{Euler totient function}. We also have $\phi(n) =|\{1 \le a \le n ~|~ (a,n) = 1\}|$. 
\end{defn}

\textbf{Remark.} $\mathbb{Z}/n\mathbb{Z}$ is a field $\iff \phi(n) = n -1 \iff n$ is prime. 

\begin{theorem}[Euler-Fermat theorem]
    If $(a,n)=1$, then $a^{\phi(n)} \equiv 1\pmod{n}$.
\end{theorem}
\begin{proof}
    Apply Lagrange's theorem to the group $G = (\mathbb{Z}/n\mathbb{Z})^{\times}$. Then for $a \in G$, its order divides $|G|= \phi(n)$.
\end{proof}
As a corollary:
\begin{theorem}[Fermat's little theorem]
    If $p \nmid a$, then $a^{p-1} \equiv 1\pmod{p}$.
\end{theorem}

\begin{lemma}%lemma 2.2 in lectures
    Let $G$ be a cyclic group of order $n$. We have \[
    |\{g \in G ~|~ \text{order}(g) = d\}| = \begin{cases}
        \phi(d) &\text{if }d \mid n \\
        0 &\text{otherwise}
    \end{cases}
    \]
    In particular, $\sum_{d \mid n}^{} \phi(d) = n$.
\end{lemma}
\begin{proof}
    WLOG let $G = (\mathbb{Z}/n\mathbb{Z}, +)$. We have $|\{g \in G ~|~ \text{order}(g) = n\}| \stackrel{(*)}{=} \phi(n)$ by Lemma 2.2. If $d \mid n$, say $n = dk$, then the elements of order dividing $d$ are the classes $0, k, 2k, \ldots, (d-1)k \pmod{n}$. These form a cyclic subgroup of order $d$. Applying $(*)$ to this cyclic subgroup shows that there are $\phi(d)$ elements of order $d$.
\end{proof}

\begin{example}
    Consider the simultaneous linear congruences $x \equiv 7 \pmod{10}$ and $x \equiv 3 \pmod{13}$. Suppose we can find $u,v \in \mathbb{Z}$ such that \[
    \begin{cases}
        u \equiv 1 \pmod{10}\\
        u \equiv 0 \pmod{13}
    \end{cases},
    \begin{cases}
        v \equiv 0 \pmod{10}\\
        v \equiv 1 \pmod{13}
    \end{cases}.
    \]
    Then $x = 7u + 3v$ is a solution. But $(10,13) = 1 \implies \exists r,s \in \mathbb{Z}$ such that $10r+13s = 1$, and we can just take $u = 13s, v = 10r$. To find $r,s$, we can use Euclid's algorithm to get $r=4, s= -3$, so $u = -39, v = 40$, and so $x \equiv 7 \cdot (-39) + 3 \cdot 40 \equiv 107 \pmod{130}$.
\end{example}

\marginpar{11 Oct 2022, Lecture 3}

\begin{theorem}[Chinese Remainder Theorem]
    Let $m_1,\ldots, m_k$ be pairwise coprime integers greater than 1. Let $a_1,\ldots, a_k \in \mathbb{Z}$. Let $M = m_1 m_2 \ldots m_k$. Then $\exists  x \in \mathbb{Z}$ satisfying 
    \[
    \begin{cases}
        x \equiv a_1 \pmod{m_1} \\
        \vdots \\
        x \equiv a_k \pmod{m_k}
        \end{cases}.
    \]
        Moreover, the solution is unique mod $M$.
\end{theorem}
\begin{proof}
    Uniqueness: Suppose $x \equiv x' \pmod{m_i} ~\forall i$. Then by considering the prime factorization of $x-x'$ and using the fact that the $m_i$ are pairwise coprime, we get $x \equiv x' \pmod{M}$.
    \vspace{1mm}
    
    Existence: Put $M_i = \frac{M}{m_i}$, so $(M_i, m_i) = 1 ~\forall i$. Hence we can find $u_i \in \mathbb{Z}$ such that $u_i M_i \equiv 1 \pmod{m_i} ~\forall i$. Let $x = \sum_{j=1}^{k} a_j u_j M_j$. Then $x \equiv a_iu_iM_i \equiv a_i \pmod{m_i}$.
\end{proof}
We can write this theorem in one ling using ring theory.
\begin{defn}
    Let $R_i = \mathbb{Z}/m_i\mathbb{Z}$, and define $R_1 \times \ldots \times R_k = \{(r_1,\ldots,r_k) ~|~ r_i \in R_i\}$ with addition and multiplication defined componentwise. This is a ring.
\end{defn}
\begin{theorem}[CRT, ring-theoretic version] %Theorem 2.3 in lectures
    Let $m_1,\ldots, m_k$ be pairwise coprime integers greater than 1 and put $M = m_1\ldots m_k$. Then the map 
    \begin{align*}
        \theta: \mathbb{Z}/M\mathbb{Z} \to \mathbb{Z}/m_1\mathbb{Z} \times \ldots \times \mathbb{Z}/m_k\mathbb{Z} \\
        a + M\mathbb{Z} \mapsto (a + m_1\mathbb{Z}, \ldots, a + m_k\mathbb{Z})
    \end{align*}
    is an isomorphism of rings.
\end{theorem}
\begin{proof}
    $\theta$ is a well defined ring homomorphism since $m_i | M ~\forall i$. Injectivity of $\theta$ follows from uniqueness in CRT, and surjectivity of $\theta$ follows from existence in CRT.
\end{proof}
\begin{cor}
    $\theta$ induces an isomorphism of groups under multiplication 
    \begin{align*}
        (\mathbb{Z}/m\mathbb{Z})^\times \cong (\mathbb{Z}/m_1\mathbb{Z})^\times \times \ldots \times (\mathbb{Z}/m_k\mathbb{Z})^\times \\
        a + M\mathbb{Z} \mapsto (a + m_1\mathbb{Z}, \ldots, a+m_k\mathbb{Z}).
    \end{align*}
\end{cor}
\textbf{Remark.} If $a \in \mathbb{Z}$, then $(a,M) = 1 \iff (a,m_i) = 1 ~\forall i$.

In particular, by looking at orders of the LHS and the RHS above, we get $\phi(M) = \phi(m_1)\ldots \phi(m_k)$, i.e. the Euler phi function is multiplicative.

\begin{defn}
    A function $f : \mathbb{Z}^+ \to \mathbb{C}$ is \textbf{multiplicative} if $f(m) = f(m)f(n)$ whenever $(m,n)=1$.
\end{defn}
\textbf{Examples}: 
\begin{itemize}
    \item $\phi(n) = |(\mathbb{Z}/n\mathbb{Z})^{\times}|$;
    \item $\tau(n) = \sum_{d \mid n}^{} 1$, the number of divisors of $n$;
    \item $\sigma(n) = \sum_{d \mid n}^{} d$, the sum of divisors of $n$;
    \item more generally, $\sigma_k(n) = \sum_{d \mid n}^{} d^k$, so $\sigma_0 = \tau$ and $\sigma_1 = \sigma$.
\end{itemize} 
To prove this:
\begin{lemma} % Lemma 2.4 in lectures
    If $f : \mathbb{Z}^+ \to \mathbb{C}$ is multiplicative, then so is $g : \mathbb{Z}^+ \to \mathbb{C}$, defined by $g(n) = \sum_{d \mid n}^{} f(d)$.
\end{lemma}
\begin{proof}
    Let $m,n$ be coprime. Note that every divisor $d$ of $mn$ can be written as $d = d_1d_2$, where $d_1 \mid m$, $d_2 \mid  n$ and $(d_1,d_2)=1$. Thus \[
    g(mn) = \sum_{d \mid mn}^{} f(d) = \sum_{d_1 \mid m}^{} \sum_{d_2 \mid n}^{} f(d_1d_2) = \sum_{d_1 \mid m}^{} \sum_{d_2 \mid n}^{} f(d_1)f(d_2) = g(m)g(n).  
    \]
\end{proof}
\begin{lemma} % Lemma 2.5 in lectures
    \begin{enumerate}[(i)]
        \item For $p$ a prime, $\phi(p^k) = p^{k-1}(p-1) = p^k(1-\frac{1}{p})$.
        \item $\phi(n) = n \prod_{p \mid n}^{} (1-\frac{1}{p})$.
    \end{enumerate}
\end{lemma}
\begin{proof}
    (i): $\phi(p^k)$ counts the number of integers $a$ between $1$ and $p^k$ such that $(p^k,a) = (p,a) = 1$. So we have $p^a$ numbers, and we don't count the multiples of $p$, so $\phi(p^k) = p^k - p^{k-1}$.
    \vspace{1mm}
    
    (ii): Follows from the fact that $\phi$ is multiplicative. 
\end{proof}

\textbf{Alternative proof} that $\sum_{d \mid n}^{} \phi(d) = n$ (cf Lemma 2.6).
\begin{proof}
    Obviously the RHS is multiplicative. Since $\phi(n)$ is multiplicative, the LHS is multiplicative by Lemma 2.13, so it suffices to check for $n$ a prime power, say $n=p^k$. To this end, compute \[
    \sum_{d \mid p^k}^{} \phi(d) = \phi(1) + \phi(p) + \ldots + \phi(p^k) = 1 + (p-1) + (p^2-p) + \ldots + (p^k-p^{k-1}) = p^k.
    \]
\end{proof}

\subsection{Polynomial congruences}
Let $R = \mathbb{Z},\mathbb{Q}, \mathbb{Z}/n\mathbb{Z}$ (or more generally any commutative ring). Set $R[X] = \{\text{\textbf{polynomials} with coefficients in }R\}$, i.e. $a_n X^n + a_{n-1}X^{n-1} + \ldots + a_1 X + a_0$ for $a_i \in R$.

By definition, two polynomials are equal if and only if they have the same coefficients. We can check that $R[X]$ is a ring (with usual $+$ and $\times$).
\vspace{1mm}

\textbf{Warning}. The map $R[X] \to \{\text{functions }R \to R\}$ by $f \mapsto (\alpha \mapsto f(\alpha))$ is not always injective. For example, if $R=\mathbb{Z}/p\mathbb{Z}$ for $p$ a prime, and $f(X)=X^p - X$, then $f(\alpha)=0 ~\forall \alpha \in R$, but $f$ is not the zero function.
\vspace{1mm}

\textbf{Question.} Can we show thatif $f \in R[X]$ has degree $n$, then $f$ has at most $n$ roots in $R$?
\vspace{1mm}

\textbf{Answer.} No. For example, take $R = \mathbb{Z}/8\mathbb{Z}$, then $f(X)=X^2-1$ has 4 solutions in $\mathbb{Z}/8\mathbb{Z}$.

\marginpar{13 Oct 2022, Lecture 4}
\vspace{1mm}

Let $R=\mathbb{Z},\mathbb{Q}, \mathbb{Z}/n\mathbb{Z}$ (or any commutative ring).

We have a \textbf{division algorithm} on $R[X]$: 

Let $f,g \in R[X]$ and suppose the leading coefficient of $g$ is a unit. Then $\exists q, r \in R[X]$ such that $f(X)=Q(X)g(X)+r(X)$ and $\text{deg}(r)<\text{deg}(g)$.

\begin{proof}
    By induction on $\text{deg}(f)$. If $\text{deg}(f) < \text{deg}(g)$, take $q=0, r=f$. Otherwise, let $f(X) = a X^m + \ldots$ and $g(X) = b X^n + \ldots$ with $m \ge n$ and $b$ a unit.
    \vspace{1mm}
    
    Let $f_1(X) = f(X)- ab^{-1}X^{m-n}g(X)$. Then $\text{deg}(f_1) < \text{deg}(f)$, so by the induction hypothesis, $f_1(x)=q_1(x)g(x) + r_1(x)$ for some $q_1,r_1 \in R[X]$ and $\text{deg}(r_1) < \text{deg}(g)$. Now take $q(X) = ab^{-1}X^{m-n} + q_1(X)$ and $r=r_1$, so we're done.
\end{proof}

\begin{cor} % Corollary 2.6 in lectures
    If $f \in R[X]$ and $\alpha \in R$ is such that $f(\alpha)=0$, then $f(X) = (X-\alpha)f_1(X)$ for some $f_1 \in R[X]$. 
\end{cor}
\begin{proof}
    By the division algorithm, $f(X) = (X-\alpha) f_1(X) + r$ for some $r \in R$ (as $\text{deg}(r) < \text{deg}(X-\alpha)$). Plug in $X=\alpha$ to get $r=0$.
\end{proof}
\begin{defn}
    $R$ is an \textbf{integral domain} if $R$ has no zero divisors, i.e. $\alpha,\beta \in R$, $\alpha \beta=0 \implies \alpha=0$ or $\beta=0$.
\end{defn}
\textbf{Note.} Let $n>1$. Then $\mathbb{Z}/n\mathbb{Z}$ is an integral domain $\iff$ $n$ is prime.

\begin{theorem} % Theorem 2.7 in lectures
    If $R$ is an integral domain, then any polynomial $f \in R[X]$ of degree $n$ has at most $n$ roots.
\end{theorem}
\begin{proof}
    By induction on $n$, the degree of $f$. If $n=0$, then our polynomial is a nonzero constant and we're done. Now suppose $\exists \alpha \in R$ such that $f(\alpha)=0$ (otherwise we're done). By Corollary 2.15, $f(X) = (X-\alpha)f_1(X)$. Since $R$ is an integral domain, every root of $f$, except possibly $\alpha$ is a root of $f_1$. By induction, $f_1$ has at most $n-1$ roots, hence $f$ has at most $n$ roots and we're done.
\end{proof}
\begin{cor}[Lagrange's Theorem]
    Let $p$ be a prime and $a_0,\ldots,a_n \in \mathbb{Z}$ with $p \nmid a_n$. Then the congruence 
    \begin{align*}
        a_nx^n + a_{n-1}x^{n-1} + \ldots + a_1x + a_0 \equiv 0 \pmod{p}
    \end{align*}
    has at most $n$ solutions mod $p$.
\end{cor}
\begin{proof}
    Take $R=\mathbb{Z}/p\mathbb{Z}$ in Theorem 2.17.
\end{proof}
\textbf{Remark.} In this course, we will refer to the above theorem as Lagrange's Theorem.
\begin{example}
    Let $p$ be a prime. We will factor $X^{p-1}- 1 \pmod{p}$. Let $f(X) = X^{p-1}-1 - \prod_{a=1}^{p-1}(X-\alpha)$ in $\mathbb{Z}/p\mathbb{Z}[X]$. By Fermat's Little Theorem, $f$ has at least $p-1$ roots mod $p$. But $\text{deg}(f) < p-1$, since the $X^{p-1}$ terms cancel out, so by Lagrange's Theorem, $f = 0$, i.e. $X^{p-1} -1 = \prod_{a=1}^{p-1} (X-a)$ in $\mathbb{Z}/p\mathbb{Z}[X]$. Plugging in $X=0$ gives $(p-1)! \equiv -1 \pmod{p}$, i.e. Wilson's Theorem.
\end{example}
\begin{example}
    Working mod $7$, the powers of $3$ (starting from 0) are $1,3,2,6,4,5$. So $(\mathbb{Z}/7\mathbb{Z})^{\times}$ is cyclic, generated by $3$.
\end{example}
\begin{theorem}%Theorem 2.8 in lectures
    Let $p$ be a prime. Then $(\mathbb{Z}/p\mathbb{Z})^{\times}$ is cyclic.
\end{theorem}
\begin{proof}
    Let $S_d = \{a \in (\mathbb{Z}/p\mathbb{Z})^{\times} ~|~ \text{ord}(a) = d\}$. Suppose $S_d \neq \emptyset$, say $a \in S_d$. Then $1,a,a^2,\ldots,a^{d-1}$ are distinct elements in $\mathbb{Z}/p\mathbb{Z}$ and they are solutions of $x^d \equiv 1 \pmod{p}$. By Lagrange's theorem, this has at most $d$ solutions, and we found $d$ solutions, so those are all of them, i.e. $S_d \subseteq \{1,a,a^2,\ldots,a^{d-1}\}$. Note that the LHS is a cyclic group of order $d$, so this has $\phi(d)$ elements of order $d$.
    \vspace{1mm}
    
    We conclude that for every $d$, $|S_d| = 0$ or $|S_d| = \phi(d)$. In particular, ${|S_d| \le \phi(d)}$. Hence
    \begin{align*}
       p-1 \stackrel{(\star)}{=} \sum_{d \mid (p-1)}^{} |S_d| \le  \sum_{d \mid (p-1)}^{} \phi(d) = p-1,
    \end{align*}
    where $(\star)$ follows since we just count all the elements in $(\mathbb{Z}/p\mathbb{Z})^{\times}$. Hence ${|S_d| = \phi(d) ~\forall  d \mid (p-1)}$. In particular, $S_{p-1} \neq \emptyset$, i.e. $(\mathbb{Z}/p\mathbb{Z})^{\times}$ contains elements of order $p-1$, i.e. $(\mathbb{Z}/p\mathbb{Z})^{\times}$ is cyclic. 
\end{proof}
\textbf{Remark.} The same argument shows that any finite subgroup of the multiplicative group of a field is cyclic.
\begin{defn}
    An integer $a$ such that $a \pmod{n}$ generates $(\mathbb{Z}/n\mathbb{Z})^{\times}$ is called a \textbf{primitive root} mod $n$. 
\end{defn}
Theorem 2.21 showed that primitive roots exist mod $p$.
\begin{example}
    Let $p = 19$. Let $d$ be the order of $2$ in $(\mathbb{Z}/19\mathbb{Z})^{\times}$. We know $d \mid 18$, so we work out 
    \begin{align*}
        2^3 &\equiv 8 \pmod{19}\\
        2^6 &\equiv 7 \not\equiv 1 \pmod{19} \implies d \nmid 6\\
        2^9 &\equiv -1 \not\equiv 1 \pmod{19} \implies d \nmid 9,
    \end{align*}
    so $d=18$ and hence 2 is a primitive root mod $19$.
\end{example}
In general, $g \in \mathbb{Z}$ (coprime to $p$) is a primitive root mod $p$ if and only if $g^{\frac{p-1}{q}} \not\equiv 1 \pmod{p}$ $~\forall \text{primes }q \mid (p-1)$.

\marginpar{15 Oct 2022, Lecture 5}

\textbf{Remark.} The number of primitive roots mod $p$ is $\phi(p-1)=\phi(\phi(p))$.
\vspace{1mm}

Here are some (open) problems concerning primitive roots:
\begin{enumerate}[(i)]
    \item Artin's conjecture (1927) -- Let $a>1$ be an integer which is not a square. Then $a$ is a primitive root mod $p$ for infinitely many primes $p$. This is unknown for $a=2$. Hooley (1967) proved this assuming GRH. Heath-Brown (1986) proved that Artin's conjecture holds for at least one of $2,3$ or $5$. In fact, he proved something stronger: he proved the conjecture fails for at most 2 prime values of $a$.   
    \item How large is the smallest primitive root mod $p$? Burgess (1962) showed it is $\le c p^{1/4 + \epsilon} ~\forall  \epsilon>0$ and some constant $c=c(\epsilon)$. Shoup (1992) showed it is $\le c(\log p)^6$ assuming GRH.
\end{enumerate}

We now consider $\mathbb{Z}/p^n\mathbb{Z}$ for $n>1$. For $n \ge 3$, there is a surjective group homomorphism from $(\mathbb{Z}/2^n\mathbb{Z})^{\times} \to (\mathbb{Z}/8\mathbb{Z})^{\times} = \{\pm 1, \pm3\} \cong C_2 \times C_2$, so $(\mathbb{Z}/2^{n}\mathbb{Z})^{\times}$ is not cyclic (since generators map to generators).

\begin{theorem}% Theorem 2.9 in lectures
    Let $p$ be an odd prime. Then $(\mathbb{Z}/p^n\mathbb{Z})^{\times}$ is cyclic $~\forall n\ge 1$.
\end{theorem}
We divide the proof into 3 lemmas.
\begin{lemma}%Lemma 2.10 in lectures
    Let $n \ge 2$. Then $g$ is a primitive root mod $p^n$ if and only if the following two conditions hold: 
    \[
    \begin{cases}
        g \text{ is a primitive root mod }p \\
        g^{p^{n-2}(p-1)} \not\equiv 1 \pmod{p^n}
    \end{cases}.
    \]
\end{lemma}
\begin{proof}
    ($\implies $) is clear, as $\phi(p^n)=p^{n-1}(p-1)$.

    ($\impliedby$): Let $d$ be the order of $g$ in $(\mathbb{Z}/p^n\mathbb{Z})^{\times}$. Then $d \mid \phi(p^n) = p^{n-1}(p-1)$. Since $g^d \equiv 1 \pmod{p^n}$, we have $g^d \equiv 1 \pmod{p}$. Hence by assumption 1, we have $(p-1)\mid d$. Say $d = p^j (p-1)$ for some $0\le j \le n-1$. If $j\le n-2$, then this contradicts assumption 2. Hence $j=n-1$, so $d=\phi(p^n)$ is a primtive root mod $p^n$.
\end{proof}
Next we show $\exists g \in \mathbb{Z}$ satisfying conditions 1 and 2 in the case $n=2$.
\begin{lemma}
    $\exists g \in \mathbb{Z}$ a primitive root mod $p$ such that $g^{p-1} \not\equiv 1 \pmod{p^2}$.
\end{lemma}
\begin{proof}
    Let $g$ be a primtive root mod $p$. If $g^{p-1} \equiv 1\pmod{p^2}$, then consider $g+p$, which is still a primtive root mod $p$, but
    \[
    (g+p)^{p-1}=g^{p-1} + (p-1)g^{p-2}p + \ldots \equiv 1 + (p-1)g^{p-2}p \pmod{p^2},
    \]
    where the second term is not divisible by $p^2$, so $(g+p)^{p-1} \not\equiv 1 \pmod{p^2}$.
\end{proof}
Next we show that if $g$ is a primitive root mod $p^2$, then it is a primitive root mod $p^n ~\forall n\ge 2$.
\begin{lemma}
    If $g^{p-1} \not\equiv 1 \pmod{p^2}$, then $g^{p^{n-2}(p-1)} \not\equiv 1\pmod{p^n} ~\forall n\ge 2$. 
\end{lemma}
\begin{proof}
    By induction on $n$, the case $n=2$ being given. Suppose the result is true for $n$. By Euler-Fermat, $g^{p^{n-2}(p-1)} \equiv 1 \pmod{p^{n-1}}$, so $g^{p^{n-2}(p-1)} = 1 +b p^{n-1}$ for some $b \in \mathbb{Z}$, where $p \nmid b$ by the induction hypothesis. Taking $p^{\text{th}}$ powers gives 
    \begin{align*}
        g^{p^{n-1}(p-1)} = (1+bp^{n-1})^p = 1 + bp^n + {{p} \choose {2}}b^2p^{2(n-1)}+\ldots \equiv \\ 1 + bp^n + {{p} \choose {2}}b^2 p^{2(n-1)} \stackrel{\star}{\equiv } 1 + bp^n \pmod{p^{n+1}},
    \end{align*}
    where $\star$ follows since $p$ is odd, so $p \mid {{p}\choose{2}}$ (and also we use $3(n-1)\ge n+1$ and $2(n-1)+1\ge n+1$). Thus $g^{p^{n-1}(n-1)} \equiv 1 +bp^n \not\equiv 1 \pmod{p^{n+1}}$, so the result follows for $n+1$.
\end{proof}
This completes the proof of Theorem 2.24.

\begin{example}
    We saw $3$ is a primitive root mod $7$. We calculate $3^3 = -1 + 4\cdot 7$, so $3^6 \equiv 1 - 8\cdot 7 \not\equiv 1 \pmod{7^2}$. Hence $3$ is a primitive root mod $7^n ~\forall n$.
\end{example}

For the case $p=2$, let $G = \{a \in (\mathbb{Z}/2^n\mathbb{Z})^{\times} ~|~ a \equiv 1 \pmod{4}\}$. Then $(\mathbb{Z}/2^n\mathbb{Z})^{\times} \cong \{\pm 1\} \times G$ by $a+2^n\mathbb{Z} \mapsto \begin{cases}
    (1, a+2^n\mathbb{Z}) &\text{ if }a \equiv 1\pmod{4}\\
    (-1, -a+2^n\mathbb{Z}) &\text{ if }a \equiv 3\pmod{4}.
\end{cases}$

\textbf{Exercise.} Show that $G$ is cyclic (and generated by 5).

\textbf{Exercise.} For which $n$ is $(\mathbb{Z}/n\mathbb{Z})^{\times}$ cyclic? 

\marginpar{18 Oct 2022, Lecture 6}

\section{Quadratic residues}

Let $p$ be an odd prime and $a \in \mathbb{Z}$. By Lagrange's theorem, the congruence $x^2 \equiv a \pmod{p}$ has at most 2 solutions. If $a \not\equiv 0\pmod{p}$, then there are either 0 or 2 solutions. Indeed, if $x$ is a solution, then so is $-x \not\equiv x \pmod{p}$.

\begin{defn}
    Suppose $a \not\equiv 0\pmod{p}$. We say $a$ is a \textbf{quadratic residue} (QR) if $x^2 \equiv a\pmod{p}$ is soluble. We say $a$ is a \textbf{quadratic nonresidue} (NQR) if $x^2\equiv a \pmod{p}$ is unsoluble.
\end{defn}
\begin{example}
    $p=7$. $1,2,4$ are QRs and $3,5,6$ are QNRs.
\end{example}
\begin{lemma}
    Let $p$ be an odd prime. Then there are $\frac{p-1}{2}$ quadratic residues mod $p$ (and hence also $\frac{p-1}{2}$ quadratic nonresidues).
\end{lemma}
\begin{proof}[Proof 1.]
    Let $\mathbb{F}_p = \mathbb{Z}/p\mathbb{Z}$ (a field with $p$ elements). We show that the map $\mathbb{F}_p^\times \to \mathbb{F}_p^\times$ by $x \mapsto x^2$ is exactly 2--to--1. 
    
    Indeed, if $x^2 \equiv y^2 \pmod{p}$, then $p \mid x^2-y^2$, so $p \mid (x-y)$ or $p \mid (x+y)$, so $x \equiv \pm y \pmod{p}$.
\end{proof}
\begin{proof}[Proof 2.]
    Let $g$ be a primitive root mod $p$. Then $\mathbb{F}_p^\times = \{1,g,g^2,\ldots,g^{p-2}\}$.

    We claim that $g^i$ is a QR $\iff$ $i$ is even.
    \vspace{1mm}
    
    $\impliedby$ is clear. For $\implies$, suppose $g^i \equiv x^2\pmod{p}$. Then we can write $x = g^j \pmod{p}$, so $g^i \equiv g^{2j} \pmod{p} \implies i \equiv 2j \pmod{p-1}$. But $p-1$ is even, so $i = 2j + k(p-1)$ is even.
\end{proof}

\begin{defn}[Legendre symbol]
    Let $p$ be an odd prime, $a \in \mathbb{Z}$. Then
    \[
    \left(\frac{a}{p}\right) = \begin{cases}
        0 &\text{ if } p \mid a\\
        1 &\text{ if } a \text{ is a QR mod } p\\
        -1&\text{ if } a \text{ is a QNR mod } p
    \end{cases}
    \]
\end{defn}
\begin{theorem}[Euler's Criterion]
    Let $p$ be an odd prime and $a \in \mathbb{Z}$. Then 
    \[
        \left(\frac{a}{p}\right) \equiv a^{\frac{p-1}{2}} \pmod{p}.
    \]
\end{theorem}
\begin{proof}
    This is obvious if $p \mid a$, so suppose $(a,p)=1$. By Fermat's little theorem, $a^{p-1}\equiv 1\pmod{p} \implies a^{\frac{p-1}{2}} \equiv \pm 1 \pmod{p}$.

    If $\left(\frac{a}{p}\right) =1$, then $a \equiv b^2 \pmod{p}$ for some $b \in \mathbb{Z}$, but then $a^{\frac{p-1}{2}} \equiv b^{p-1} \equiv 1 \pmod{p}$. This gives $\frac{p-1}{2}$ solutions to the congruence $x^{\frac{p-1}{2}} \equiv 1 \pmod{p}$. By Lagrange's theorem, these are all the solutions. Hence if $\left(\frac{a}{p}\right) = -1$, then $a^{\frac{p-1}{2}}\not\equiv 1\pmod{p}$, so $a^{\frac{p-1}{2}} \equiv -1 \pmod{p}$ and we're done.
\end{proof}
\begin{cor} % Corolalry 3.2 in lectures
    $\left(\frac{ab}{p}\right) = \left(\frac{a}{p}\right)\left(\frac{b}{p}\right)$.
\end{cor}
\begin{proof}
    \[
    \left(\frac{ab}{p}\right) \equiv (ab)^{\frac{p-1}{2}} \equiv a^{\frac{p-1}{2}}b^{\frac{p-1}{2}} \equiv \left(\frac{a}{p}\right)\left(\frac{b}{p}\right) \pmod{p}.
    \]
    Since $0, \pm 1$ are distinct mod $p$, we have equality in the above.
\end{proof}
The corollary is equivalent to the statements:
\begin{itemize}
    \item $\mathcal{X}: \mathbb{F}_p^{\times} \to \{\pm 1\}$ by $a \mapsto \left(\frac{a}{p}\right)$ is a group homomorphism.
    \item \begin{enumerate}[(i)]
        \item QR $\cdot$ QR $=$ QR
        \item QR $\cdot$ QNR $=$ QNR
        \item QNR $\cdot$ QNR $=$ QR
    \end{enumerate}    
\end{itemize}
We can give an alternative proof for this:
\begin{enumerate}[(i)]
    \item $a \equiv x^2\pmod{p}, b \equiv y^2 \pmod{p} \implies ab \equiv (xy)^2 \pmod{p}$.
    \item If $a \equiv x^2$ and $ab \equiv z^2 \pmod{p}$, then $b \equiv (x^{-1}z)^2 \pmod{p}$, a contradiction.
    \item Suppose $a$ is a QNR. The map $\mathbb{F}_p^\times \to \mathbb{F}_p^\times$ by $x \mapsto ax$ is a bijection sending QRs to NQRs by (ii). By Lemma 3.1, it sends QNRs to QRs, done.
\end{enumerate}
\textbf{Remark.} We can also prove Euler's criterion using primitive roots.

\begin{cor}
    Let $p$ be a odd prime. Then 
    \[
        \left(\frac{-1}{p}\right) = (-1)^{\frac{p-1}{2}} = \begin{cases}
            1 &\text{ if } p \equiv 1\pmod{4}.\\
            -1 &\text{ if } p \equiv -1 \pmod{4}.
        \end{cases}
    \]
\end{cor}
In the next lecture, we show 
\[
\left(\frac{2}{p}\right) = (-1)^{\frac{p^2-1}{8}} = \begin{cases}
    1 &\text{ if } p \equiv \pm 1\pmod{8}.\\
    -1 &\text{ if } p \equiv \pm 3 \pmod{8}.
\end{cases}
\]
Let $p,q$ be distinct odd primes. The law of quadratic reciprocity gives a relation between $\left(\frac{p}{q}\right)$ and $\left(\frac{q}{p}\right)$. Generalizing this result (in many different ways) has been one of the main goals of number theory ever since.

\begin{theorem}[Law of quadratic reciprocity]
    Let $p,q$ be distinct odd primes. Then 
    \[
    \left(\frac{q}{p}\right) = \begin{cases}
        \left(\frac{p}{q}\right) &\text{ if } p\equiv 1\pmod{4} \text{ or }q \equiv 1 \pmod{4}. \\
        -\left(\frac{p}{q}\right) &\text{ if } p \equiv q \equiv 3 \pmod{4}.
    \end{cases}
    \]
\end{theorem}
\begin{example}
    $$\left(\frac{19}{73}\right) = \left(\frac{73}{19}\right) = \left(\frac{16}{19}\right) = 1.$$
\end{example}

\marginpar{20 Oct 2022, Lecture 7}

\textbf{Another proof of Fermat's little theorem:}

If $(a,p)=1$, then working mod $p$, the set $\{a,2a,3a,\ldots,(p-1)a\}$ is the same as $\{1,2,\ldots,(p-1)\}$. Taking the product gives $a^{p-1}(p-1)! \equiv (p-1)! \pmod{p} \implies a^{p-1}\equiv 1\pmod{p}$ as desired.
\vspace{1mm}

We can use the same idea to compute $a^{\frac{p-1}{2}}$ mod $p$:

\begin{lemma}[Gauss' Lemma]
    Let $p$ be an odd prime, let $a \in\mathbb{Z}$ be coprime to $p$, and put $m=\frac{p-1}{2}$. For $j=1,2,\ldots,m$ let $a_j$ be the unique integer such that
    \begin{enumerate}[(i)]
        \item $a_j \equiv ja \pmod{p}$
        \item $-m\le a_j \le m$.
    \end{enumerate}
    Then $\left(\frac{a}{p}\right) = (-1)^\nu$, where $\nu = \{\# 1\le j \le m ~|~ a_j < 0\}$.
\end{lemma}
\begin{proof}
    Consider $a_1,\ldots,a_m \in \{\pm1,\pm2,\ldots,\pm m\}$. Can any two of these be the same? No, since $a_i \equiv a_j \implies ai\equiv aj \implies i \equiv j \pmod{p}$. 

    Can any two differ by a sign? No, since $a_i \equiv -a_j \implies ia \equiv -ja \implies i \equiv -j \pmod{p}$.

    Hence $a_1,\ldots,a_m$ are $\pm1,\pm2, \ldots, \pm m$ in some order with some choice of signs. Taking the product gives $$a_1\ldots a_m \equiv (-1)^\nu 1\cdot \ldots \cdot m \pmod{p} \implies a^m m! \equiv (-1)^\nu m! \pmod{p}.$$
    So by Euler's criterion, $\left(\frac{a}{p}\right) \equiv  a^m \equiv (-1)^\nu \pmod{p}$.
\end{proof}

\begin{cor}
    Let $p$ be an odd prime. Then $$\left(\frac{2}{p}\right) = (-1)^{\frac{p^2-1}{8}}=  \begin{cases}
        1 &\text{ if } p \equiv  \pm 1 \pmod{8}.\\
        -1 &\text{ if } p \equiv  \pm 3 \pmod{8}.
    \end{cases}$$
\end{cor}
\begin{proof}
    Let $m =\frac{p-1}{2}$. Then $a_j = \begin{cases}
        2j &\text{ for } 1\le j\le \frac{m}{2}. \\
        2j-p &\text{ for } \frac{m}{2}< j \le m.
    \end{cases}$
    Hence $$\nu = m - \left\lfloor \frac{m}{2} \right\rfloor = \begin{cases}
        \frac{m}{2} \text{ if } m \text{ is even.}\\
        \frac{m+1}{2}\text{ if } m \text{ is odd.}
    \end{cases}$$
    It follows that $\left(\frac{2}{p}\right) = 1 \iff \nu$ is even $\iff m \equiv 0,3$ mod $4 \iff p \equiv \pm 1 \pmod{8}$.
\end{proof}
\vspace{1mm}

\begin{theorem}[Law of quadratic reciprocity]
    Let $p,q$ be distinct odd primes. Then \[
    \left(\frac{p}{q}\right)\left(\frac{q}{p}\right) \equiv (-1)^{\frac{p-1}{2}\frac{q-1}{2}}.
    \]
\end{theorem} 
\begin{proof}
    Step 1: Let $a,p,\nu$ be as in Gauss' Lemma (with $a\ge 1$).
    \vspace{1mm}
    
    Claim: $$\nu = \sum_{i=1}^{2n} (-1)^{i} \left\lfloor \frac{ip}{2a} \right\rfloor$$ where $n = \left\lfloor \frac{a}{2} \right\rfloor$. Moreover, $\frac{ip}{2a} \not\in \mathbb{Z} ~\forall~1\le i\le 2n$.
    \vspace{1mm}
    
    Proof: Consider all multiples of $a$ less than $\frac{ap}{2}$ ($=np$ or $(n+\frac{1}{2})p)$. Hence $\nu$ is the number of multiples of $a$ in the intervals $$\left[\frac{1}{2}p,p\right], \left[\frac{3}{2}p, 2p\right],\ldots, \left[(n-\frac{1}{2})p,np\right].$$ On dividing through by $a$, we see that $\nu$ is the number of integers in $$\left[\frac{p}{2a},\frac{2p}{2a}\right],\left[\frac{3p}{2a},\frac{4p}{2a}\right],\ldots,\left[\frac{(2n-1)p}{2a},\frac{2np}{2a}\right].$$ The end points are not in $\mathbb{Z}$, since the end points of the original intervals are not multiples of $a$. Hence $\# ([\alpha,\beta] \cap \mathbb{Z}) = \left\lfloor \beta \right\rfloor - \left\lfloor \alpha \right\rfloor$. This proves the claim.
    \vspace{1mm}
    
    Step 2: Let $p_1,p_2$ be primes and $a \in \mathbb{Z}$ coprime to $p_1p_2$. By Gauss' lemma, $\left(\frac{a}{p_i}\right) = (-1)^{\nu_i}$.
    \begin{enumerate}[(i)]
        \item Suppose $p_1 \equiv p_2 \pmod{4a}$. Then $\left\lfloor \frac{ip_1}{2a} \right\rfloor \equiv \left\lfloor \frac{ip_2}{2a} \right\rfloor \pmod{2}$. By Step 1, we have $\nu_1 \equiv \nu_2 \pmod{2}$. Hence $\left(\frac{a}{p_1}\right) = \left(\frac{a}{p_2}\right)$.
        \item Suppose $p_1 \equiv -p_2 \pmod{4a}$. Then $\left\lfloor \frac{ip_1}{2a} \right\rfloor \equiv \left\lfloor \frac{ip_2}{2a} \right\rfloor +1 \pmod{2}$. (We use the fact that if $\alpha \in \mathbb{R}/\mathbb{Z}$, then $\left\lfloor -\alpha \right\rfloor = -\left\lfloor \alpha \right\rfloor-1$). By Step 1, we again deduce $\left(\frac{a}{p_1}\right) = \left(\frac{a}{p_2}\right)$.
    \end{enumerate}
    \vspace{1mm}
    
    Step 3: Conclusion of the proof. 
    \begin{enumerate}[(i)]
        \item Suppose $p \equiv q \pmod{4}$, say $p = 4a +q$. Then $\left(\frac{p}{q} \right) = \left(\frac{4a+q}{q} \right) = \left(\frac{a}{q} \right)$, and $\left(\frac{q}{p} \right) = \left(\frac{p-4a}{p} \right) = \left(\frac{-1}{p} \right) \left(\frac{a}{p} \right) $. But $p \equiv q \pmod{4a}\stackrel{\text{Step 2(i)}}{\implies} \left(\frac{a}{p} \right) =\left(\frac{a}{q} \right)$, hence we conclude 
        \[
            \left(\frac{p}{q}\right)\left(\frac{q}{p}\right) \equiv (-1)^{\frac{p-1}{2}\frac{q-1}{2}}.
        \]
        \item Suppose $p \neq q \pmod{4}$, say $p+q = 4a$. Then $\left(\frac{p}{q} \right) = \left(\frac{4a-q}{q} \right) = \left(\frac{a}{q} \right)$ and $\left(\frac{q}{p}  \right) = \left(\frac{4a-p}{p} \right) = \left(\frac{a}{p} \right)$. But $p \equiv -q \pmod{4a} \stackrel{\text{Step 2(ii)}}{\implies} \left(\frac{a}{p} \right) = \left(\frac{a}{q} \right)$, so $\left(\frac{p}{q} \right) =\left(\frac{q}{p} \right)$, done.
    \end{enumerate}
\end{proof}

\marginpar{22 Oct 2022, Lecture 8}
\begin{example}
    Compute the Legendre symbol $\left(\frac{7411}{9283} \right)$. In fact, 7411 and 9283 are both prime. Hence
    \begin{align*}
        \left(\frac{7411}{9283}\right) = -\left(\frac{9283}{7411} \right) = -\left(\frac{1872}{7411} \right).  
    \end{align*}
    As $1872 = 2^4 \cdot 3^2 \cdot 13$, we get
    \begin{align*}
        -\left(\frac{1872}{8411} \right) = -\left(\frac{13}{7411} \right) = - \left(\frac{7411}{13} \right) = -\left(\frac{1}{13} \right) = -1.   
    \end{align*}
    Hence $7411$ is not a QR mod $9283$.
\end{example}
Recall that the Legendre symbol $\left(\frac{a}{p} \right) $ is only defined for $p$ an odd prime.
\begin{defn}
    Let $n$ be an odd positive integer, say $n=p_1\ldots p_k$ for $p_i$ (not necessarily distinct) odd primes. Let $a \in \mathbb{Z}$. We define the \textbf{Jacobi symbol} as \[
    \left(\frac{a}{n} \right) = \prod_{i=1}^{k} \left(\frac{a}{p_i} \right).
    \]
\end{defn}
\textbf{Remark.} If $(a,n)\neq 1$, then $\left(\frac{a}{n} \right)=0$.
\begin{prop}
    \begin{enumerate}[(i)]
        \item $\left(\frac{a}{n} \right) $ depends only on $a$ mod $n$.
        \item $\left(\frac{ab}{n} \right) =\left(\frac{a}{n} \right)\left(\frac{b}{n} \right)$ and $\left(\frac{a}{mn} \right) = \left(\frac{a}{m} \right) \left(\frac{a}{n} \right)$.
        \item $\left(\frac{-1}{n} \right) = (-1)^{\frac{n-1}{2}}$.
        \item $\left(\frac{2}{n} \right) =(-1)^{\frac{n^2-1}{8}}$.
    \end{enumerate}
\end{prop}
\begin{proof}
    \begin{enumerate}[(i)]
        \item Clear, since the Legendre symbol only depends on $a$ mod $p$.
        \item The first part follows since the Legendre symbol is totally multiplicative, and the second follows from the definition of the Jacobi symbol.
        \item This holds for $n=p$ a prime by previous results. We will now show that if they hold for odd integers $m,n$, then they hold for $mn$. But \[
        \left(\frac{-1}{mn} \right) = \left(\frac{-1}{m} \right) \left(\frac{-1}{n} \right) = (-1)^{\frac{m-1}{2}}(-1)^{\frac{n-1}{2}} \stackrel{\star}{=} (-1)^{\frac{mn-1}{2}},
        \]
        where we can check that $\star$ holds, since $(m-1)(n-1) \equiv  0 \pmod{4}$, which gives $mn-1 \equiv (m-1) + (n-1) \pmod{4}$.
        \item This is analogous to above, except we get \[
        (-1)^{\frac{m^2-1}{8}} (-1)^{\frac{n^2-1}{8}} = (-1)^{\frac{(mn)^2-1}{8}},
        \]
        since $(m^2-1)(n^2-1)\equiv 0 \pmod{16}$, so $(mn)^2-1 \equiv (m^2-1)+(n^2-1) \pmod{16}$.
    \end{enumerate}
\end{proof}
\begin{theorem}[Law of Quadratic Reciprocity for Jacobi Symbols]
    If $m,n$ are odd positive integers, then \[
    \left( \frac{m}{n} \right) = (-1)^{\frac{m-1}{2}\frac{n-1}{2}} \left(\frac{n}{m} \right).
    \]
\end{theorem}
\textbf{Remark.} If $(m,n) \neq 1$, this says $0=0$.
\begin{proof}
    Again, we deduce this from the corresponding result for the Legendre symbol. Assume $(m,n)=1$. Write $m = \prod_{i=1}^{k} p_i$ and $n=\prod_{j=1}^{l} q_j$ for $p_i,q_j$ (not necessarily distinct) primes.
    \vspace{1mm}
    
    Let $r$ count the number of $p_i$ with $p_i \equiv 3 \pmod{4}$ and $s$ count the number of $q_j$ with $q_j \equiv 3 \pmod{4}$. Then
    \begin{align*}
        \left(\frac{m}{n} \right) = \prod_{i=1}^{k} \prod_{j=1}^{l} \left(\frac{p_i}{q_j} \right) = \prod_{i=1}^{k} \prod_{j=1}^{l} (-1)^{\frac{p_i-1}{2}\frac{q_j-1}{2}} \left(\frac{q_j}{p_i} \right) = \\
        (-1)^{rs} \prod_{i=1}^{k} \prod_{j=1}^{l} \left(\frac{q_j}{p_i} \right) = (-1)^{rs} \left(\frac{n}{m}\right).  
    \end{align*}
    But $m \equiv 1 \pmod{4} \iff r$ is even, and $n \equiv 1 \pmod{4} \iff s$ is even, hence $(-1)^{rs} = (-1)^{\frac{m-1}{2}\frac{n-1}{2}}$. 
\end{proof}
\textbf{Remark.} The Jacobi symbol $\left(\frac{a}{n} \right) $ tells us surprisingly little about whether the congruence $x^2 \equiv a \pmod{n}$ is soluble. 

If $x^2 \equiv a\pmod{n}$ is soluble, then so is $x^2\equiv  a \pmod{p}$ for all primes $p \mid n$. So $\left(\frac{a}{p} \right) = 1 ~\forall p \mid n$, hence $\left(\frac{a}{n} \right) = 1$.

But the converse is false. For example, $\left(\frac{2}{15} \right) = \left(\frac{2}{3} \right) \left(\frac{2}{5} \right) = (-1)\cdot (-1) = 1$, yet $x^2 \equiv 2 \pmod{15}$ is not soluble.

The point of the Jacobi symbol is rather that it allows us to compute Legendre symbols without having to factor (except for removing powers of 2).
\begin{example}
    \[
    \left(\frac{33}{73} \right) = \left(\frac{73}{33} \right) = \left(\frac{7}{33} \right) = \left(\frac{33}{7} \right) = \left(\frac{5}{7} \right) = -1, 
    \]    
    so $33$ is not a QR mod $73$.
\end{example}

Three tricks to evaluate Legendre symbols:
\begin{example}
    \begin{enumerate}[(i)]
        \item $\sum_{a=1}^{p-1} \left(\frac{a}{p} \right) = 0$
        \item $\sum_{a=1}^{p-1} a \left(\frac{a}{p} \right) \equiv 0\pmod{p}$ if $p > 3$.
        \item $\sum_{a=1}^{p-1} \left(\frac{a(a+1)}{p} \right) \equiv -1$.
    \end{enumerate}
\end{example}
\begin{proof}
    \begin{enumerate}[(i)]
        \item We have already done this since we have an equal number of QRs and QNRs. However, alternate proof:
        \vspace{1mm}
        
        Let $b$ be a QNR $\pmod{p}$. Then 
        \[
        \sum_{a=1}^{p-1} \left(\frac{a}{p} \right) = \sum_{a=1}^{p-1} \left(\frac{ab}{p} \right) = \left(\frac{b}{p} \right) \sum_{a=1}^{p-1} \left(\frac{a}{p} \right) = - \sum_{a=1}^{p-1} \left(\frac{a}{p} \right), 
        \]
        so $\sum_{a=1}^{p-1} \left(\frac{a}{p} \right) = 0$.
        \item Since $p>3$, we can choose $b \not\equiv 0, \pm 1 \pmod{p}$, whence \[
        \sum_{a=1}^{p-1} a \left(\frac{a}{p} \right) \equiv  \sum_{a=1}^{p-1} ab \left(\frac{ab}{p} \right) \equiv  \pm b \sum_{a=1}^{p-1} a \left(\frac{a}{p} \right) \pmod{p}.
        \]
        Since $b \not\equiv \pm 1 \pmod{p}$, we deduce $\sum_{a=1}^{p-1} a \left(\frac{a}{p} \right) \equiv 0 \pmod{p}$.
        \item  If $ab \equiv 1\pmod{p}$, then \[
        \left(\frac{a(a+1)}{p} \right) \equiv \left(\frac{a^2(1+b)}{p} \right) = \left(\frac{b+1}{p} \right).  
        \]
        Then \[
        \sum_{a=1}^{p-1} \left(\frac{a(a+1)}{p} \right) = \sum_{b=1}^{p-1} \left(\frac{b+1}{p} \right) = -1.
        \]
    \end{enumerate}
\end{proof}

\section{Binary quadratic forms}

\marginpar{25 Oct 2022, Lecture 9}

\textbf{Question.} Which numbers can be written as the sum of two squares?

Fermat gave an answer around 1630, and Euler published the first proof in 1749.

\begin{theorem}
    Let $N$ be a positive integer. Then $N$ is the sum of two squares if and only if every prime $p \equiv 3 \pmod{4}$ that divides $N$ divides it to an even power.
\end{theorem}
\begin{proof}[Proof of the easy direction.]
    $\implies$: Suppose $N=x^2+y^2$ and $p \mid N$, then $x^2 + y^2 \equiv 0 \pmod{p}$. If $p \equiv 3 \pmod{4}$, then $\left(\frac{-1}{p} \right) = -1$, so we must have $x\equiv y \equiv 0 \pmod{p}$. Then divide $N$ by $p^2$ and repeat until $p \nmid N$.
    \vspace{1mm}
    
    $\impliedby$: Since $(x^2+y^2)(z^2+t^2)=(xz-yt)^2+(xt+yz)^2$, it suffices to prove the result the case $N=p$ with $p=2$ or $p \equiv 1\pmod{4}$. $p=2$ is easy, but $p \equiv 1 \pmod{4}$ is a little more involved, and we will prove it a later lecture.
\end{proof}

Euler also studied $x^2+2y^2, x^2+3y^2$, etc. In this section we study \textbf{binary quadratic forms} with integer coefficients, i.e. $f(x,y)=ax^2+bxy+cy^2$ for $a,b,c \in \mathbb{Z}$.
\begin{defn}
    We say $f$ \textbf{represents} n if $f(x,y)=n$ for some $x,y \in \mathbb{Z}$.
\end{defn}
We may write $f$ as $(a,b,c)$ or in matrix notation as $$f(x,y) = \begin{pmatrix} x & y \end{pmatrix} \begin{pmatrix} a & \frac{b}{2} \\ \frac{b}{2} & c \end{pmatrix} \begin{pmatrix} x \\ y \end{pmatrix}.$$
\begin{example}
    $f(x,y)=x^2+y^2$ may be written as $(1,0,1)$ or $\begin{pmatrix} 1 & 0\\0&1 \end{pmatrix}$.

    $g(x,y) = 4x^2 + 12xy + 10y^2$ may be written as $(4,12,10)$ or $\begin{pmatrix} 4 & 6 \\6 & 10 \end{pmatrix}$.
\end{example}
Note that $g(x,y) = (2x+3y)^2+y^2 = f(2x+3y,y)$. Do $f$ and $g$ represent the same numbers? No, as $g$ only represents even numbers.
\vspace{1mm}

Let $X=2x+3y, Y=y$, then $$\begin{pmatrix} X\\Y \end{pmatrix} = \begin{pmatrix} 2 & 3\\0 & 1 \end{pmatrix}\begin{pmatrix} x \\y \end{pmatrix} \implies \begin{pmatrix} x\\y \end{pmatrix} = \frac{1}{2}\begin{pmatrix} 1 & -3 \\0 & 2 \end{pmatrix}\begin{pmatrix} X \\ Y \end{pmatrix}.$$
Note that we can have $X,Y \in \mathbb{Z}$, yet $x,y \not\in \mathbb{Z}$. 

\begin{defn}
    A \textbf{unimodular substitution} is one of the form $X = \alpha x + \gamma y, Y = \beta X + \delta Y$ where $\alpha,\beta,\gamma,\delta \in \mathbb{Z}$ and $\alpha \delta - \beta \gamma = 1$. 
\end{defn}
\begin{defn}
    Two BQFs $f$ and $g$ are \textbf{equivalent}, written $f\sim g$, if they are related by a unimodular substitution.
\end{defn}
Exercise: Check $\sim$ is an equivalence relation (this is on the example sheet).

\textbf{Note.} Equivalent forms represent the same integers.
\vspace{1mm}

The group $SL_2(\mathbb{Z}) = \left\{\begin{pmatrix} \alpha &\beta \\ \gamma& \delta \end{pmatrix} ~|~ \alpha,\beta,\gamma,\delta \in \mathbb{Z}, \alpha \delta - \beta \gamma = 1\right\}$ acts on the set of BQFs via $\begin{pmatrix} \alpha &\beta \\ \gamma&\delta \end{pmatrix} : f(x,y) \mapsto f(\alpha x + \gamma y, \beta x + \delta y)$. The equivalence classes are the orbits of this action.

To check a group action, we need to check 
\begin{enumerate}[(i)]
    \item $\begin{pmatrix} 1 & 0\\0&1 \end{pmatrix} f = f$, which is true.
    \item $\sigma(\tau f) = (\sigma \tau)f ~\forall \sigma, \tau \in SL_2(\mathbb{Z})$.
\end{enumerate}
Suppose $f=(a,b,c)$ and $g=(a',b',c')$ are equivalent, say $g = \sigma f$ for $\sigma = \begin{pmatrix}  \alpha & \beta\\ \gamma &\delta \end{pmatrix}$. Then 
\begin{align*}
    g(x,y)=f(\alpha x + \gamma y, \beta x + \delta y) = \begin{pmatrix} \alpha x + \gamma y & \beta x + \delta y \end{pmatrix} \begin{pmatrix} a & \frac{b}{2} \\ \frac{b}{2} & c \end{pmatrix} \begin{pmatrix} \alpha x  + \gamma y \\ \beta x + \delta y \end{pmatrix} = \\
    \begin{pmatrix} x & y \end{pmatrix}\begin{pmatrix} \alpha & \beta \\ \gamma & \delta \end{pmatrix}\begin{pmatrix}  a & \frac{b}{2} \\ \frac{b}{2} & c \end{pmatrix}\begin{pmatrix} \alpha & \gamma \\ \beta & \delta \end{pmatrix}\begin{pmatrix} x \\ y \end{pmatrix}.
\end{align*}
Hence $\begin{pmatrix} a' & \frac{b'}{2} \\ \frac{b'}{2} & c' \end{pmatrix} = \sigma \begin{pmatrix} a & \frac{b}{2} \\ \frac{b}{2} & c \end{pmatrix} \sigma^\top$. Call this $(\star)$.

To check (ii), we note that \[
\sigma\left(\tau \begin{pmatrix} a & \frac{b}{2} \\ \frac{b}{2} & c \end{pmatrix} \tau^\top\right)\sigma^\top = (\sigma \tau)\begin{pmatrix} a & \frac{b}{2} \\ \frac{b}{2} & c \end{pmatrix} (\sigma \tau)^\top.
\]
\begin{defn}
    The \textbf{discriminant} of $f(x,y) = ax^2+bxy+cy^2$ is $$\text{disc}(f) = b^2-4ac.$$
\end{defn}
\begin{example}
    $\text{disc}(1,0,1) = -4, \text{disc}(4,12,10) = -16$.
\end{example}
\begin{lemma}
    Equivalent BQFs have the same discriminant.
\end{lemma}
\begin{proof}
    Taking determinants in $(\star)$ gives
    \[
        a'c' - \left(\frac{b'}{2}\right)^2 = \left(\text{det }\sigma\right)^2\left(ac - \left(\frac{b}{2}\right)^2\right).
    \]
    But $\det \sigma = 1$, so multiplying both sides by $-4$ gives $(b')^2-4a'c' = b^2-4ac$ as desired.
\end{proof}
\textbf{Remark.} The converse is not true, i.e. there exist BQFs with the same discriminant which are not equivalent.

For example, $(1,0,6)$ and $(2,0,3)$ both have discriminant $-24$, but $(1,0,6)$ represents 1 (with $x=1,y=0$), but $(2,0,3)$ does not.
\begin{lemma}
    There exists a BQF $f$ with $\text{disc}(f)=d \iff d \equiv 0,1 \pmod{4}$. 
\end{lemma}
\begin{proof}
    $\implies$: $d = b^2-4ac \equiv b^2 \equiv 0,1 \pmod{4}$.

    $\impliedby$: If $d \equiv 0 \pmod{4}$, let $f = (1,0,-\frac{d}{4})$. If $d \equiv 1 \pmod{4}$, take $f = (1,1,\frac{1-d}{4})$.
\end{proof}

\marginpar{27 Oct 2022, Lecture 10}

\begin{defn}
    A quadratic form $f(x_1,\ldots,x_n)=\sum_{i\le j}^{} a_{ij}x_ix_j$ with $a_{ij} \in \mathbb{R}$ is:
    \begin{itemize}
        \item \textbf{positive definite} if $f(x)>0 ~\forall 0 \neq x \in \mathbb{R}^n$.
        \item \textbf{negative definite} if $f(x)<0 ~\forall 0 \neq x \in \mathbb{R}^n$.
        \item \textbf{indefinite} if $f(x)>0$ and $f(x')<0$ for some $x,x' \in \mathbb{R}^n$. 
    \end{itemize}
\end{defn}
We are interested in the case $n=2$ and $a_{ij} \in \mathbb{Z}$.

\begin{lemma}
    Let $f(x,y)=ax^2+bxy+cy^2$ be a BQF which has discriminant $d=b^2-4ac$. 
    \begin{enumerate}[(i)]
        \item If $d<0$ and $a>0$, then $f$ is positive definite.
        \item If $d<0$ and $a<0$, then $f$ is negative definite.
        \item If $d>0$, then $f$ is indefinite.
        \item If $d=0$, then $f= \lambda(mx+ny)^2$ for $\lambda,m,n \in \mathbb{Z}$.
    \end{enumerate}
\end{lemma}
\begin{proof}
    \begin{align*}
        4af(x,y)=4a^2x^2+4abxy+&4acy^2 = \\ 
        (2ax+by)^2 + &(4ac-b^2)y^2 = (2ax+by)^2 - dy^2.
    \end{align*}

    (i) and (ii): If $d<0$ and $a\neq 0$, then it follows that $4af(x,y)\ge 0$ with equality if and only if $x=y=0$. The cases $a>0$ and $a<0$ now show $f$ is either positive or negative definite as desired.
    \vspace{1mm}
    
    (iii): Suppose $d>0$. If $a\neq 0$, then the above equation shows us that $4af(1,0)>0$ and $4af(-b,2a)<0$, so $f$ is indefinite. 
    
    If $a=0$, then replace $f(x,y) \mapsto f(y,x)$. This works unless $a=c=0$, but then $b\neq 0$, so $f(x,y)=bxy$, which is obviously indefinite.
    \vspace{1mm}
    
    (iv): Omitted (not interesting nor difficult).
\end{proof}

\textbf{Remark.} It is possible for a BQF $(a,b,c)$ with $a,b,c>0$ to be indefinite, e.g. $(1,3,1)$.

It is also possible for $(a,b,c)$ with $b<0$ to be positive definite, e.g. $(1,-1,2)$.
\vspace{1mm}

From now on, we will concentrate on positive definite BQFs, i.e. forms $(a,b,c)$ with $d=b^2-4ac<0$ and $a>0$ (and hence $c>0$).

We have an equivalence relation $\sim$ on positive definite BQFs, and we want to study the equivalence classes. It will help if we can specify a "simplest" form for each equivalence class.

\begin{example}
    Consider $(10,34,29)$. The middle coefficient is large -- can we decrease it? If $f(x) = ax^2+bxy+cy^2$, then one substitution we may try is 
    \begin{align*}
        f(x+\lambda y,y)=a(x+\lambda y)^2+ b(x + \lambda y)y + &cy^2 = \\ 
        ax^2 &+ (b+2\lambda a)xy + (\lambda^2 a + \lambda b + c)y^2.
    \end{align*}
    Taking $\lambda = \pm 1$ shows $$(a,b,c) \sim (a,b \pm 2a, a \pm b + c). ~~~~(\dagger)$$
    In our example, we get $(10,34,29) \sim (10,14,5) \sim (10, -6, 1)$.

    Making the substitution $X=y, Y=-x$ gives $$(a,b,c) \sim (c,-b, a). ~~~~ (\ddagger)$$ In our example we now get $$(10,-6,1) \sim (1, 6, 10) \sim (1,4, 5) \sim (1,2,2) \sim (1,0,1).$$
\end{example}
\textbf{Remark.} It is a good idea to check that the discriminant doesn't change (to catch mistakes).

\textbf{Remark.} We can ensure $|b|\le a$ via $(\dagger)$, and $a\le c$ via $(\ddagger)$.
\begin{defn}
    A positive definite BQF is \textbf{reduced} if either \[
    -a < b \le a < c ~\text{, or } ~ 0\le b \le a = c.
    \]
\end{defn}
(Think of this as $|b|\le a \le c$ with some extra conditions).
\begin{lemma}\label{4.5}
    Every positive definite BQF is equivalent to a reduced form.
\end{lemma}
\begin{proof}
    We have operations $$S : (a,b,c) \mapsto (c, -b, a), ~T_{\pm}:(a,b,c) \mapsto (a, b \pm 2a, a \pm b +c).$$
    If $a>c$, then use $S$ to decrease $a$ while leaving $|b|$ unchanged. If $a\le c$ and $|b|>a$, then use $T_{\pm}$ to decrease $|b|$ while leaving $a$ unchanged. 
    
    Repeat these steps. Each step decreases $a + |b|$, so this procedure must eventually reach a form with $|b|\le a\le c$. Finally, to get the form we want in the lemma:

    \begin{itemize}
        \item If $b=-a$, then apply $T_+$ to replace $(a,-a,c) \mapsto (a, a, c)$.
        \item If $a=c$ and $b<0$, then apply $S$ to get $b>0$.
    \end{itemize}
\end{proof}

\marginpar{29 Oct 2022, Lecture 11}

\begin{lemma}\label{4.6}
    Let $f=(a,b,c)$ be a reduced positive definite BQF with discriminant $d$. Then $|b|\le a \le \sqrt{\frac{|d|}{3}}$ and $b \equiv d \pmod{2}$.
\end{lemma}
\begin{proof}
    Being reduced implies $|b|\le a\le c$, and $d = b^2-4ac \le ac -4ac = -3ac \le -3a^2 \implies a^2 \le \frac{|d|}{3}$. Also $d = b^2-4ac \implies b \equiv d \pmod{2}$.
\end{proof}
\begin{example}
    Consider $d = -4$. We must have $a=1$ by the lemma above (as $a>0$), and $b=0$ (by parity), so solve for $c$ to get $c=1$, i.e. $x^2+y^2$ is the only positive definite reduced BQF with discriminant $-4$. 
\end{example}

We can now return to the beginning of this section and answer our original question: which numbers can be written as the sum of two squares?
\begin{proof}[Proof of Theorem 4.1 (continued)]
    Let $p$ be a prime, $p \equiv 1\pmod{4}$. We have $\left(\frac{-1}{p} \right) = 1$, so $\exists u \in \mathbb{Z}$ such that $u^2 \equiv -1 \pmod{p} \implies u^2 = -1 + kp$ for some $k \in \mathbb{Z}$. Let $f = (p,2u,k)$, so $\text{disc}(f) = 4u^2 - 4pk = -4$.
    \vspace{1mm}
    
    By Lemma \ref{4.5}, $f \sim g$ for some reduced form $g$, but by our above example, $g(x,y)= x^2+y^2$. Now $f$ represents $p$ (take $x=1,y=0$), so $g$ also represents $p$, i.e. $p$ is the sum of two squares as required.
\end{proof}
\textbf{Question.} Can reduced forms be equivalent?

\begin{defn}
    Let $f$ be a BQF and $n \in \mathbb{Z}$. We say $f$ \textbf{represents} $n$ if $n=f(x,y)$ for some $x,y \in \mathbb{Z}$. We say $f$ \textbf{properly represents} $n$ if $n=f(x,y)$ for some coprime $x,y \in \mathbb{Z}$.
\end{defn}
\textbf{Remark.} Equivalent forms properly represent the same integers, since if $X=\alpha x  + \gamma y, Y = \beta x + \delta y$ with $\alpha,\beta,\gamma,\delta \in \mathbb{Z}$, then $\alpha \delta - \beta \gamma = 1$ implies $\gcd(X,Y)=1 \iff \gcd(x,y)=1.$
\begin{lemma}\label{4.7}
    The smallest integers properly represented by a reduced positive definite BQF $f=(a,b,c)$ are $a,c, a - |b| + c$ in that order.\footnote{Values on this list are repeated if they are represented in more than one way, not counting repeats of the form $f(x,y)=f(-x,-y)$.}
\end{lemma}
\begin{proof}
    $f$ reduced $\implies |b| \le a \le c \implies a \le c \le a - |b| + c$. We have $f(1,0)=a$, $f(0,1)=c$. If $x=0$, then $\gcd(x,y)=1 \implies y=\pm 1$. Likewise, if $y=0$, then $x = \pm 1$.

    So it remains to show that the smallest number represented by $f$ using nonzero $x,y$ is $a-|b|+c$. But if $|x|\ge |y|\ge 1$, then 
    \begin{align*}
        f(x,y)=ax^2+bxy+cy^2\ge ax^2 - |b||x||y| + cy^2 \ge (a-|b|)x^2 + cy^2 \ge a - |b| + c.
    \end{align*}
    We can achieve equality with $f(1,\pm 1)$. We proceed similarly if $|y|\ge |x|\ge 1$.
\end{proof}
\begin{theorem}\label{4.8}
    Every positive definite BQF is equivalent to a unique reduced form.
\end{theorem}
\begin{proof}
    Existence follows from Lemma \ref{4.5}.

    Uniqueness: Suppose $f=(a,b,c)$ and $g=(a',b',c')$ are equivalent reduced BQFs. We want to show $a=a',b=b',c=c'$. By Lemma \ref{4.7}, $a=a',c=c'$ and $a-|b|+c = a' -|b'|+ c'$, so $(a,b,c)=(a',\pm b',c')$.
    \vspace{1mm}
    
    If $b=0$, we're done. If $b\neq 0$, can $(a,b,c)$ and $(a,-b,c)$ both be reduced? If yes, then $a<c$ (since $a=c$ requires $b\ge 0$ by definition) and $|b| <a$ (since we can't have $b=-a$). Hence $a < c < a - |b| + c$. By Lemma \ref{4.7} again, $f(x,y)=a \iff (x,y) = (\pm 1,0)$ and $f(x,y) = c \iff (x,y) = (0,\pm 1)$, and likewise for $g$.
    \vspace{1mm}
    
    Suppose $g(x,y) = f(\alpha x + \gamma y, \beta x + \delta y) = f(X,Y)$. Then 
    \begin{align*}
        (X,Y) = (\pm 1, 0) \iff (x,y) = (\pm 1, 0) \\
        (X,Y) = (0, \pm 1) \iff (x,y) = (0, \pm 1),
    \end{align*}
    i.e. $\begin{pmatrix} \alpha & \beta \\ \gamma & \delta \end{pmatrix} = \begin{pmatrix} \pm 1 & 0 \\ 0 & \pm 1 \end{pmatrix}$. But $\alpha \delta - \beta \gamma = 1$, so $\begin{pmatrix} \alpha & \beta \\ \gamma & \delta \end{pmatrix} = \pm \begin{pmatrix} 1 & 0 \\ 0 & 1 \end{pmatrix}$, so $f=g$ as required.
\end{proof}

\marginpar{01 Nov 2022, Lecture 12}

\textbf{Question.} How many reduced forms are there with a given discriminant?

\begin{example}
    Consider $d = -24$. We want to find $f = (a,b,c)$ reduced with $b^2-4ac=-24$. By Lemma \ref{4.6}, $|b|\le a \le \sqrt{8}$ and $b$ is even.
    \begin{itemize}
        \item If $a=1$, then $b=0$ and hence $c=6 \implies (1,0,6)$. We can check that this is reduced.
        \item If $a=2$, then $c=\frac{b^2+24}{8}$.
        \begin{itemize}
            \item If $b=0$, then $c=3$. This is reduced.
            \item If $b=\pm 2$, then $c \not\in \mathbb{Z}$.
        \end{itemize}
    \end{itemize}
    So the only reduced forms with discriminant $-24$ are $(1,0,6)$ and $(2,0,3)$.
\end{example}
More generally, Lemma \ref{4.6} shows that for every $d$, there are only finitely many reduced forms with discriminant $d$.
\begin{defn}
    The \textbf{class number} of $d$, denoted $h(d)$ is the number of equivalence classes of positive definite BQFs with discriminant $d$.
\end{defn}
By Theorem \ref{4.8}, this is the number of reduced forms with discriminant $d$, hence finite by the last remark.
\begin{example}
    As we have already seen, $h(-4)=1, h(-24)=2$.
\end{example}
\begin{defn}
    $d \equiv 0,1 \pmod{4}$ is a \textbf{fundamental discriminant} if it is not of the form $d=k^2d_1$ for some integer $k\ge 1$ and $d_1 \equiv 0,1 \pmod{4}$.
\end{defn}

Aside:

\textbf{Remark.} Let $d<0$ be a fundamental discriminant. Gauss defined a group law on the set of equivalence classes of positive definite BQFs with discriminant $d$. The abelian group obtained in this way is the same as the class group of the field $\mathbb{Q}(\sqrt{d})$ (see Part II Number Fields). We insisted that $\alpha \delta - \beta \gamma = 1$ in the definition of equivalence (not just $=\pm 1$), since otherwise inverse elements in the class group would be the same element, hence it is no longer a group. End of aside.

\textbf{Some theorems about class numbers.}

\begin{enumerate}[(i)]
    \item (Mertens 1874). \[
    \sum_{-X<d<0}^{} h(d) \sim \frac{\pi}{18} X^{\frac{3}{2}} \text{ as } X \to \infty.
    \]
    \item (Heilbronn 1934) $h(d) \to \infty$ as $|d| \to \infty$.
    \item (Siegel 1935) For every $\epsilon>0, \exists c>0$ such that $h(d)>c |d|^{\frac{1}{2}-\epsilon}$.
    \item (Baker-Stark 1967) $h(d)=1 \iff d \in \{-3,-4,-7,-8,-11,-19,-43,-67,-163\}$.
\end{enumerate}
End of aside.

\begin{lemma}\label{4.9}
    Let $f$ be a BQF and $n \in \mathbb{Z}$. Then $f$ properly represents $n$ if and only if $f$ is equivalent to a form with first coefficient $n$.
\end{lemma}
\begin{proof}
    $\impliedby$: Suppose $f \sim g(n,b,c)$. Then $g(1,0)=n \implies g$ properly represents $n$, so $f$ properly represents $n$.
    \vspace{1mm}
    
    $\implies$: $f(\alpha,\beta)=n$ for some $\alpha,\beta \in \mathbb{Z}$ coprime. By Euclid's algorithm, $\exists \gamma,\delta \in \mathbb{Z}$ such that $\alpha \delta - \beta \gamma = 1$. Then $f$ is equivalent to $g(x,y) = f(\alpha x + \gamma y, \beta x + \delta y)$ with first coefficient $g(1,0) = f(\alpha,\beta)=n$.
\end{proof}
\begin{theorem}
    Let $n$ be a positive integer and $d<0$ a discriminant. Then $n$ is properly represented by some positive definite BQF with discriminant $d$ if and only if the congurence $$x^2 \equiv d \pmod{4n}$$ is soluble.
\end{theorem}
\begin{proof}
    $\implies$: Lemma \ref{4.9} shows $f \sim g$ with $g = (n,b,c)$. Then $$d = \text{disc}(f) = \text{disc}(g) = b^2 - 4nc \equiv b^2 \pmod{4n}.$$
    \vspace{1mm}
    
    $\impliedby$: We are given $b,c \in \mathbb{Z}$ such that $b^2 = d + 4nc$. Then $f = (n,b,c)$ is a form of discriminant $d$ and it properly represents $n$ (with $x=1,y=0$).
\end{proof}
\begin{example}
    Which integers are properly represented by $f(x,y)={x^2+xy+2y^2}$?
    
    We have $\text{disc}(f)=-7$, so $f$ is positive definite. By Lemma \ref{4.6}, any reduced form with discriminant $-7$ satisfies $|b|\le a \le 1$ and $b$ is odd. Hence $(a,b,c)=(1,1,2)$ or $(a,b,c) = (-1,-1,2)$. But the second one is not reduced, hence $h(-7)=1$ and all positive definite BQFs with discriminant $-7$ are equivalent.
    \vspace{1mm}
    
    Hence $n$ is properly represented by $x^2+xy+2y^2$ if and only if $x^2 \equiv -7 \pmod{4n}$ is soluble.

    \marginpar{03 Nov 2022, Lecture 13}

    Assume $n=p$ is prime and $p \neq 2,7$. By CRT, the above is equal to 
    \[
    \begin{cases}
        x^2 \equiv -7 \pmod{4}. \text{ This is soluble.} \\
        x^2 \equiv  -7 \pmod{p}. \text{This is soluble }\iff \left(\frac{-7}{p} \right) = 1. 
    \end{cases}
    \]
    But $\left(\frac{-7}{p} \right)  = \left(\frac{-1}{p}\right)\left(\frac{7}{p}\right) = (-1)^{\frac{p-1}{2}}(-1)^{\frac{p-1}{2}}\left(\frac{p}{7}\right)=\left(\frac{p}{7}\right)$.

    We conclude that $p = x^2+xy+2y^2$ for some $x,y \in \mathbb{Z}$ means that $p \equiv 1,2,4 \pmod{7}$ or $p=7$ (we check $p=2,7$ separately).
\end{example}
\begin{lemma}
    Let $p$ be an odd prime and $a \in \mathbb{Z}$. If $\left(\frac{a}{p} \right) =1$, then the congruence $x^2 \equiv a \pmod{p^n}$ is soluble $~\forall n\ge 1$.
\end{lemma}
\begin{proof}
    Induction on $n$. The case $n=1$ is clear.
    \vspace{1mm}
    
    Now let $n\ge 1$ and suppose $x^2 \equiv a \pmod{p^n}$, i.e. $x^2 = a +k p^n, k \in \mathbb{Z}$. For $t \in \mathbb{Z}$, we have $(x + tp^n)^2 \equiv x^2 + 2xt p^n \equiv a + (2xt+k)p^n \pmod{p^{n+1}}$. Now we have $(2x,p)=1$, so we can solve $2xt+k \equiv 0\pmod{p}$, so we're done.
\end{proof}
\textbf{Remark.} A similar argument shows that $a \in \mathbb{Z}$ with $a \equiv 1\pmod{8}$, then $x^2 \equiv a \pmod{2^n}$ is soluble $~\forall n\ge 1$.

\textbf{Above example continued:} Write $n = 2^{\alpha}7^{\beta}p_1^{\gamma_1}\ldots p_{r}\gamma^r$ for $p_i$ distinct powers. Then
\[
x^2 \equiv -7 \pmod{4n} \text{ is soluble} \iff \begin{cases}
    &x^2 \equiv -7 \pmod{2^{\alpha+2}} \text{ is soluble.} \\
    &x^2 \equiv -7 \pmod{7^{\beta+1}} \text{ is soluble.} \\
    &x^2 \equiv -7 \pmod{p_i^{\gamma_i}} \text{ is soluble}~\forall 1\le i\le r. \\
\end{cases}
\]
The first condition is always true by the remark above. The second one has no solutions mod $49$, so hence $\beta\le 1$. For the last condition, use the above lemma to get that we need $\left(\frac{-7}{p_i} \right) = 1 ~\forall 1 \le i \le r$.
\vspace{1mm}

Hence we want $7^2 \nmid n$ and all primes $p \mid n$ with $p \neq 7$ satisfy $p \equiv 1,2,4 \pmod{7}$.
\vspace{1mm}

The integers represented by $x^2+xy+y^2$ (not necessarily properly) are then of the form $k^2n$ for $k \in \mathbb{Z}$ and $n$ as described above.
\vspace{1mm}

\textbf{Conclusion.} $n = x^2+xy+2y^2$ for some $x,y \in \mathbb{Z} \iff$ every prime $p \equiv 3,5,6$ which divides $n$ divides it to an even power.
\vspace{1mm}

\textbf{Remarks.}
\begin{enumerate}[(i)]
    \item If $h(d)=1$, we have shown how to solve the problem of which integers are represented by a given form of discriminant $d<0$.
    
    If $h(d)>1$, we can determine which integers are represented by \textit{some} form of discriminant $d$. For some values of $d$ we can still distinguish which forms represent which numbers using congruence conditions.
    \item What about quadratic forms in more variables?
    \begin{theorem}[Lagrange 1770]
        Every positive integer is a sum of four squares.
    \end{theorem}
    \begin{theorem}[Legendre 1797]
        A positive integer $n$ is a sum of 3 squares if and only if $n \neq 4^a(8b+7)$ for some integers $a,b \ge 0$.
    \end{theorem}
    \item A geometric way to think about reduction: Let $f(x,y) = ax^2+bxy+cy^2$ be a positive definite BQF, so $d=b^2-4ac<0$. Let $\tau \in \mathbb{C}$ with $f(\tau,1)=0$ and $\text{Im}(\tau)>0$, so $\tau = \frac{-b \pm \sqrt{|d|} i}{2a}$, and $|\tau|^2 = \frac{b^2-d}{4a^2} = \frac{c}{a}$.
    
    So $|b|\le a\le c \iff |\text{Re}(\tau)|\le \frac{1}{2}$ and $|\tau| \ge 1$. Let $\mathcal{F}$ be this subregion of $\mathbb{C}$. Then $SL_2(\mathbb{Z}) \subset  SL_2(\mathbb{R})$ acts on $\mathcal{H} = \{\tau \in \mathbb{C} \mid \text{Im}(\tau)>0\}$ via $\begin{pmatrix} a & b\\c &d \end{pmatrix} : \tau \to \frac{a\tau + b}{c\tau + d}$, and the operations $S$ and $T_{\pm}$ in the proof of Lemma \ref{4.5} correspond to the Möbius maps $S: \tau \mapsto \frac{-1}{\tau}$ and $T_{\pm } : \tau \mapsto \tau \pm 1$. So we just start somewhere in the complex plane and apply these transformations until we end up in $\mathcal{F}$.

    \item Extra conditions in the definition of a reduced form correspond to conditions concerning the boundary of $\mathcal{F}$.
\end{enumerate}


\end{document}

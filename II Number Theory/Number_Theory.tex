\documentclass{article}
%build with recipe latexmk
\usepackage[utf8]{inputenc}
\usepackage[T1]{fontenc}
\usepackage{textcomp}
\usepackage{fancyhdr}
\pagestyle{fancy}
\usepackage{tcolorbox}
\tcbuselibrary{theorems}
\usepackage{babel}
\usepackage{enumerate}
\usepackage{amsmath, amssymb, amsthm}
%\usepackage{a4wide}
\usepackage{float}
\usepackage{bbm}
\usepackage{tikz-cd}
\usepackage{tikz}
\usepackage{graphicx}
\usepackage{wrapfig}
\graphicspath{ {./images/} }
\usepackage{setspace}
\setstretch{1.1}
\usepackage{color}
\usepackage{hyperref}
\hypersetup{
    colorlinks=true, %set true if you want colored links
    linktoc=all,     %set to all if you want both sections and subsections linked
    linkcolor=black,  %choose some color if you want links to stand out
}

\theoremstyle{definition}
\newtheorem{theorem}{Theorem}[section]
\newtheorem{lemma}[theorem]{Lemma}
\newtheorem{cor}[theorem]{Corollary}
\newtheorem{prop}[theorem]{Proposition}
\newtheorem{example}[theorem]{Example}
\newtheorem{defn}[theorem]{Definition}

\theoremstyle{remark}
\newtheorem*{remark}{Remark}

\title{Part II - Number Theory
    \\ \large
    Lectured by Prof. T. A. Fisher
}
\author{Artur Avameri}
\date{Lent 2022}

% figure support
\usepackage{import}
\usepackage{xifthen}
\pdfminorversion=7
\usepackage{pdfpages}
\usepackage{transparent}
\newcommand{\incfig}[1]{%
    \def\svgwidth{\columnwidth}
    \import{./figures/}{#1.pdf_tex}
}

\pdfsuppresswarningpagegroup=1

\setcounter{section}{-1}
\begin{document}
\maketitle
\tableofcontents
\newpage

\section{Introduction}

\marginpar{06 Oct 2022, Lecture 1}

Books: 
\begin{itemize}
    \item A. Baker, \textit{A concise introduction to the theory of numbers}, CUP 1984
    \item N. Koblitz, \textit{A course in number theory \& cryptography}, Springer 1994
    \item H. Davenport, \textit{The higher arithmetic}, CUP 2008
\end{itemize}

Number theory studies the hidden and mysterious properties of the integers and the rational numbers.

It has always been an experimental science. Examining numerical data leads to \textbf{conjectures}, many of which are very old and still unproven today.

\begin{example}
    \begin{enumerate}[(i)]
        \item Let $N\ge 1$ be an integer of the form $8n+5, 8n+6$ or $8n+7$. Does there exist a right-angled triangle of area $N$, all of whose sides have rational length? We don't know.
        \item Let $\pi(x)$ be the number of primes less than or equal to $x$ and define $\text{li}(x) = \int_{2}^{x} \frac{dt}{\log t}$. Then for all $x \ge 3$, $|\pi(x) - \text{li}(x)| \le \sqrt{x}\log x$. This is in fact equivalent to the Riemann hypothesis.
        \item There are infinitely many twin primes. We now know there is an integer $N\le 246$ such that there are infinitely many pairs of primes the form $p, p+N$.
    \end{enumerate}
\end{example}

\newpage

\section{Euclid's algorithm and factoring}

\begin{defn}[Division algorithm]
    Given $a,b \in \mathbb{Z}$, with $b>0$, there exist $q,r \in \mathbb{Z}$ such that $a = qb + r$, and $0 \le r < b$.
\end{defn}

\textbf{Notation.} If $r=0$, then we write $b | a$, else $b\nmid a$. 

\begin{proof}
    Let $S = \{a - nb ~|~ n \in \mathbb{Z}\}$. This certainly contains integers $\ge 0$, so take the smallest one $r$. We claim $r < b$. Indeed, if not, then $r - b \ge 0$, contradicting minimality.
\end{proof}

Given $a_1, \ldots, a_n \in \mathbb{Z}$ not all zero, let $I = \{\lambda_1 a_1 + \ldots + \lambda_n a_n ~|~ \lambda_i \in \mathbb{Z}\}$.

\begin{lemma}
    $I = d \mathbb{Z}$ for some $d > 0$.
\end{lemma}
\begin{proof}
    $I$ certainly contains integers $\ge 0$. Let $d$ be the least positive element of $I$. We claim it works. Take $a \in I$, then $a = qd + r$ with $0 \le r < d$. But $r = a - qd \in I \implies r = 0$.
\end{proof}

\textbf{Remark.} We get from this that $d$ divides each $a_i$, and any common divisor of the $a_i$ must divide $d$. Why?

We write $d = \text{gcd}(a_1, \ldots, a_n)$ for the \textbf{greatest common divisor} (or \textbf{highest common factor}), or just use the shorthand $d = (a_1, \ldots, a_n)$.

\begin{cor}
    Let $a,b,c \in \mathbb{Z}$. Then there exist $x,y \in \mathbb{Z}$ such that $ax+by=c$ if and only if $(a,b) | c$.
\end{cor}

The division algorithm gives a very efficient way to compute $(a,b)$. Assume $a>b>0$. Apply the division algorithm recursively to get
\begin{align*}
    a &= q_1 b + r_1 & 0 \leq r_1 < b\\
    b &= q_2 r_1 + r_2 & 0 \leq r_2 < r_1\\
    r_1 &= q_3 r_2 + r_3 & 0 \leq r_3 < r_2\\
    &~ \vdots \\
    r_{k-2} &= q_k r_{k-1} + r_k & 0 \leq r_k < r_{k-1}, r_k \neq0\\
    r_{k-1} &= q_{k+1} r_k + 0
\end{align*}

\textbf{Claim.} 
    $r_k = (a,b)$.
Indeed, $(a,b) = (b, r_1) = (r_1, r_2) = \ldots = (r_{k-1}, r_k) = r_k$. This is called \textbf{Euclid's algorithm}.

\textbf{Remark.} If $d = (a,b)$, then by Lemma 1.2, there exist $r,s \in Z$ such that $ra + st = d$. Euclid's algorithm gives us a way to find $r$ and $s$.

\vspace{3mm}


In the following table, $x$ and $y$ stand for 34 and 25, and we then compute remainders as linear combinations of them.

We can use a trick here to speed this up: find each row as $q \cdot$ the row before it + the second row before it, then figure out signs at the end. (In fact, the minus signs zigzag down).

\begin{equation*}
    \begin{array}{r|cc}
        & x & y \\
        a=34 & 1 & 0 \\
        b=25 & 0 & 1 \\
        34 = 1 \cdot 25 + \color{red}{9}& 1 & -1 \\
        25 = 2 \cdot 9 + \color{red}{7}& -2 & 3 \\
        9 = 1 \cdot 7 + \color{red}{2}& 3 & -4 \\
        7 = 3 \cdot 2 + \color{red}{1}& -11 & 15 \\
    \end{array}
\end{equation*}

We hence get $-11 \cdot 34 + 15 \cdot 25 = 1$.

\begin{defn}
    An integer $n>1$ is \textbf{prime} if its only positive divisors are $1$ and $n$. Otherwise $n$ is \textbf{composite}.
\end{defn}
\begin{lemma} %1.2 in lectures
    Let $p$ be a prime, and $a,b \in \mathbb{Z}$. If $p | ab$, then $p \mid a$ or $p \mid b$.
\end{lemma}
\begin{proof}
    Assume $p \nmid a$. Then $(a,p) = 1$. By Lemma 1.2, $\exists r,s \in \mathbb{Z}$ such that $ra + sp = 1 \implies rab + spb = b$. Since $p \mid ab$, $p \mid b$ follows.
\end{proof}

\begin{theorem}[\textbf{Fundamental Theorem of Arithmetic}] 
    Every integer ${n>1}$ can be written as a product of primes. This representation is unique up to reordering.
\end{theorem}
\begin{proof}
    Existence is obvious. For uniqueness, suppose $n = p_1p_2 \ldots p_r = q_1 q_2 \ldots q_s$ for $p_i, q_i$ primes. We have $p_1 \mid q_1 q_2 \ldots q_r$, so by Lemma 1.5, $p_1 \mid q_j$ for some $j$, so $p_1 = q_j$. Now cancel these out and induct.
\end{proof}

\textbf{Remark.} If $m = \prod_{i=1}^{k} p_i^{\alpha_i}$ and $n = \prod_{i=1}^{k} p_i^{\beta_i}$ for $p_i$ distinct primes and $\alpha_i, \beta_i \ge 0$, then \[
(m,n) = \prod_{i=1}^{k} p_i^{\text{min}(\alpha_i,\beta_i)}.
\]

However, if $m$ and $n$ are large, it is more efficient to compute $(m,n)$ using Euclid's algorithm.

\marginpar{08 Oct 2022, Lecture 2}

\vspace{1mm}

Suppose we have some large positive integer $N$. An obvious algorithm for factoring $N$ is to trial divide by 2 and the odd integers up to $\sqrt{N}$.

\begin{defn}
    An algorithm with input a positive integer $N$ is \textbf{polynomial} or a \textbf{polynomial time} algorithm if it takes $\le c (\log N)^b$ \textbf{elementary operations}  for some constants $b$ and $c$. 
\end{defn}

\textbf{Remark.} An elementary operation is just adding/multiplying two numbers in $\{0,1,\ldots,9\}$.

\textbf{Remark.} "Polynomial" makes sense here as it takes $\log N$ digits to write $N$.

\vspace{1mm}

Polynomial algorithms are known for: 
\begin{itemize}
    \item Adding and multiplying integers (the usual way);
    \item Computing gcd's (via Euclid's algorithm);
    \item Detecting $n^{\text{th}}$ powers (compute $\sqrt[n]{}$ numberically and round)
    \item More remarkably, primality testing (Agrawal, Kayal, Saxena in 2002)
\end{itemize}

But trial division up to $\sqrt{N}$ is not polynomial.

\vspace{1mm}

\textbf{Fundamental question:} Is there a polynomial time algorithm for factoring? This is unknown.

Later in this course we study the distribution of the prime numbers, in particular the function $\pi(x)$, the number of primes $\le x$.

\begin{theorem}
    There are infinitely many prime numbers, i.e. ${\lim_{x \to \infty} \pi(x) \to \infty}$.
\end{theorem}
\begin{proof}
    Suppose there are only finitely many, say $p_1,\ldots, p_k$. Consider $N = \prod_{i=1}^{k} p_k + 1$. Then $N$ must be divisible by some prime other than the $p_i$, so we're done.
\end{proof}

All the largest known primes are of the form $2^n-1$ for $n$ a prime. These are called \textbf{Mersenne primes}. 51 of them are known, the largest being $2^{82589933}-1$.


\section{Congruences}

Fix a positive integer $n$ (the modulus).

\begin{defn}
    We say $a \equiv b \pmod{n}$, or that $a$ is congruent to $b \pmod{n}$ if $n$ divides $a-b$. 
\end{defn}

This defines an equivalence relation on $\mathbb{Z}$, and we write $\mathbb{Z}/n\mathbb{Z}$ for the set of equivalence classes. We can denote these by $a + n\mathbb{Z}$, or (more commonly) by ${a \pmod{n}}$. We can check that addition and multiplication are well-defined.

\vspace{1mm}

\textbf{Remark.} $n\mathbb{Z}$ is a subgroup/ideal of $\mathbb{Z}$ and $\mathbb{Z}/n\mathbb{Z}$ is the quotient group/ring.

\begin{lemma}
    Let $a \in \mathbb{Z}/n\mathbb{Z}$. Then the following are equivalent:
    \begin{enumerate}[(i)]
        \item $(a,n) = 1$
        \item $\exists b \in \mathbb{Z}$ such that $ab \equiv 1 \pmod{n}$
        \item $a$ is a generator for $\mathbb{Z}/n\mathbb{Z}$.
    \end{enumerate}
\end{lemma}
\begin{proof}
    (i)$\implies $ (ii): $(a,n)=1 \implies \exists r,s \in \mathbb{Z}$ such that $ra + sn = 1$, so $ra \equiv 1 \pmod{n}$.

    (ii) $\implies $ (i): $ab \equiv 1 \pmod{n} \implies ab + kn = 1$ for some $k \in \mathbb{Z} \implies {(a,b)=1}$.

    (ii) $\iff $(iii): $\exists b \in \mathbb{Z}$ s.t. $ab \equiv 1\pmod{n} \iff 1$ belongs to the subgroup of $\mathbb{Z}/n\mathbb{Z}$ generated by $a$.
\end{proof}

\textbf{Notation.} $(\mathbb{Z}/n\mathbb{Z})^{\times}$ is the group of \textbf{units} in $\mathbb{Z}/n\mathbb{Z}$, i.e. the elements with an inverse under multiplication.

\begin{defn}
    $\phi(n) = |(\mathbb{Z}/n\mathbb{Z})^{\times}|$ is called the \textbf{Euler totient function}. We also have $\phi(n) =|\{1 \le a \le n ~|~ (a,n) = 1\}|$. 
\end{defn}

\textbf{Remark.} $\mathbb{Z}/n\mathbb{Z}$ is a field $\iff \phi(n) = n -1 \iff n$ is prime. 

\begin{theorem}[Euler-Fermat theorem]
    If $(a,n)=1$, then $a^{\phi(n)} \equiv 1\pmod{n}$.
\end{theorem}
\begin{proof}
    Apply Lagrange's theorem to the group $G = (\mathbb{Z}/n\mathbb{Z})^{\times}$. Then for $a \in G$, its order divides $|G|= \phi(n)$.
\end{proof}
As a corollary:
\begin{theorem}[Fermat's little theorem]
    If $p \nmid a$, then $a^{p-1} \equiv 1\pmod{p}$.
\end{theorem}

\begin{lemma}%lemma 2.2 in lectures
    Let $G$ be a cyclic group of order $n$. We have \[
    |\{g \in G ~|~ \text{order}(g) = d\}| = \begin{cases}
        \phi(d) &\text{if }d \mid n \\
        0 &\text{otherwise}
    \end{cases}
    \]
    In particular, $\sum_{d \mid n}^{} \phi(d) = n$.
\end{lemma}
\begin{proof}
    WLOG let $G = (\mathbb{Z}/n\mathbb{Z}, +)$. We have $|\{g \in G ~|~ \text{order}(g) = n\}| \stackrel{(*)}{=} \phi(n)$ by Lemma 2.2. If $d \mid n$, say $n = dk$, then the elements of order dividing $d$ are the classes $0, k, 2k, \ldots, (d-1)k \pmod{n}$. These form a cyclic subgroup of order $d$. Applying $(*)$ to this cyclic subgroup shows that there are $\phi(d)$ elements of order $d$.
\end{proof}

\begin{example}
    Consider the simultaneous linear congruences $x \equiv 7 \pmod{10}$ and $x \equiv 3 \pmod{13}$. Suppose we can find $u,v \in \mathbb{Z}$ such that \[
    \begin{cases}
        u \equiv 1 \pmod{10}\\
        u \equiv 0 \pmod{13}
    \end{cases},
    \begin{cases}
        v \equiv 0 \pmod{10}\\
        v \equiv 1 \pmod{13}
    \end{cases}.
    \]
    Then $x = 7u + 3v$ is a solution. But $(10,13) = 1 \implies \exists r,s \in \mathbb{Z}$ such that $10r+13s = 1$, and we can just take $u = 13s, v = 10r$. To find $r,s$, we can use Euclid's algorithm to get $r=4, s= -3$, so $u = -39, v = 40$, and so $x \equiv 7 \cdot (-39) + 3 \cdot 40 \equiv 107 \pmod{130}$.
\end{example}

\marginpar{11 Oct 2022, Lecture 3}

\begin{theorem}[Chinese Remainder Theorem]
    Let $m_1,\ldots, m_k$ be pairwise coprime integers greater than 1. Let $a_1,\ldots, a_k \in \mathbb{Z}$. Let $M = m_1 m_2 \ldots m_k$. Then $\exists  x \in \mathbb{Z}$ satisfying 
    \[
    \begin{cases}
        x \equiv a_1 \pmod{m_1} \\
        \vdots \\
        x \equiv a_k \pmod{m_k}
        \end{cases}.
    \]
        Moreover, the solution is unique mod $M$.
\end{theorem}
\begin{proof}
    Uniqueness: Suppose $x \equiv x' \pmod{m_i} ~\forall i$. Then by considering the prime factorization of $x-x'$ and using the fact that the $m_i$ are pairwise coprime, we get $x \equiv x' \pmod{M}$.
    \vspace{1mm}
    
    Existence: Put $M_i = \frac{M}{m_i}$, so $(M_i, m_i) = 1 ~\forall i$. Hence we can find $u_i \in \mathbb{Z}$ such that $u_i M_i \equiv 1 \pmod{m_i} ~\forall i$. Let $x = \sum_{j=1}^{k} a_j u_j M_j$. Then $x \equiv a_iu_iM_i \equiv a_i \pmod{m_i}$.
\end{proof}
We can write this theorem in one ling using ring theory.
\begin{defn}
    Let $R_i = \mathbb{Z}/m_i\mathbb{Z}$, and define $R_1 \times \ldots \times R_k = \{(r_1,\ldots,r_k) ~|~ r_i \in R_i\}$ with addition and multiplication defined componentwise. This is a ring.
\end{defn}
\begin{theorem}[CRT, ring-theoretic version] %Theorem 2.3 in lectures
    Let $m_1,\ldots, m_k$ be pairwise coprime integers greater than 1 and put $M = m_1\ldots m_k$. Then the map 
    \begin{align*}
        \theta: \mathbb{Z}/M\mathbb{Z} \to \mathbb{Z}/m_1\mathbb{Z} \times \ldots \times \mathbb{Z}/m_k\mathbb{Z} \\
        a + M\mathbb{Z} \mapsto (a + m_1\mathbb{Z}, \ldots, a + m_k\mathbb{Z})
    \end{align*}
    is an isomorphism of rings.
\end{theorem}
\begin{proof}
    $\theta$ is a well defined ring homomorphism since $m_i | M ~\forall i$. Injectivity of $\theta$ follows from uniqueness in CRT, and surjectivity of $\theta$ follows from existence in CRT.
\end{proof}
\begin{cor}
    $\theta$ induces an isomorphism of groups under multiplication 
    \begin{align*}
        (\mathbb{Z}/m\mathbb{Z})^\times \cong (\mathbb{Z}/m_1\mathbb{Z})^\times \times \ldots \times (\mathbb{Z}/m_k\mathbb{Z})^\times \\
        a + M\mathbb{Z} \mapsto (a + m_1\mathbb{Z}, \ldots, a+m_k\mathbb{Z}).
    \end{align*}
\end{cor}
\textbf{Remark.} If $a \in \mathbb{Z}$, then $(a,M) = 1 \iff (a,m_i) = 1 ~\forall i$.

In particular, by looking at orders of the LHS and the RHS above, we get $\phi(M) = \phi(m_1)\ldots \phi(m_k)$, i.e. the Euler phi function is multiplicative.

\begin{defn}
    A function $f : \mathbb{Z}^+ \to \mathbb{C}$ is \textbf{multiplicative} if $f(m) = f(m)f(n)$ whenever $(m,n)=1$.
\end{defn}
\textbf{Examples}: 
\begin{itemize}
    \item $\phi(n) = |(\mathbb{Z}/n\mathbb{Z})^{\times}|$;
    \item $\tau(n) = \sum_{d \mid n}^{} 1$, the number of divisors of $n$;
    \item $\sigma(n) = \sum_{d \mid n}^{} d$, the sum of divisors of $n$;
    \item more generally, $\sigma_k(n) = \sum_{d \mid n}^{} d^k$, so $\sigma_0 = \tau$ and $\sigma_1 = \sigma$.
\end{itemize} 
To prove this:
\begin{lemma} % Lemma 2.4 in lectures
    If $f : \mathbb{Z}^+ \to \mathbb{C}$ is multiplicative, then so is $g : \mathbb{Z}^+ \to \mathbb{C}$, defined by $g(n) = \sum_{d \mid n}^{} f(d)$.
\end{lemma}
\begin{proof}
    Let $m,n$ be coprime. Note that every divisor $d$ of $mn$ can be written as $d = d_1d_2$, where $d_1 \mid m$, $d_2 \mid  n$ and $(d_1,d_2)=1$. Thus \[
    g(mn) = \sum_{d \mid mn}^{} f(d) = \sum_{d_1 \mid m}^{} \sum_{d_2 \mid n}^{} f(d_1d_2) = \sum_{d_1 \mid m}^{} \sum_{d_2 \mid n}^{} f(d_1)f(d_2) = g(m)g(n).  
    \]
\end{proof}
\begin{lemma} % Lemma 2.5 in lectures
    \begin{enumerate}[(i)]
        \item For $p$ a prime, $\phi(p^k) = p^{k-1}(p-1) = p^k(1-\frac{1}{p})$.
        \item $\phi(n) = n \prod_{p \mid n}^{} (1-\frac{1}{p})$.
    \end{enumerate}
\end{lemma}
\begin{proof}
    (i): $\phi(p^k)$ counts the number of integers $a$ between $1$ and $p^k$ such that $(p^k,a) = (p,a) = 1$. So we have $p^a$ numbers, and we don't count the multiples of $p$, so $\phi(p^k) = p^k - p^{k-1}$.
    \vspace{1mm}
    
    (ii): Follows from the fact that $\phi$ is multiplicative. 
\end{proof}

\textbf{Alternative proof} that $\sum_{d \mid n}^{} \phi(d) = n$ (cf Lemma 2.6).
\begin{proof}
    Obviously the RHS is multiplicative. Since $\phi(n)$ is multiplicative, the LHS is multiplicative by Lemma 2.13, so it suffices to check for $n$ a prime power, say $n=p^k$. To this end, compute \[
    \sum_{d \mid p^k}^{} \phi(d) = \phi(1) + \phi(p) + \ldots + \phi(p^k) = 1 + (p-1) + (p^2-p) + \ldots + (p^k-p^{k-1}) = p^k.
    \]
\end{proof}

\subsection{Polynomial congruences}
Let $R = \mathbb{Z},\mathbb{Q}, \mathbb{Z}/n\mathbb{Z}$ (or more generally any commutative ring). Set $R[X] = \{\text{\textbf{polynomials} with coefficients in }R\}$, i.e. $a_n X^n + a_{n-1}X^{n-1} + \ldots + a_1 X + a_0$ for $a_i \in R$.

By definition, two polynomials are equal if and only if they have the same coefficients. We can check that $R[X]$ is a ring (with usual $+$ and $\times$).
\vspace{1mm}

\textbf{Warning}. The map $R[X] \to \{\text{functions }R \to R\}$ by $f \mapsto (\alpha \mapsto f(\alpha))$ is not always injective. For example, if $R=\mathbb{Z}/p\mathbb{Z}$ for $p$ a prime, and $f(X)=X^p - X$, then $f(\alpha)=0 ~\forall \alpha \in R$, but $f$ is not the zero function.
\vspace{1mm}

\textbf{Question.} Can we show thatif $f \in R[X]$ has degree $n$, then $f$ has at most $n$ roots in $R$?
\vspace{1mm}

\textbf{Answer.} No. For example, take $R = \mathbb{Z}/8\mathbb{Z}$, then $f(X)=X^2-1$ has 4 solutions in $\mathbb{Z}/8\mathbb{Z}$.

\marginpar{13 Oct 2022, Lecture 4}
\vspace{1mm}

Let $R=\mathbb{Z},\mathbb{Q}, \mathbb{Z}/n\mathbb{Z}$ (or any commutative ring).

We have a \textbf{division algorithm} on $R[X]$: 

Let $f,g \in R[X]$ and suppose the leading coefficient of $g$ is a unit. Then $\exists q, r \in R[X]$ such that $f(X)=Q(X)g(X)+r(X)$ and $\text{deg}(r)<\text{deg}(g)$.

\begin{proof}
    By induction on $\text{deg}(f)$. If $\text{deg}(f) < \text{deg}(g)$, take $q=0, r=f$. Otherwise, let $f(X) = a X^m + \ldots$ and $g(X) = b X^n + \ldots$ with $m \ge n$ and $b$ a unit.
    \vspace{1mm}
    
    Let $f_1(X) = f(X)- ab^{-1}X^{m-n}g(X)$. Then $\text{deg}(f_1) < \text{deg}(f)$, so by the induction hypothesis, $f_1(x)=q_1(x)g(x) + r_1(x)$ for some $q_1,r_1 \in R[X]$ and $\text{deg}(r_1) < \text{deg}(g)$. Now take $q(X) = ab^{-1}X^{m-n} + q_1(X)$ and $r=r_1$, so we're done.
\end{proof}

\begin{cor} % Corollary 2.6 in lectures
    If $f \in R[X]$ and $\alpha \in R$ is such that $f(\alpha)=0$, then $f(X) = (X-\alpha)f_1(X)$ for some $f_1 \in R[X]$. 
\end{cor}
\begin{proof}
    By the division algorithm, $f(X) = (X-\alpha) f_1(X) + r$ for some $r \in R$ (as $\text{deg}(r) < \text{deg}(X-\alpha)$). Plug in $X=\alpha$ to get $r=0$.
\end{proof}
\begin{defn}
    $R$ is an \textbf{integral domain} if $R$ has no zero divisors, i.e. $\alpha,\beta \in R$, $\alpha \beta=0 \implies \alpha=0$ or $\beta=0$.
\end{defn}
\textbf{Note.} Let $n>1$. Then $\mathbb{Z}/n\mathbb{Z}$ is an integral domain $\iff$ $n$ is prime.

\begin{theorem} % Theorem 2.7 in lectures
    If $R$ is an integral domain, then any polynomial $f \in R[X]$ of degree $n$ has at most $n$ roots.
\end{theorem}
\begin{proof}
    By induction on $n$, the degree of $f$. If $n=0$, then our polynomial is a nonzero constant and we're done. Now suppose $\exists \alpha \in R$ such that $f(\alpha)=0$ (otherwise we're done). By Corollary 2.15, $f(X) = (X-\alpha)f_1(X)$. Since $R$ is an integral domain, every root of $f$, except possibly $\alpha$ is a root of $f_1$. By induction, $f_1$ has at most $n-1$ roots, hence $f$ has at most $n$ roots and we're done.
\end{proof}
\begin{cor}[Lagrange's Theorem]
    Let $p$ be a prime and $a_0,\ldots,a_n \in \mathbb{Z}$ with $p \nmid a_n$. Then the congruence 
    \begin{align*}
        a_nx^n + a_{n-1}x^{n-1} + \ldots + a_1x + a_0 \equiv 0 \pmod{p}
    \end{align*}
    has at most $n$ solutions mod $p$.
\end{cor}
\begin{proof}
    Take $R=\mathbb{Z}/p\mathbb{Z}$ in Theorem 2.17.
\end{proof}
\textbf{Remark.} In this course, we will refer to the above theorem as Lagrange's Theorem.
\begin{example}
    Let $p$ be a prime. We will factor $X^{p-1}- 1 \pmod{p}$. Let $f(X) = X^{p-1}-1 - \prod_{a=1}^{p-1}(X-\alpha)$ in $\mathbb{Z}/p\mathbb{Z}[X]$. By Fermat's Little Theorem, $f$ has at least $p-1$ roots mod $p$. But $\text{deg}(f) < p-1$, since the $X^{p-1}$ terms cancel out, so by Lagrange's Theorem, $f = 0$, i.e. $X^{p-1} -1 = \prod_{a=1}^{p-1} (X-a)$ in $\mathbb{Z}/p\mathbb{Z}[X]$. Plugging in $X=0$ gives $(p-1)! \equiv -1 \pmod{p}$, i.e. Wilson's Theorem.
\end{example}
\begin{example}
    Working mod $7$, the powers of $3$ (starting from 0) are $1,3,2,6,4,5$. So $(\mathbb{Z}/7\mathbb{Z})^{\times}$ is cyclic, generated by $3$.
\end{example}
\begin{theorem}%Theorem 2.8 in lectures
    Let $p$ be a prime. Then $(\mathbb{Z}/p\mathbb{Z})^{\times}$ is cyclic.
\end{theorem}
\begin{proof}
    Let $S_d = \{a \in (\mathbb{Z}/p\mathbb{Z})^{\times} ~|~ \text{ord}(a) = d\}$. Suppose $S_d \neq \emptyset$, say $a \in S_d$. Then $1,a,a^2,\ldots,a^{d-1}$ are distinct elements in $\mathbb{Z}/p\mathbb{Z}$ and they are solutions of $x^d \equiv 1 \pmod{p}$. By Lagrange's theorem, this has at most $d$ solutions, and we found $d$ solutions, so those are all of them, i.e. $S_d \subseteq \{1,a,a^2,\ldots,a^{d-1}\}$. Note that the LHS is a cyclic group of order $d$, so this has $\phi(d)$ elements of order $d$.
    \vspace{1mm}
    
    We conclude that for every $d$, $|S_d| = 0$ or $|S_d| = \phi(d)$. In particular, ${|S_d| \le \phi(d)}$. Hence
    \begin{align*}
       p-1 \stackrel{(\star)}{=} \sum_{d \mid (p-1)}^{} |S_d| \le  \sum_{d \mid (p-1)}^{} \phi(d) = p-1,
    \end{align*}
    where $(\star)$ follows since we just count all the elements in $(\mathbb{Z}/p\mathbb{Z})^{\times}$. Hence ${|S_d| = \phi(d) ~\forall  d \mid (p-1)}$. In particular, $S_{p-1} \neq \emptyset$, i.e. $(\mathbb{Z}/p\mathbb{Z})^{\times}$ contains elements of order $p-1$, i.e. $(\mathbb{Z}/p\mathbb{Z})^{\times}$ is cyclic. 
\end{proof}
\textbf{Remark.} The same argument shows that any finite subgroup of the multiplicative group of a field is cyclic.
\begin{defn}
    An integer $a$ such that $a \pmod{n}$ generates $(\mathbb{Z}/n\mathbb{Z})^{\times}$ is called a \textbf{primitive root} mod $n$. 
\end{defn}
Theorem 2.21 showed that primitive roots exist mod $p$.
\begin{example}
    Let $p = 19$. Let $d$ be the order of $2$ in $(\mathbb{Z}/19\mathbb{Z})^{\times}$. We know $d \mid 18$, so we work out 
    \begin{align*}
        2^3 &\equiv 8 \pmod{19}\\
        2^6 &\equiv 7 \not\equiv 1 \pmod{19} \implies d \nmid 6\\
        2^9 &\equiv -1 \not\equiv 1 \pmod{19} \implies d \nmid 9,
    \end{align*}
    so $d=18$ and hence 2 is a primitive root mod $19$.
\end{example}
In general, $g \in \mathbb{Z}$ (coprime to $p$) is a primitive root mod $p$ if and only if $g^{\frac{p-1}{q}} \not\equiv 1 \pmod{p}$ $~\forall \text{primes }q \mid (p-1)$.

\marginpar{15 Oct 2022, Lecture 5}

\textbf{Remark.} The number of primitive roots mod $p$ is $\phi(p-1)=\phi(\phi(p))$.
\vspace{1mm}

Here are some (open) problems concerning primitive roots:
\begin{enumerate}[(i)]
    \item Artin's conjecture (1927) -- Let $a>1$ be an integer which is not a square. Then $a$ is a primitive root mod $p$ for infinitely many primes $p$. This is unknown for $a=2$. Hooley (1967) proved this assuming GRH. Heath-Brown (1986) proved that Artin's conjecture holds for at least one of $2,3$ or $5$. In fact, he proved something stronger: he proved the conjecture fails for at most 2 prime values of $a$.   
    \item How large is the smallest primitive root mod $p$? Burgess (1962) showed it is $\le c p^{1/4 + \epsilon} ~\forall  \epsilon>0$ and some constant $c=c(\epsilon)$. Shoup (1992) showed it is $\le c(\log p)^6$ assuming GRH.
\end{enumerate}

We now consider $\mathbb{Z}/p^n\mathbb{Z}$ for $n>1$. For $n \ge 3$, there is a surjective group homomorphism from $(\mathbb{Z}/2^n\mathbb{Z})^{\times} \to (\mathbb{Z}/8\mathbb{Z})^{\times} = \{\pm 1, \pm3\} \cong C_2 \times C_2$, so $(\mathbb{Z}/2^{n}\mathbb{Z})^{\times}$ is not cyclic (since generators map to generators).

\begin{theorem}% Theorem 2.9 in lectures
    Let $p$ be an odd prime. Then $(\mathbb{Z}/p^n\mathbb{Z})^{\times}$ is cyclic $~\forall n\ge 1$.
\end{theorem}
We divide the proof into 3 lemmas.
\begin{lemma}%Lemma 2.10 in lectures
    Let $n \ge 2$. Then $g$ is a primitive root mod $p^n$ if and only if the following two conditions hold: 
    \[
    \begin{cases}
        g \text{ is a primitive root mod }p \\
        g^{p^{n-2}(p-1)} \not\equiv 1 \pmod{p^n}
    \end{cases}.
    \]
\end{lemma}
\begin{proof}
    ($\implies $) is clear, as $\phi(p^n)=p^{n-1}(p-1)$.

    ($\impliedby$): Let $d$ be the order of $g$ in $(\mathbb{Z}/p^n\mathbb{Z})^{\times}$. Then $d \mid \phi(p^n) = p^{n-1}(p-1)$. Since $g^d \equiv 1 \pmod{p^n}$, we have $g^d \equiv 1 \pmod{p}$. Hence by assumption 1, we have $(p-1)\mid d$. Say $d = p^j (p-1)$ for some $0\le j \le n-1$. If $j\le n-2$, then this contradicts assumption 2. Hence $j=n-1$, so $d=\phi(p^n)$ is a primtive root mod $p^n$.
\end{proof}
Next we show $\exists g \in \mathbb{Z}$ satisfying conditions 1 and 2 in the case $n=2$.
\begin{lemma}
    $\exists g \in \mathbb{Z}$ a primitive root mod $p$ such that $g^{p-1} \not\equiv 1 \pmod{p^2}$.
\end{lemma}
\begin{proof}
    Let $g$ be a primtive root mod $p$. If $g^{p-1} \equiv 1\pmod{p^2}$, then consider $g+p$, which is still a primtive root mod $p$, but
    \[
    (g+p)^{p-1}=g^{p-1} + (p-1)g^{p-2}p + \ldots \equiv 1 + (p-1)g^{p-2}p \pmod{p^2},
    \]
    where the second term is not divisible by $p^2$, so $(g+p)^{p-1} \not\equiv 1 \pmod{p^2}$.
\end{proof}
Next we show that if $g$ is a primitive root mod $p^2$, then it is a primitive root mod $p^n ~\forall n\ge 2$.
\begin{lemma}
    If $g^{p-1} \not\equiv 1 \pmod{p^2}$, then $g^{p^{n-2}(p-1)} \not\equiv 1\pmod{p^n} ~\forall n\ge 2$. 
\end{lemma}
\begin{proof}
    By induction on $n$, the case $n=2$ being given. Suppose the result is true for $n$. By Euler-Fermat, $g^{p^{n-2}(p-1)} \equiv 1 \pmod{p^{n-1}}$, so $g^{p^{n-2}(p-1)} = 1 +b p^{n-1}$ for some $b \in \mathbb{Z}$, where $p \nmid b$ by the induction hypothesis. Taking $p^{\text{th}}$ powers gives 
    \begin{align*}
        g^{p^{n-1}(p-1)} = (1+bp^{n-1})^p = 1 + bp^n + {{p} \choose {2}}b^2p^{2(n-1)}+\ldots \equiv \\ 1 + bp^n + {{p} \choose {2}}b^2 p^{2(n-1)} \stackrel{\star}{\equiv } 1 + bp^n \pmod{p^{n+1}},
    \end{align*}
    where $\star$ follows since $p$ is odd, so $p \mid {{p}\choose{2}}$ (and also we use $3(n-1)\ge n+1$ and $2(n-1)+1\ge n+1$). Thus $g^{p^{n-1}(n-1)} \equiv 1 +bp^n \not\equiv 1 \pmod{p^{n+1}}$, so the result follows for $n+1$.
\end{proof}
This completes the proof of Theorem 2.24.

\begin{example}
    We saw $3$ is a primitive root mod $7$. We calculate $3^3 = -1 + 4\cdot 7$, so $3^6 \equiv 1 - 8\cdot 7 \not\equiv 1 \pmod{7^2}$. Hence $3$ is a primitive root mod $7^n ~\forall n$.
\end{example}

For the case $p=2$, let $G = \{a \in (\mathbb{Z}/2^n\mathbb{Z})^{\times} ~|~ a \equiv 1 \pmod{4}\}$. Then $(\mathbb{Z}/2^n\mathbb{Z})^{\times} \cong \{\pm 1\} \times G$ by $a+2^n\mathbb{Z} \mapsto \begin{cases}
    (1, a+2^n\mathbb{Z}) &\text{ if }a \equiv 1\pmod{4}\\
    (-1, -a+2^n\mathbb{Z}) &\text{ if }a \equiv 3\pmod{4}.
\end{cases}$

\textbf{Exercise.} Show that $G$ is cyclic (and generated by 5).

\textbf{Exercise.} For which $n$ is $(\mathbb{Z}/n\mathbb{Z})^{\times}$ cyclic? 

\end{document}

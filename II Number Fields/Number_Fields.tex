\documentclass{article}
%build with recipe latexmk
\usepackage[utf8]{inputenc}
\usepackage[T1]{fontenc}
\usepackage{textcomp}
\usepackage{fancyhdr}
\pagestyle{fancy}
%\addtolength{\headwidth}{\marginparwidth}
%\addtolength{\headwidth}{\marginparsep}
%\addtolength{\headwidth}{\marginparsep}
\usepackage{tcolorbox}
\tcbuselibrary{theorems}
\usepackage{babel}
\usepackage{enumerate}
\usepackage{amsmath, amssymb, amsthm}
%\usepackage{a4wide}
\usepackage{float}
\usepackage{bbm}
\usepackage{tikz-cd}
\usepackage{tikz}
\usepackage{graphicx}
\usepackage{wrapfig}
\graphicspath{ {./images/} }
\usepackage{setspace}
\setstretch{1.1}
\usepackage{color}
\usepackage{hyperref}
\hypersetup{
    colorlinks=true, %set true if you want colored links
    linktoc=all,     %set to all if you want both sections and subsections linked
    linkcolor=black,  %choose some color if you want links to stand out
}

\theoremstyle{definition}
\newtheorem{theorem}{Theorem}[section]
\newtheorem{lemma}[theorem]{Lemma}
\newtheorem{cor}[theorem]{Corollary}
\newtheorem{prop}[theorem]{Proposition}
\newtheorem{example}{Example}[section]
\newtheorem{defn}{Definition}[section]
\newtheorem{problem}{Problem}

\title{Part II - Number Fields
    \\ \large
    Lectured by  
}

\author{Artur Avameri}
\date{Lent 2023}

% figure support
\usepackage{import}
\usepackage{xifthen}
\pdfminorversion=7
\usepackage{pdfpages}
\usepackage{transparent}
\newcommand{\incfig}[1]{%
    \def\svgwidth{\columnwidth}
    \import{./figures/}{#1.pdf_tex}
}

\pdfsuppresswarningpagegroup=1

\setcounter{section}{-1}

\begin{document}
\maketitle
\tableofcontents
\newpage

\section{Introduction}
\marginpar{19 Jan 2022, Lecture 1}

This is one of the three deep undergraduate courses, the others being Algebraic Geometry and Algebraic Topology. You should take all three if you one day want to be a mathematician.

\section{some title idk what to put here yet}
\begin{defn}
    Recall that for $K, L$ fields, $K \subset L$, we have a \textbf{finite extension} if $\text{dim}_{K} L < \infty$. This dimension is also written as $[L:K]$, is called the degree of the field extension, and is the dimension of $L$ as a $K$--vector space.
\end{defn}
\begin{defn}
    A \textbf{number field} is a finite extension over $\mathbb{Q}$.
\end{defn}
\begin{defn}
    For $\mathbb{Q} \subset L$ a number field, $\alpha \in L$ is an \textbf{algebraic integer} if $\exists f \in \mathbb{Z}[X]$ monic such that $f(\alpha)=0$. 
    \vspace{1mm}
    
    We write $\mathcal{O}_L = \{\alpha \in L \mid \alpha \text{ is an algebraic integer}\}$, the "integers of $L$".
\end{defn}
\begin{lemma}
    $\mathcal{O}_{\mathbb{Q}} = \mathbb{Z}$, i.e. $\alpha \in \mathbb{Q}$ is an algebraic integer if and only if $\alpha \in \mathbb{Z}$.
\end{lemma}
\begin{proof}
    $(\impliedby)$: if $\alpha \in \mathbb{Z}$, then $f(x)=x-\alpha$ is a monic polynomial in $\mathbb{Z}[X]$ such that $f(\alpha)=0$.
    \vspace{1mm}
    
    $(\implies)$: Let $\alpha \in \mathbb{Q}$ with $\alpha=\frac{r}{s}$ for $r,s$ coprime in $\mathbb{Z}$. Say $f(x)=x^n+a_{n-1}x^{n-1}+\ldots+a_0 \in \mathbb{Z}[X]$ such that $f(\alpha)=0$. Clear denominators to get \[
    r^n + a_{n-1}r^{n-1}s + \ldots + a_0s^n = 0.
    \] 
    But $s$ divides everything but the first term, so $s \mid r^n$. If $s \neq 1$, then let $p \mid s$ for $p$ a prime, then $p \mid r$, contradicting $\gcd(r,s)=1$.
\end{proof}
\begin{theorem}
    $\mathcal{O}_L$ is a ring, i.e. if $\alpha,\beta \in L$, then $\alpha \pm \beta, \alpha \beta \in L$.
\end{theorem}
\textbf{Remark.} $\frac{1}{\alpha}$ usually isn't in $\mathcal{O}_L$.
\vspace{1mm}

Recall from Galois Theory that for $\alpha,\beta \in L$ and $K \subset L$, if $\alpha,\beta$ are algebraic over $K$, then so are $\alpha \pm \beta$ and $\alpha \beta$. So finite extensions imply algebraic.

\begin{defn}
    Let $R \subset S$ be rings. Then
    \begin{enumerate}[(i)]
        \item $\alpha \in S$ is \textbf{integral over $R$} if there exists a monic polynomial $f(x)=x^n + a_{n-1}x^{n-1} + \ldots + a_0 \in R[X]$ such that $f(\alpha)=0$.
        \item $S$ is \textbf{integral over $R$} if all $\alpha \in S$ are integral over $R$.
        \item $S$ is \textbf{finitely generated over $R$} (abbreviated to f.g. $/R$) if there exist elements $\alpha_1,\ldots,\alpha_n \in S$ such that any element of $S$ can be written as an $R$--linear combination $\alpha_1,\ldots,\alpha_n$, i.e. if the map $R^n \to S$ taking $(r_1,\ldots,r_n) \to \sum_{}^{} r_i \alpha_i$ is surjective. 
    \end{enumerate}
\end{defn}
\begin{example}
    Say we have a number field arising from the following picture: *picture*

    Then $\alpha \in L$ is an algebraic integer $\iff$ $\alpha$ is integral over $\mathbb{Z}$ (this follows by definition), and $\mathcal{O}_L$ is integral over $\mathbb{Z}$ (once we know it is a ring).
\end{example}
\textbf{Notation:} If $\alpha_1,\ldots,\alpha_r \in S$, then write $R[\alpha_1,\ldots,\alpha_r]$ for the subring of $S$ generated by $R,\alpha_1,\ldots,\alpha_r$. This is the image of the polynomial ring $R[x_1,\ldots,x_r] \to S$ given by $x_i \mapsto \alpha_i$.
\begin{prop}\label{1.3}
    \begin{enumerate}[(i)]
        \item If $S=R[s]$ with $s$ integral over $R$, then $S$ is finitely generated over $R$.
        \item If $S=R[s_1,\ldots,s_n]$ with each $s_i$ integral over $R$, then $S$ is finitely generated over $R$.
    \end{enumerate}
\end{prop}
\begin{proof}
    \begin{enumerate}[(i)]
        \item $S$ is spanned by $1,s,s^2,\ldots$ over $R$. By assumption $\exists a_0,\ldots,a_{n-1} \in R$ such that $s^n = \sum_{i=0}^{n-1} a_is^i$, so the $R$--module spanned by $1,\ldots,s^{n-1}$ is stable by multiplication by $s$, so it contains $s^{n+1},s^{n+2},\ldots$, so is $S$.
        \item Let $S_i = R[s_1,\ldots,s_{i-1}]$, so $S_{i+1}=S_i[s_{i+1}]$ and $s_{i+1}$ is integral over $R$, hence integral over $S_i$. Hence by (i), $S_{i+1}$ is finitely generated over $S_i$. To finish, we need to show the following fact, which is left as an exercise:
        \vspace{1mm}
        
        \textbf{Exercise:} For $A \subset B \subset C$, if $B$ is finitely generated over $A$ and $C$ is finitely generated over $B$, then $C$ is finitely generated over $A$.
        \vspace{1mm}
        
        \textbf{Sketch of proof:} Let $b_1,\ldots,b_r$ generate $B$ over $A$ and $c_1,\ldots,c_s$ generated $C$ over $B$, then $b_ic_j$ generate $C$ over $A$.
    \end{enumerate}
\end{proof}
\begin{theorem}\label{1.4}
    $S$ is finitely generated over $R$ $\implies$ $S$ is integral over $R$.
\end{theorem}
\begin{proof}
    Let $\alpha_1,\ldots,\alpha_n$ generate $S$ as an $R$--module. Since we can always add elements, we may assume $\alpha_1=1$. Let $s \in S$. We will consider $m_s : S \to S$ given by $x \mapsto sx$ (multiplication by $s$). Then $s \alpha_i = \sum_{}^{} b_{ij}\alpha_j$ for some choice of $b_{ij} \in R$. Let $B=(b_{ij})$. By definition, $(sI-B)\begin{pmatrix} \alpha_1 \\ \vdots \\ \alpha_n \end{pmatrix} \stackrel{(\dagger)}{=}  0$. Now recall that for any matrix $X$, $\text{adj}(X)X = \det(X)$.\footnote{Recall that $\text{adj}(X)_{ij}$ is the determinant of the matrix we get when removing the $i^{\text{th}}$ row and $j^{\text{th}}$ column.} Multiply $(\dagger)$ by $\text{adj}(sI-B)$ to get $\det(sI-B)\begin{pmatrix} \alpha_1 \\ \vdots \\ \alpha_n \end{pmatrix} = 0$. In particular, $\det(sI-B)\alpha_1 = \det(sI-B) = 0$, so if we define $f(t)=\det(tI-B) \in R[t]$ which is a monic polynomial, then $f(s)=0$, so $s$ is integral over $R$.
\end{proof}
\begin{cor}
    If $\mathbb{Q} \subset L$ is a number field, then $\mathcal{O}_L$ is a ring.
\end{cor}
\begin{proof}
    If $\alpha,\beta \in L$ are algebraic integers, then $\mathbb{Z}[\alpha,\beta]$ is finitely generated over $\mathbb{Z}$ by Proposition \ref{1.3}, so it is integral over $\mathbb{Z}$ by Theorem \ref{1.4}, so as $\alpha \pm \beta, \alpha \beta \in \mathbb{Z}[\alpha,\beta]$, they are algebraic integers as well.
\end{proof}
\begin{cor}
    If $A \subset B \subset C$ are ring extensions with $B/A$ integral and $C/B$ integral, then $C/A$ is integral.
\end{cor}
\begin{proof}
    If $c \in C$, let $f(x)=\sum_{i=0}^{N} b_i x^i$ be the monic polynomial over $B[x]$ it satisfies. Set $B_0 = A[b_0,\ldots,b_{N}]$ and $C_0=B[c]$, then $B_0/A$ is finitely generated as $b_0,\ldots,b_N \in B$ are algebraic over $A$, $C_0$ is finitely generated over $B_0$ as $c$ is integral over $B_0$, hence $C_0$ is finitely generated over $A$, so $C$ is integral over $A$ by Theorem \ref{1.4}.
\end{proof}
\textbf{Remark.} $C$ could have had infinitely many generators, for example $C = \{\alpha \in \mathbb{C} \mid \alpha \text{ is an algebraic integer}\}$, which is why we passed to $C_0$.
\vspace{1mm}

\textbf{Exercise:} Show that $\mathcal{O}_{\mathbb{Q}[i]}=\mathbb{Z}[i]$, the ring of Gaussian integers.
\vspace{1mm}

If $K \subset L$ are fields, then recall that the minimal polynomial of $\alpha \in L$ is \underline{the monic} polynomial $p_{\alpha}(x) \in K[x]$ of minimal degree such that $p_{\alpha}(\alpha)=0$.
\begin{lemma}\label{1.7}
    If $f(x) \in K[x]$ has $f(\alpha)=0$, then $p_{\alpha} \mid f$.
\end{lemma}
\begin{proof}
    Euclid implies $f=p_{\alpha} h + r$ for $r \in K[x]$ with $\text{deg}(r)<\text{deg}(p_{\alpha})$. Hence $0 = f(\alpha) = p_{\alpha}(\alpha)h(\alpha)+r(\alpha)= r(\alpha)$. If $r \neq 0$, then this contradicts minimality of $\text{deg}(p_\alpha)$.
\end{proof}
The converse is obvious, and the lemma implies that $p_{\alpha}$ is unique.
\begin{prop}
    Let $L$ be a number field. Then $\alpha \in \mathcal{O}_L \iff$ the minimal polynomial $p_{\alpha}(x) \in \mathbb{Q}[x]$ of $\alpha$ is in $\mathbb{Z}[x]$.
\end{prop}
\begin{proof}
    $(\impliedby)$: By definition, since $\alpha$ is integral.
    \vspace{1mm}
    
    $(\implies)$: Let $\alpha \in \mathcal{O}_L$ with $p_{\alpha}$ the minimal polynomial. Let $M \supset L$ be a splitting field for $p_\alpha$, i.e. the field in which $p(x)=\prod_{i=1}^{n} (x-\alpha_i)$. Let $h(x)$ be a monic polynomial which $\alpha$ satisfies. By Lemma \ref{1.7}, $p_\alpha \mid h$, so each $\alpha_i \in M$ is an algebraic integer. But by Theorem \ref{1.4}, sums and products of algebraic integers are algebraic integers, so the coefficients of $p(x)$ are algebraic integers. But $p(x) \in \mathbb{Q}[x]$, so its coefficients are rational algebraic integers, hence they are integers.
\end{proof}
\textbf{Exercise:} Deduce the proposition from Lemma \ref{1.7} and Gauss' lemma.
\begin{lemma}
    The field of fractions of $\mathcal{O}_L$ is $L$. We even prove something stronger: if $\alpha \in L$, then $\exists n \in \mathbb{Z}$ such that $n \alpha \in \mathcal{O}_L$.
\end{lemma}
\begin{proof}
    Let $\alpha \in L$ and let $g$ be the minimal polynomial of $\alpha$ (which is monic and in $\mathbb{Q}[x]$). Then $\exists n \in \mathbb{Z}, n \neq 0$ such that $ng \in \mathbb{Z}[x]$, so $h(x)=n^{\text{deg}(g)}g\left(\frac{x}{n}\right) \in \mathbb{Z}[x]$ is monic, and this is the minimal polynomial of $n \alpha$, so $n \alpha \in \mathcal{O}_L$.
\end{proof}
 
\end{document}

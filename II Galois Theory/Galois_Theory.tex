\documentclass{article}
%build with recipe latexmk
\usepackage[utf8]{inputenc}
\usepackage[T1]{fontenc}
\usepackage{textcomp}
\usepackage{fancyhdr}
\pagestyle{fancy}
%\addtolength{\headwidth}{\marginparwidth}
%\addtolength{\headwidth}{\marginparsep}
%\addtolength{\headwidth}{\marginparsep}
\usepackage{tcolorbox}
\tcbuselibrary{theorems}
\usepackage{babel}
\usepackage{enumerate}
\usepackage{amsmath, amssymb, amsthm}
%\usepackage{a4wide}
\usepackage{float}
\usepackage{tikz-cd}
\usepackage{tikz}
\usepackage{graphicx}
\usepackage{wrapfig}
\graphicspath{ {./images/} }
\usepackage{setspace}
\setstretch{1.1}
\usepackage{color}
\usepackage{hyperref}
\hypersetup{
    colorlinks=true, %set true if you want colored links
    linktoc=all,     %set to all if you want both sections and subsections linked
    linkcolor=black,  %choose some color if you want links to stand out
}

\theoremstyle{definition}
\newtheorem{theorem}{Theorem}[section]
\newtheorem{lemma}[theorem]{Lemma}
\newtheorem{cor}[theorem]{Corollary}
\newtheorem{prop}[theorem]{Proposition}
\newtheorem{example}{Example}[section]
\newtheorem{defn}{Definition}[section]

\title{ Part II - Galois Theory
    \\ \large
Lectured by Prof. A. J. Scholl
}
\author{Artur Avameri}
\date{Michaelmas 2022}

% figure support
\usepackage{import}
\usepackage{xifthen}
\pdfminorversion=7
\usepackage{pdfpages}
\usepackage{transparent}
\newcommand{\incfig}[1]{%
    \def\svgwidth{\columnwidth}
    \import{./figures/}{#1.pdf_tex}
}

\pdfsuppresswarningpagegroup=1

\setcounter{section}{-1}
\begin{document}
\maketitle
\tableofcontents
\newpage

\section{Introduction}

\marginpar{06 Oct 2022, Lecture 1}


Galois Theory begins with polynomial equations and trying to solve them. Galois discovered certain \textbf{symmetries} of equations, which led to symmetries of fields (Steinitz, Artin).

\vspace{1mm}


Babylonians were able to solve the quadratic equation $X^2 + bX + c$ thousands of years ago, and so can we - write it as $(X + b/2)^2 + c - b^2/4$, which leads to the quadratic formula, or use Vieta's formulas to get $x_1x_2 = c, x_1 + x_2 = -b$, from which we can solve for $x_1$ by doing $x_1 = \frac{1}{2} \left( (x_1 + x_2) + (x_1 - x_2) \right)$ and $(x_1 - x_2)^2 = (x_1 + x_2)^2 - 4x_1x_2$.

\vspace{1mm}


A lot later people figured out how to solve the cubic equation, $X^3 + aX^2 +bX + c$. We get $x_1 + x_2 + x_3 = -a, x_1x_2 + x_2x_3 + x_3x_1 = b, x_1x_2x_3 = -c$. If we replace $X \mapsto X- a/3$, we end up with a cubic equation without a quadratic term. Now \[
x_1 = \frac{1}{3}\left[(x_1+x_2+x_3) + (x_1 + \omega x_2 + \omega^2 x_3) + (x_1 + \omega^2 x_2 + \omega x_3) \right]
\] for $\omega = e^{2\pi i/3}$ a cube root of unity. Let $u = (x_1 + \omega x_2 + \omega^2 x_3), v = (x_1 + \omega^2 x_2 + \omega x_3)$.

If we cyclically permute $x_1,x_2,x_3$, we find $u \mapsto \omega u \mapsto \omega^2 u$ and $v \mapsto \omega v \mapsto \omega^2 v$, so $u^3$ and $v^3$ are invariant under cyclic permutations of the roots. Hence $u^3+v^3$ and $u^3v^3$ are invariant under permutations of the roots, so (as we prove in the next lecture) we can express them in terms of the coefficients of the polynomial.

In fact, they're given by $u^3 + v^3 = -27c, u^3v^3 = -27b^2$, hence $u^3,v^3$ are roots of $Y^2 + 27cY - 27b^2$, from which we can find $u,v$ and hence $x_1$. This is \textbf{Cardano's formula}.

\vspace{1mm}

If we proceed similarly for quartics, we end up with a cubic equation which we can solve as above. Unfortunately, this doesn't work for quintics. The reason for this lies in group theory.

\newpage

\section{Polynomials}

In this course, all rings will be commutative, with a one, and nonzero. For a ring $R$, $R[X]$ is the ring of polynomials over $R$, i.e. just the formal expressions $\sum_{i=0}^{n} a_i X^i$ for $a_i \in R$.

A polynomial $f \in R[X]$ determines a \textbf{function} $R \to R$. However, the polynomial $r \mapsto f(r)$ isn't in general determined by the function. For example, if $R = \mathbb{Z}/p\mathbb{Z}$ for $p$ a prime, then $\forall a \in R, a^p = a$, so the polynomials $X^p$ and $X$ represent the same function, while being different polynomials.

In the case where $R = K$ is a field, we know $K[X]$ is a Euclidean domain, so it has a division algorithm: if $f,g \in K[X]$ and $g$ is nonzero, then there exist unique $q,r$ such that $f = gq + r$ and $\text{deg}(r) < \text{deg}(g)$ (note that $\text{deg}(0) = - \infty$). If $g = X-a$ is linear, then we get $f =(X-a)q + f(a)$, the \textbf{remainder theorem}.

$K[X]$ is also a PID and UFD, so every polynomial is a product of irreducible polynomials, and there are GCDs, which we can compute using Euclid's algorithm.

\begin{prop}
    If $K$ is a field and $f \in K[x]$ is nonzero, then $f$ has at most $\text{deg}(f)$ roots in $K$.\footnote{Note that this is not true if $K$ is a ring.}
\end{prop}
\begin{proof}
    If $f$ has no roots, we're done. Otherwise, let $f(a) = 0$ and write ${f = (X-a)g}$ with $\text{deg}(g) = \text{deg}(f) - 1$. But if $b$ is a root of $f$, then ${f(b) = 0} \implies b=a$ or $g(b)=0$, so $f$ has at most $(1 + \text{number of roots of g})$ roots and the claim follows by induction.
\end{proof}

\section{Symmetric polynomials}

Let $R$ be a ring and consider $R[X_1, \ldots, X_n]$ for some $n\ge 1$.
\begin{defn}
    A polynomial $f \in R[X_1,\ldots, X_n]$ is \textbf{symmetric} if for every permutation $\sigma \in S_n$, $f(X_{\sigma(1)}, \ldots, X_{\sigma(n)}) = f$.
\end{defn}

The set of symmetric polynomials is a subring of $R[X_1,\ldots,X_n]$.

\begin{example}
    $X_1 + \ldots + X_n$, or more generally, $P_k = \sum_{i=1}^{n} X_i^k$ are symmetric polynomials.
\end{example}

Alternative definition:
\begin{defn}
    If $f \in R[X_1,\ldots,X_n]$, define $f \sigma = f(X_{\sigma(1)}, \ldots, X_{\sigma(n)})$. This is a (right) action on the group $S_n$. We say $f$ is \textbf{symmetric}  if $f \sigma = f ~\forall \sigma \in S_n$.
\end{defn}

The \textbf{elementary symmetric polynomials} are \[
    s_r(X_1, \ldots, X_n) = \sum_{i_1 < \ldots < i_r}^{} X_{i_1}\ldots X_{i_r}.
\]
\begin{example}
    For $n=3$, $s_1 = X_1 + X_2 + X_3$, $s_2 = X_1X_2 + X_1X_3 + X_2X_3$, $s_3 = X_1X_2X_3$.
\end{example}

\begin{theorem}%theorem 2.1 in lectures
    \begin{enumerate}[(i)]
        \item Every symmetric polynomial over $R$ can be expressed as a polynomial in $\{s_r ~|~ 1\le r\le n\}$ with coefficients in $R$.
        \item There are no non-trivial relations between $s_1, \ldots, s_n$ - they're independent.
    \end{enumerate}
\end{theorem}

\marginpar{08 Oct 2022, Lecture 2}

\textbf{Remarks.} \begin{enumerate}[(a)]
    \item Consider the homomorphism $$\theta: R[Y_1,\ldots,Y_n] \to R[X_1,\ldots,X_n]$$ by ${\theta(Y_r) = S_r}$ (and identity on $R$). Then (i) says that the image of $\theta$ is the set of symmetric polynomials, and (ii) says that $\theta$ is injective.
    \item An equivalent definition of the $\{s_r\}$ is $$\prod_{i=1}^{n} (T+x_i) = T_n + s_1 T^{n-1} + \ldots + s_{n-1}T + s_n.$$
    \item If we need to specify the number of variables, we write $s_{r,n}$ instead of $s_r$.
\end{enumerate}

\begin{proof}[Proof of Theorem 2.1.]
    Terminology: 
    \begin{itemize}
        \item A \textbf{monomial} is some $X_{I} = X_1^{i_1}\ldots X_n^{i_n}$ for some $I \in \mathbb{Z}_{\ge 0}^n$.
        \item Its \textbf{(total) degree} is $\sum_{}^{} i_\alpha$.
        \item A \textbf{term} $\beta$ is some $c X_I, 0 \neq c \in R$, so a polynomial is uniquely a sum of terms.
        \item The total degree of $f$ is the maximal degree of any of the terms. 
    \end{itemize}   

    Define a lexicographical ordering on monomials $X_I$ as follows: $X_I > X_J$ if either $i_1 > j_1$ or for some $1 \le r < n$, $i_1=j_1,\ldots, i_r = j_r$ and $i_{r+1} > j_{r+1}$. This is a \textbf{total ordering}: for each pair $I \neq J$, exactly one of $X_I > X_J$ or $X_J > X_i$ holds.

    Existence: Let $d$ be the total degree of some symmetric polynomial $f$ and let $X_I$ be the lexicographically largest monomial in $f$ with coefficient $c \in R$. As $f$ is symmetric, we must have $i_1\ge i_2 \ge \ldots \ge i_n$ (if not, say $i_r < i_{r+1}$, then exchanging $X_r$ and $X_{r+1}$ gives a monomial occuring in $f$ which is bigger than $X_I$). 
    So $$X_I = X_1^{i_1-i_2}(X_1X_2)^{i_2-i_3}\ldots(X_1\ldots X_n)^{i_n}.$$
    Consider $g = s_1^{i_1-i_2}s_2^{i_2-i_3}\ldots s_{n-1}^{i_{n-1}-i_n}s_n^{i_n}$. The leading monomial (i.e. largest in lexicographical order) of $g$ is $X_I$, and $g$ is symmetric, so $f-cg$ is also symmetric, of total degree $\le d$, and its leading term is smaller (lexicographically) than $X_I$. As the set of monomials of degree $\le d$ is finite, this process terminates.

    \vspace{1mm}
    
    Uniqueness: By induction on $n$. Say $G \in R[Y_1,\ldots,Y_n]$ with $$G(s_{n,1},\ldots,s_{n,n}) = 0.$$ We want to show $G = 0$. If $n=1$, this is trivial ($s_{1,1} = X_1$). If $G = Y_n^k H$ with $Y_n \nmid H$, then $s_{n.n}^k H(s_{n,1},\ldots,s_{n,n}) = 0$. As $s_{n,n} = X_1\ldots X_n$, $s_{n,n}$ is not a zero divisor in $R[X_1,\ldots,X_n]$, hence $H(s_{1,n},\ldots,s_{n,n}) = 0$. So we may assume WLOG that $G$ is not divisible by $Y_n$.

    Replace $X_n$ by 0. Then $$s_{n,r}(X_1,\ldots,X_{n-1},0) = \begin{cases}
        s_{n-1,r}(X_1, \ldots, X_{n-1}) &\text{if }r<n \\
        0 &\text{if } r=n
    \end{cases}$$
    and so $G(s_{n-1,1},\ldots,s_{n-1,n-1},0) = 0$. So by induction, $G(Y_1, \ldots, Y_{n-1}, 0) = 0$, so $Y_n \mid G$, contradiction and we're done.
\end{proof}

\begin{example}
    Say $f = \sum_{i \neq j} X_i^2 X_j$ for some $n \ge 3$. Its leading term is ${X_1^2X_2 = X_1(X_1X_2)}$. Then $$s_1s_2 = \sum_{i}^{} \sum_{j<k}^{} X_iX_jX_k = \sum_{i \neq j}^{} X_i^2 X_j + 3 \sum_{i<j<k}^{} X_iX_jX_k.$$ 
    So $f = s_1s_2 - 3s_3$.
\end{example}

Computing, say $\sum_{}^{} X_i^5$ by hand is tedious. But there are formulae for this! Recall $p_k = \sum_{i=1}^{n} X_i^k$.

\begin{theorem}[Newton's formulae]
    Let $n\ge 1$. Then $~\forall k \ge 1$, $$p_k - s_1p_{k-1} + \ldots + (-1)^{k-1}s_{k-1}p_1 + (-1)^k k s_k = 0.$$
    (By convention, $s_0 = 1$ and $s_r = 0$ if $r>n$).
\end{theorem}
\begin{proof}
    We may assume $R = \mathbb{Z}$. Consider the generating function \[
    F(T) = \prod_{i=1}^{n} (1-X_iT) = \sum_{r=0}^{n} (-1)^r s_r T^r.
    \]
    Take the logarithmic derivative w.r.t $T$, i.e. \[
    \frac{F'(T)}{F(T)} = \sum_{i=1}^{n} \frac{-X_i}{1-X_iT} = -\frac{1}{T} \sum_{i=1}^{n} \sum_{r=1}^{\infty} X_i^r T^r = -\frac{1}{T} \sum_{r=1}^{\infty} p_r T^r.
    \]
    Thus $-TF'(T) = s_1T - 2s_2T^2 + \ldots + (-1)^{n-1}ns_nT^n$ from our generating function above, but we also have (from the previous line) that $$-TF'(T) = F(T)\sum_{r=1}^{\infty} p_r T^r = (s_0 - s_1T + \ldots + (-1)^n s_n T^n)(p_1 T + p_2 T^2 + \ldots).$$
    Comparing coefficients of $T^k$ gives the identity. 
\end{proof}

The \textbf{discriminant} polynomial is $D(X_1,\ldots,X_n) = \Delta(X_1,\ldots,X_n)^2$ where $\Delta = \prod_{i<j} (X_i-X_j)$. (Recall from IA Groups that applying $\sigma \in S_n$ to $\Delta$ multiplies $\Delta$ by $\text{sgn}(\sigma)$). So $D$ is symmetric. So $D(X_1,\ldots,X_n) = d(s_1,\ldots,s_n)$ for some polynomial $d$ (with coefficients in $\mathbb{Z}$).
\begin{example}
    If $n=2$, then $D = (X_1 - X_2)^2 = s_1^2 - 4s_2$.
\end{example}
\begin{defn}
    Let $f =T^n + \sum_{i=0}^{n-1} a_{n-i}T^i \in R[T]$ be monic. Then its \textbf{discriminant} is $\text{Disc}(f) = d(-a_1,a_2,-a_3,\ldots,(-1)^n a_n) \in R$.
\end{defn}
Observe that if $f = \prod_{i=1}^{n} (T-x_i), x_i \in R$, then $a_r = (-1)^r s_r(x_1,\ldots,x_n)$, so $\text{Disc}(f) = \prod_{i<j}^{} (x_i-x_j)^2 = D(x_1,\ldots,x_n)$. If moreover $R = K$ is a field, then $\text{Disc}(f) = 0$ if and only if $f$ has a repeated root (i.e. $x_i=x_j$ for some $i \neq j$).
\begin{example}
    $\text{Disc}(T^2+bT+c) = b^2 - 4c$.
\end{example}

\marginpar{11 Oct 2022, Lecture 3}

\section{Fields}

Recall that a \textbf{field}  is a ring $K$ (commutative, nonzero, with a 1) in which every nonzero element has a multiplicative inverse. The set of nonzero elements of $K$ is then a \textbf{group} $K^*$ (or $K^\times$), called the multiplicative group of $K$.
\vspace{1mm}

The \textbf{characteristic} of $K$ is the least positive integer $p$ (if it exists) such that $p \cdot 1_K = 0_K$, or 0 if no such $p$ exists.
For example, $\mathbb{Q}$ has characteristic $0$, and $\mathbb{F}_p = \mathbb{Z}/p\mathbb{Z}$ has characteristic $p$. 
\vspace{1mm}

The characteristic $\text{char}(K)$ of $K$ is always either 0 or prime. Inside $K$, there is a smallest subfield, called the \textbf{prime subfield} of $K$, which is either isomorphic to $\mathbb{Q}$ (if $\text{char}(K)=0$) or to $\mathbb{F}_p$ (if $\text{char}(K)=p$). 

\begin{prop}
    Let $\phi : K \to L$ be a homomorphism of fields. Then $\phi$ is an injection.
\end{prop}
\begin{proof}
    $\phi(1_K) = 1_L \neq 0_L$, so $\text{ker}(\phi) \subset K$ is a proper ideal of $K$, so $\text{ker}(\phi) = (0)$.
\end{proof}

\begin{defn}
    Let $K \subset L$ be fields (where the field operations on $K$ are the same as those in $L$). We say $K$ is a \textbf{subfield}  of $L$, and $L$ is an \textbf{extension} of $K$, denoted $L/K$, "$L$ over $K$".
\end{defn}
\textbf{Remarks.} (i) This has nothing to do with quotients.
\vspace{1mm}

(ii): It is useful to be more general - if $i : K \to L$ is a homomorphism of fields, then by Prop 3.1 $i$ is an isomorphism of $K$ and the subfield $i(K) \subset L$. In this situation, we also say that "$L$ is an extension of $K$".

\begin{example}
    We have extensions $\mathbb{C}/\mathbb{R}$, $\mathbb{R}/\mathbb{Q}$, $\mathbb{Q}[i] = \{a+bi ~|~ a,b \in \mathbb{Q}\}/ \mathbb{Q}$.
\end{example}

\textbf{Notation/definition.} Suppose we have two field $K \subset L$ and $x \in L$. Define $K[x] = \{p(x) ~|~ p \in K[T]\}$, the set of polynomials in $x$. This is a \textbf{subring} of $L$.
\vspace{1mm}

We also define $K(x) = \{\frac{p(x)}{q(x)} ~|~ p,q \in K[T], q(x) \neq 0\}$. This is a \textbf{subfield} of $L$ (read "$K$ adjoin $x$").
\vspace{1mm}

For $x_1,\ldots,x_n \in L$, similarly define $$K(x_1,\ldots,x_n) = \left\{\frac{p(x_1,\ldots,x_n)}{q(x_1,\ldots,x_n)} ~|~ p,q \in K[T_1,\ldots,T_n, q(x) \neq 0\right\}.$$

We can check that $K(x_1,\ldots,x_{n-1})(x_n) = K(x_1,\ldots,x_n)$, and likewise for $K[x_1,\ldots,x_n]$.

\vspace{1mm}

If we have $L / K$ a field extension, then $L$ is naturally a vector space over its subfield $K$ (just forget multiplication by elements of $L$). We can ask whether this is a \textbf{finite-dimensional} vector space.
\begin{itemize}
    \item If so, we say $L/K$ is a \textbf{finite extension} and write $[L : K] = \text{dim}_K(L)$ for the dimension. We call this the \textbf{degree} of the extension.
    \item If not, write $[L : K] = \infty$.
\end{itemize}

$\text{dim}_K$ is the dimension as a $K$-vector space. Since $L$ is a vector space over $L$, we have $\text{dim}_L(L) = 1$. As a $K$-vecor space, $L \cong K^{[L : K]}$.

\begin{example}
    \begin{enumerate}[(i)]
        \item $\mathbb{C}/\mathbb{R}$ is a finite extension with $[\mathbb{C} : \mathbb{R}] = 2$.
        \item Let $K$ be any field, $K(X)$ the field of rational functions in $X$, i.e. the field of fractions of the polynomial ring $K[X]$. Then $[K(X) : K] = \infty$ since $1,x,x^2,\ldots$ are linearly independent.
        \item $[\mathbb{R}:\mathbb{Q}] = \infty$ (use countability: every finite dimensional $\mathbb{Q}$-vector space is countable).
    \end{enumerate}
\end{example}
This course is largely about preperties (and symmetries) of \textbf{finite} field extensions.

\begin{defn}
    We say an extension $L/K$ is \textbf{quadratic} if $[L:K] = 2$. Similarly for \textbf{cubic}, etc.
\end{defn}
\begin{prop}
    Suppose $K$ is a \textbf{finite} field (necessarily of characteristic $p>0$). Then the number of elements of $K$ is a power of $p$.
\end{prop}
\begin{proof}
    Certainly $K/\mathbb{F}_p$ is finite, so $K \cong (\mathbb{F}_p)^n$ for $n = [K : \mathbb{F}_p]$, so $|K| = p^n$.
\end{proof}
Later we will show that for any prime power $q = p^n$ there exists a finite field $\mathbb{F}_q$ with $q$ elements. We have $\mathbb{F}_p = \mathbb{Z}/p\mathbb{Z}$, but $\mathbb{F}_{p^n} \neq \mathbb{Z}/p^n\mathbb{Z}$ if $n>1$.
\vspace{1mm}

A simple, yet powerful fact:
\begin{theorem}[Tower law]
    Suppose we have two field extensions $M/L$ and $L/K$. Then $M/K$ is a finite extension if and only if both $M/L$ and $L/K$ are finite, and if so, then \[
    [M : K] = [M : L][L : K].
    \]
\end{theorem}
In fact, a slightly more general statement by taking $V=M$ in the above:
\begin{theorem}
    Let $L/K$ be a field extension, $V$ a $L$-vector space. Then $$\text{dim}_K V = [L : K] \cdot \text{dim}_L V$$
    (with the obvious meaning if any of these are infinite).
\end{theorem}
\begin{example}
    $V = \mathbb{C}^{n}= \mathbb{R}^{2n}$.
\end{example}
\begin{proof}
    Let $\text{dim}_L V = d < \infty$. Then $V \cong L \oplus \ldots \oplus L = L^d$ as a $L$-vector space, so also certainly as a $K$-vector space. If $[L:K] = n < \infty$, then $L \cong K^n$ as a $K$-vector space, so $V = K^n \oplus \ldots \oplus K^n = K^{nd}$, so $\text{dim}_K V = [L : K] \cdot \text{dim}_L V$.
    \vspace{1mm}
    
    If $V$ is finite-dimensional over $K$, then a $K$-basis for $V$ certainly spans $V$ over $L$. So if $\text{dim}_L V = \infty$, then $\text{dim}_K V = \infty$. Likewise, if $[L:K] = \infty$ and $V \neq \emptyset$, then $V$ has a infinite linearly independent subst, so $\text{dim}_K V = \infty$.
\end{proof}
Another important fact:
\begin{prop}
    \begin{enumerate}[(i)]
        \item Let $K$ be a field and $G \subset K^\times$ a \textbf{finite} subgroup. Then $G$ is \textbf{cyclic}.
        \item If $K$ is finite, then $K^{\times}$ is cyclic.
    \end{enumerate}
\end{prop}
\begin{proof}
    (i): Write $G \cong \mathbb{Z}/m_1\mathbb{Z} \times \ldots \times \mathbb{Z}/m_k\mathbb{Z}$ as a product of cyclic groups such that $1<m_1 \mid m_2\mid \ldots\mid m_k = m$ (by GRM). So $~\forall x \in G, x^m = 1$. As $K$ is a field, the polynomial $T^m-1$ has at most $m$ roots. So $|G|\le m$, so $k=1$, and hence $G$ is cyclic.

    (ii) is now obvious.
\end{proof}
\textbf{Remark.} If $K=\mathbb{F}_p = \mathbb{Z}/p\mathbb{Z}$, the above says $\exists a \in \{1,\ldots,p-1\}$ such that $\mathbb{Z}/p\mathbb{Z} = \{0\} \cup \{a,a^2,\ldots,a^{p-1} \pmod{p}\}$. This $a$ is called a \textbf{primitive root} mod $p$. 

\marginpar{13 Oct 2022, Lecture 4}

\begin{prop}
    Let $R$ be a ring and $p$ a prime such that $p 1_{R} = 0_R$ (e.g. $R$ is a field of characteristic $p$). Then the map $$\phi_p: R \to R \text{ by }  \phi_p(x) = x^p$$ is a \textbf{homomorphism} from $R$ to itself, called the \textbf{Frobenius endomorphism} of $R$. 
\end{prop}
\begin{proof}
    We have to show that $\phi_q(1)=1, \phi_p(xy)= \phi_p(x)\phi_p(y)$ and $\phi_p(x+y)= \phi_p(x) + \phi_p(y)$. But the first two are obvious, and for the last one we get \[
    \phi_p(x+y) = (x+y)^{p} + \sum_{i=1}^{p-1} {{p} \choose {i}} x^i y^{p-i} + y^p = x^p + y^p
    ,\]
    where all the terms ${{p}\choose{i}}$ are divisible by $p$ as $p$ is a prime.
\end{proof}
\textbf{Remark.} This is a very important map. For example, this gives another proof of Fermat's little theorem $x^p \equiv x \pmod{p}$: induction on $x$ and $(x+1)^p \equiv x^p + 1 \pmod{p}$.

\section{Algebraic elements and extensions}
Let $L/K$ be an extension and $x \in L$. 
\begin{defn}
    $x$ is \textbf{algebraic} over $K$ if $\exists$ a nonzero polynomial $f \in K[T]$ such that $f(x)=0$. If $x$ is not algebraic, we say it is \textbf{transcendental over} $K$.
\end{defn}
Suppose $f \in K[T]$ with evaluation $f(x) \in L$. This gives a map $$\text{ev}_x : K[T] \to L, f \mapsto f(x),$$ which is obviously a homomorphism of rings.
\vspace{1mm}

$I = \text{ker}(\text{ev}_x) \subset K[T]$ is an ideal ( $= \{f ~|~ f(x)=0\}$). As $\text{Im}(\text{ev}_x)$ is a subring of $L$, it is an integral domain. So $I$ is a \textbf{prime} ideal, so there are two possibilities:
\begin{enumerate}[(i)]
    \item $I = \{0\} \implies $ the only $f$ with $f(x)=0$ is $f=0$, so $x$ is transcendental over $K$.
    \item $I \neq \{0\}$. AS $K[T]$ is a PID, there exists a unique monic irreducible $g \in K[T]$ such that $I = (g)$. So $f(x) = 0 \iff f$ is a multiple of $g$. So $x$ is algebraic over $K$ and we call $g$ the \textbf{minimal polynomial} of $x$ over $K$, which we might write as $m_{x,K}$. It is the unique irreducible monic polynomial with $x$ as a root (and is the monic polynomial of least degree with $x$ as a root - this depends on $K$ as well as $x$). 
\end{enumerate}
Some examples:
\begin{itemize}
    \item $x \in K$, $m_{x,K} = T-x$.
    \item $p$ a prime, $d \ge 1$. Then $T^d - p \in \mathbb{Q}[T]$ is irreducible by Eisenstein's criterion, so it is the min. poly. of $\sqrt[d]{p} = x$ over $\mathbb{Q}$.
    \item $z = e^{2 \pi i/p}$ for $p$ a prime is a root of $T^p - 1$ and $$\frac{T^p-1}{T-1} = g(T) = T^{p-1} + \ldots + T + 1 \in \mathbb{Q}[T].$$
    As $g(T+1) = \frac{(T+1)^p - 1}{T} = T^{p-1} + {{p} \choose {1}} T^{p-2} + \ldots + pT + p$, this is also irreducible by Eisenstein and hence $g$ is the min. poly. of $z$ over $\mathbb{Q}$.
\end{itemize}
\textbf{Terminology.} We say \textbf{the degree of $x$ over $K$} (where $x$ is algebraic over $K$) is the degree of $m_{x,K}$, written $\text{deg}_K(x)$ or $\text{deg}(x/K)$.
\vspace{1mm}

A ring/field-theoretic characterization of the notion of being algebraic: 
\begin{prop}
    Let $L/K, x \in L$. The following are equivalent:
    \begin{enumerate}[(i)]
        \item $x$ is algebraic over $K$.
        \item $[K(x) : K] < \infty$.
        \item $\text{dim}_K K[x] < \infty$.
        \item $K[x] = K(x)$.
        \item $K[x]$ is a field.
    \end{enumerate}
    If these hold, then $\text{deg}_K(x) = [K(x) : K]$.
\end{prop}
Recall $K[X] = \{p(x)\}$ and $K(x) = \{\frac{p(x)}{q(x)}~|~ q(x) \neq 0\}$ for $p,q \in K[T]$. The most important results here are (i) $\iff$ (ii) and the degree formula. (This is a part of a series of results relating properties of $x$ and $K(x)$).
\begin{proof}
    (ii) $\implies $ (iii) and (iv) $\iff$ (v) are trivial.
    \vspace{1mm}
    
    (iii) $\implies$ (iv) and (ii): Let $0 \neq y = g(x) \in K[x]$. Consider $K[x] \to K[x]$ by $z \mapsto yz$. It is a $K$-linear transformation, it is injective as $y\neq 0$. As $\text{dim}_K K[X] < \infty$, it is bijective. So $\exists $ s.t. $yz = 1$. So $K[x]$ is a field, equal to $K(x)$, and $[K(x):K]$ is finite-dimensional.
    \vspace{1mm}
    
    (v) $\implies$ (i): WLOG $x\neq 0$, then $x^{-1} = a_0 + a_1x + \ldots + a_n x^n \in K[X]$ for $a_i$ not all equal to $0$, so $a_nx^{n+1} + \ldots + a_0x - 1 =0$, so $x$ is algebraic over $K$.
    \vspace{1mm}
    
    (i) $\implies$ (iii) and the degree formula: The image of $\text{ev}_x : K[T] \to L$ is $K[X] \subset L$. $x$ is algebraic over $K \implies \text{ker}(\text{ev}_x) = (m_{x,K})$ is a maximal ideal (GRM, because $m$ is irreducible), so by the first isomorphism theorem, $K[T]/(m_{x,K}) \cong K[x]$. The LHS is a field, so $K[X]$ is a field. $m_{x,K}$ is monic of degree $d = \text{deg}_K(x)$, so $K[T]/(m_{x,K})$ has a $K$-basis $1,T,\ldots,T^{d-1}$. Hence $\text{dim}_K K[x] = d < \infty$ (this gives (iii)) and so $[K(x) : K] = d$ as well.
\end{proof}
\begin{cor}
    \begin{enumerate}[(i)]
        \item The elements $x_1,\ldots,x_n$ are all algebraic over $K$ if and only if ${L = K(x_1,\ldots,x_n)}$ is a finite extension of $K$. If so, then \textbf{every} element of $L$ is algebraic over $K$.
        \item If $x,y$ are algebraic over $K$, then so are $x+y$, $xy$, and $1/x$ (if $x \neq 0$).
        \item Let $L/K$ be any extension. Then $\{x \in L ~|~ x \text{ algebraic over }K\}$ is a subfield of $L$.
    \end{enumerate}
\end{cor}
\begin{proof}
    \begin{enumerate}[(i)]
        \item If $x_n$ is algebraic over $K$, it is certainly algebraic over $K(x_1,\ldots,x_{n-1})$, so $[L : K(x_1,\ldots,x_{n-1})] < \infty$. So by tower law and induction on $n$, $[L:K] < \infty$. Conversely, if $[L:K] < \infty$, then the subfield $K(y)$ is finite over $K$ for all $y$ in $L$. So $y$ is algebraic over $K$ by the previous proposition. 
        \item $x \pm y, xy, \frac{1}{x} \in K(x,y)$, so by (i), every element of this field is algebraic.
        \item This clearly follows from (ii).
    \end{enumerate}
\end{proof}
\textbf{Remark.} The key ingredient here is the tower law.

\marginpar{15 Oct 2022, Lecture 5}

\begin{example}
We saw earlier that $z = e^{2 \pi i /p}$ for $p$ an odd prime has min. poly. of degree $p-1$.

Consider now $x = 2\cos\frac{2\pi}{p} = z + z^{-1} \in \mathbb{Q}(z)$ (so $x$ is algebraic over $\mathbb{Q}$).

We have $\mathbb{Q}(z) \supset \mathbb{Q}(x) \supset \mathbb{Q}$, and $z^2-xz+1 = 0$. So $\text{deg}_{\mathbb{Q}(x)}(z) \le 2$, and we know $[\mathbb{Q}(z) : \mathbb{Q}] = p-1$, so $[\mathbb{Q}(z) : \mathbb{Q}(x)]$ is either 1 or 2.

But $z \not\in \mathbb{Q}(x) \subset \mathbb{R}$, so $[\mathbb{Q}(z) : \mathbb{Q}(x)] = 2$ and hence $\text{deg}_{\mathbb{Q}}(x) = \frac{p-1}{2}$.

To actually find this polynomial, write $$z^{\frac{p-1}{2}}+z^{\frac{p-3}{2}} + \ldots + z^{\frac{-(p-1)}{2}} = 0,$$ which remains unchanged under $z \mapsto \frac{1}{z}$, and hence we can express the above polynomial in terms of $z + \frac{1}{z} = x$ as a polynomial of degree $\frac{p-1}{2}$.
\end{example}
\begin{example}
    $x = \sqrt{m} + \sqrt{n}$ for $m,n \in \mathbb{Z}$, $m,n,mn$ not squares. We have \[
    n = (x - \sqrt{m})^2 \stackrel{\star}{=}  x^2 - 2\sqrt{m}x + m
    ,\] so $[\mathbb{Q}(x) : \mathbb{Q}(\sqrt{m})] \le 2$. Similarly, $[\mathbb{Q}(x) : \mathbb{Q}(\sqrt{n})] \le 2$. Also note that $\star$ implies that $\sqrt{m} \in \mathbb{Q}(x)$.
    
    So (by the tower law), either $[\mathbb{Q}(x):\mathbb{Q}]=4$, or $[\mathbb{Q}(x):\mathbb{Q}]=2$ and $\mathbb{Q}(x)=\mathbb{Q}(m)=\mathbb{Q}(n)$ (since $m,n$ not squares implies $[\mathbb{Q}(m): \mathbb{Q}] = [\mathbb{Q}(n): \mathbb{Q}] = 2$). But then $\mathbb{Q}(m)=\mathbb{Q}(n) \implies \sqrt{m}=a+b\sqrt{n}$ for $a,b \in \mathbb{Q} \implies m = a^2 + b^2 n + 2ab \sqrt{n}$. So $ab=0$, whence either $b=0$, so $m=a^2$ is a square, or $a=0$, so $mn=b^2n^2$ is a square. This forces $[\mathbb{Q}(x): \mathbb{Q}] = 4$.
\end{example}

\begin{defn}
    An extension $[L:K]$ is \textbf{algebraic} if every $x \in L$ is algebraic over $K$.
\end{defn}
\begin{prop}
    \begin{enumerate}[(i)]
        \item Finite extensions are algebraic.
        \item $K(x)$ is algebraic over $K$ if and only if $x$ is algebraic over $K$.
        \item If $M/L/K$, then $M/K$ is algebraic if and only if both $M/L$ and $L/K$ are algebraic.
    \end{enumerate}
\end{prop}
\begin{proof}
    \begin{enumerate}[(i)]
        \item $[L:K] <\infty \implies ~\forall x \in L, [K(x):K] <\infty \implies x$ is algebraic over $K$.
        \item $\implies$ follows by definition, $\impliedby$ follows by (i).
        \item Assume $M/K$ is algebraic. Then $~\forall x \in M$, $x$ is algebraic over $K$, so it is certainly algebraic over $L$. So $M/L$ is algebraic. As $L \subset M$, $L$ is algebraic over $K$.
        
        The other direction follows from the following lemma:
        \begin{lemma}
            Suppose we have $M/L/K$ with $L/K$ algebraic. Let $x \in M$, and suppose $X$ is algebraic over $L$. Then $x$ is algebraic over $K$.
        \end{lemma}
        \begin{proof}
            $\exists f = T^n + a_{n-1}T^{n-1} + \ldots + a_0 \in L[T]$ with $f \neq 0$ and $f(x) = 0$. Let $L_0 = K(a_0,\ldots,a_{n-1})$. As each $a_i$ is algebraic over $K$, by Corollary 4.2, $[L_0:K]$ is finite. As $f \in L_0[T]$, $x$ is algebraic over $L_0$. So $[L_0(x) : L_0]< \infty$, so $[L_0(x) : K] < \infty$ by the tower law, so $[K(x) : K] < \infty$ and we're done.
        \end{proof}
    \end{enumerate}
\end{proof}

\begin{example}
    Say $K = \mathbb{Q}, L = \{x \in \mathbb{C} ~|~ x \text{ is algebraic over }\mathbb{Q}\}$, usually written $\overline{\mathbb{Q}}$.
    Obviously $L/\mathbb{Q}$ is algebraic, but it is not finite - for every $n\ge 1, \sqrt[n]{2} \in L$, and so $[\mathbb{Q}(\sqrt[n]{2}): \mathbb{Q}] = n$ (as $T^n-2$ is irreducible over $\mathbb{Q}$). So as this holds for all $n$, $L$ cannot be finite over $\mathbb{Q}$. 
\end{example}
We will see other fields like $\overline{\mathbb{Q}}$ later on. They are called \textbf{algebraically closed fields}.

\section{Algebraic numbers in $\mathbb{R}$ and $\mathbb{C}$}
Traditionally, we say that $x \in \mathbb{C}$ is \textbf{algebraic}  if it is algebraic over $\mathbb{Q}$. Otherwise, we say it's transcendental. $\overline{\mathbb{Q}} = \{\text{algebraic }x\}$ is a subfield of $\mathbb{C}$. It is easy to see that $\overline{\mathbb{Q}} \subsetneq \mathbb{C}$, as $\mathbb{Q}[T]$ and hence $\overline{\mathbb{Q}}$ are countable, while $\mathbb{C}$ is uncountable. So in a sense, basically all complex numbers are transcendental. However, it is a lot harder to write one down explicitly, or to show that some given number is transcendental.

Aside: some history. Liouville showed that $\sum_{n\ge 1}^{} \frac{1}{10^{n!}}$ is transcendental ("algebraic numbers can't be very well approximated by rationals").

Hermite, Lindemann: $e$ and $\pi$ are transcendental.

Gelfond-Schneider ($20^{\text{th}}$ century): if $x,y$ are algebraic ($x\neq 0,1$), then $x^y$ is algebraic if and only if $y$ is rational (e.g. $\sqrt{2}^{\sqrt{3}}$ is transcendental, and $e^{\pi} = (-1)^{-i/2}$ is transcendental). End of aside.

\end{document}